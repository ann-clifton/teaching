\documentclass[12pt]{amsart}
\usepackage{amsmath}
\usepackage{enumerate}
\usepackage{multicol}
%\openup 5pt

\author[Math 111]{Blake Farman\\University of South Carolina}
\title[Final Exam]{Math 111\\Final Exam}
\date{December 12, 2017}

\pdfpagewidth 8.5in
\pdfpageheight 11in
\usepackage[margin=.5in]{geometry}

\theoremstyle{definition}
\newtheorem{thm}{}
\renewcommand{\qedsymbol}{}

\begin{document}
\maketitle

\begin{center}
  \fbox{\fbox{\parbox{5.5in}{\centering
        Answer the questions in the spaces provided on the
        question sheets and turn them in at the end of the exam period.
        If you require extra space, use the back of the page and indicate that you have done so.
        
        Unless otherwise stated, all supporting work is required.
        Unsupported or otherwise mysterious answers will \textbf{not receive credit.}
        
        You may use a calculator, but you may \textbf{not} use a Computer Algebra System (CAS)
        or any other electronic device whatsoever, \textbf{including cell phones.}
        By writing your name on the line below, you indicate that you have read and understand these directions.}}}
\end{center}

\vspace{0.2in}
\makebox[\textwidth]{Name:\enspace\hrulefill}
\vspace{0.2in}

$$
\begin{array}{|c|c|c||c|c|c|}
  \hline
  \textbf{Definition} & \textbf{Points Earned} & \textbf{Points Possible} & \textbf{Problem} & \textbf{Points Earned} & \textbf{Points Possible}\\
  \hline
  1 & & 3 & 1 & & 5\\
  \hline
  2 & & 6 & 2 & & 5\\
  \hline
  3 & & 2 & 3 & & 5\\
  \hline
  4 & & 1 & 4 & & 10\\
  \hline
  5 & & 1 & 5 & & 10 \\
  \hline
  6 & & 4 & 6 & & 10\\
  \hline
  7 & & 1 & 7 & & 15\\
  \hline
  8 & & 4 & 8 & & 15\\
  \hline
  9 & & 3 & \textbf{Subtotal} & & 75\\
  \hline
  \textbf{Subtotal} & & 25 & \textbf{Total} & & 100\\
  \hline
\end{array}
$$

\newpage

\section{Definitions}

\begin{thm}[3 Points]\label{def: special factorizations}
  Fill in the blanks with the correct factorizations.
  \vspace{.4in}
  \begin{enumerate}[(a)]
  \item
    $\displaystyle{A^2 - B^2 =\ \line(1,0){200}}$
    \vspace{.5in}
  \item
    $\displaystyle{A^2 + 2AB + B^2 =\ \line(1,0){200}}$
    \vspace{.5in}
  \item
    $\displaystyle{A^2 - 2AB + B^2 =\ \line(1,0){200}}$
    \vspace{.5in}
  \end{enumerate}
\end{thm}

\begin{thm}[6 Points]\label{def: exponents}
  Let $a, b$ be non-zero real numbers and $m, n$ rational numbers.
  Fill in the blanks
  \vspace{.15in}
  \begin{multicols}{2}
    \begin{enumerate}[(a)]
    \item
      $\displaystyle{a^0 =\ \line(1,0){100}}$
      \vspace{.4in}
    \item
      $\displaystyle{a^{-n} =\ \line(1,0){100}}$
      \vspace{.3in}
    \item
      $\displaystyle{a^m \cdot a^n =\ \line(1,0){100}}$
    \item
      $\displaystyle{\frac{a^m}{a^n}=\ \line(1,0){100}}$
      \vspace{.25in}
    \item
      $\displaystyle{\left(a \cdot b\right)^n=\ \line(1,0){100}}$
      \vspace{.25in}
    \item
      $\displaystyle{\left(\frac{a}{b}\right)^n=\ \line(1,0){100}}$
    \end{enumerate}
  \end{multicols}
\end{thm}

\begin{thm}[2 Points]
  \begin{enumerate}[(a)]
  \item
    State the Point-Slope form of a line passing through the point $(x_0, y_0)$ with slope $m$.
    \vspace{1in}
  \item
    State the Slope-Intercept form of a line with slope $m$ and $y$-intercept $b$.
    \vspace{1in}
  \end{enumerate}
\end{thm}

\begin{thm}[1 Point]\label{def: quadratic formula}
  Given a Quadratic Equation, $ax^2 + bx + c = 0$, the solutions are given by the Quadratic Formula.  State the Quadratic Formula.
  \vspace{1in}
\end{thm}

\newpage

\begin{thm}[1 Points]
  Let $f(x)$ be a function.
  State the average rate of change of $f$ between $x = a$ and $x = b$.
  \vspace{1in}
\end{thm}

\begin{thm}[4 Points]
  \begin{enumerate}[(a)]
  \item
    State the general form of an exponential function.
    Use $C$ to represent the initial value, $a$ to represent the growth/decay factor, and $t$ to represent the independent variable.
    \vspace{1in}
  \item
    Express the growth/decay rate, $r$, in terms of the growth/decay factor, $a$.
    \vspace{1in}
  \item
    When does such a function model exponential growth?
    \vspace{1in}
  \item
    When does such a function model exponential decay?
    \vspace{1in}
  \end{enumerate}
\end{thm}

\begin{thm}[1 Points]
  Let $a$ be a fixed positive number.
  The base $a$ logarithm of $x$ is defined by
  \vspace{.15in}
  $$\log_a(x) = y\  \text{if and only if}\ \ \line(1,0){200}$$
\end{thm}

\begin{thm}[4 Points]
  Let $a$ be a positive number.  Fill in the blanks.
  \begin{enumerate}[(a)]
  \item
    \vspace{.5in}
    $\displaystyle{\log_a(1) = \ \line(1,0){200}}$.
    \vspace{.5in}
  \item
    $\displaystyle{\log_a(a) = \ \line(1,0){200}}$.
    \vspace{.5in}
  \item
    $\displaystyle{\log_a(a^x) = \ \line(1,0){200}}$.
    \vspace{.5in}
  \item
    $\displaystyle{a^{\log_a(x)} = \ \line(1,0){200}}$.
  \end{enumerate}
  \vspace{.5in}
\end{thm}

\begin{thm}[3 Points]
  Let $0 < a$ and $C$ be fixed numbers.  Fill in the blanks.
  \begin{enumerate}[(a)]
  \item
    \vspace{.5in}
    $\displaystyle{\log_a(xy) = \ \line(1,0){200}}$.
    \vspace{.5in}
  \item
    $\displaystyle{\log_a\left(\frac{x}{y}\right) = \ \line(1,0){200}}$.
    \vspace{.5in}
  \item
    $\displaystyle{\log_a(x^C) = \ \line(1,0){200}}$.
  \end{enumerate}
  \vspace{.25in}
\end{thm}

\section{Problems}

\setcounter{thm}{0}
\begin{thm}[5 Points]\label{problem: quadratic}
  Let $f(x) = 3x^2 - 18x + 24.$
  \begin{enumerate}[(a)]
  \item
    By completing the square, put $f(x)$ into vertex form.
    \vspace{2in}
  \item
    Solve $f(x) = 0$ for $x$.
    \vspace{2in}
  \item
    What is the $y$-intercept of $f(x)$?
    \vspace{.5in}
  \end{enumerate}
\end{thm}

\newpage

\begin{thm}[5 point]\label{problem: average rate of change}
  Find the average rate of change of $f(x) = 3x^2 - 18x + 24$ from $x = 3$ to $x = 5$, then write down the Slope-Intercept Form of the line passing through the points $(3,f(3))$ and $(5,f(5))$.
  \vspace{2in}
\end{thm}

\begin{thm}[5 Points]\label{problem: construct tangent line}
  Write down the Slope-Intercept Form of the line passing through the point $(4,0)$ and parallel to the line in Problem~\ref{problem: average rate of change}.
  You should be careful to check that your answer is correct.
  \vspace{2in}
\end{thm}

\begin{thm}[10 Points]\label{problem: intersection}
  Find the points of intersection between the parabola $f(x) = 3x^2 - 18x + 24$ and the line in your answer to Problem~\ref{problem: construct tangent line}.
  Sketch a graph of these two functions together.
  Label the $x$-intercepts of both functions, the $y$-intercepts of both functions, the vertex of $f(x)$, and the points of intersection.
\end{thm}

\newpage

\begin{thm}[10 Points]\label{problem: domain}
  Let $f(x) = \sqrt{x}$ and $g(x) = x^2 - 9$.
  \begin{enumerate}[(a)]
  \item
    Compute the composition $(f \circ g)(x)$.
    What is the domain of this function?
    \vspace{2in}
  \item
    Compute the composition $(g \circ f)(x)$.
    What is the domain of this function?
    \vspace{2in}
  \end{enumerate}
\end{thm}

\begin{thm}[10 Points]\label{problem: exponential function}
  A biologist observes a population with initial size 400 over a sixth month period and measures that over this period of time the population grows at a rate of 50\%.
  Assuming that the population grows exponentially, find a function $P$ of $t$ years that models the size of the population.
  Use your model to predict the size of the population after 1 year.
\end{thm}

\newpage

\begin{thm}[15 Points]\label{problem: logarithm equation}
    \begin{enumerate}[(a)]
  \item
    Simplify the expression 
    $$\log_4(x + 3) + \log_4(x - 3).$$
    \vspace{2in}
  \item
    Solve the following equation for $x$
    $$\log_4(x + 3) + \log_4(x - 3) = 2$$
    \vspace{2in}
  \end{enumerate}
\end{thm}

\begin{thm}[15 Points]\label{problem: exponential equation}
  Solve the following equation for $x$.
  $$e^{x} = \left(\frac{e^{\sqrt{x}}}{e^{-\frac{1}{2}}}\right)^{-2}$$
\end{thm}
\end{document}
