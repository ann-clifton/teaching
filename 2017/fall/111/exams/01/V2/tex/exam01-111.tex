\documentclass[12pt]{amsart}
\usepackage{amsmath,amsthm,amssymb,amsfonts,enumerate,mymath,tikz-cd,fancyhdr,multicol}
\openup 5pt
\author{Blake Farman\\University of South Carolina}
\title{Math 111\\ Exam 01}
\date{September 28, 2017}
\pdfpagewidth 8.5in
\pdfpageheight 11in
\usepackage[margin=1in]{geometry}

\renewcommand{\qedsymbol}{}

\begin{document}
\maketitle

\begin{center}
  \fbox{\fbox{\parbox{5.5in}{\centering
        Answer the questions in the spaces provided on the
        question sheets and turn them in at the end of the class period.
        If you require extra space, use the back of the page and indicate that you have done so.
        
        Unless otherwise stated, all supporting work is required.
        Unsupported or otherwise mysterious answers will {\bf not receive credit.}
        
        A calculator is not necessary (or likely even helpful), but you may use one if you like.
        You may {\bf not} use a Computer Algebra System (CAS)
        or any other electronic device whatsoever, {\bf including cell phones.}}}}
\end{center}

\vspace{0.2in}
\makebox[\textwidth]{Name:\enspace\hrulefill}
\vspace{0.2in}

$$
\begin{array}{|c|c|c|}
  \hline
  \text{Problem} & \text{Points Earned} & \text{Points Possible}\\
  \hline
  1 & & 3\\
  \hline
  2 & & 6\\
  \hline
  3 & & 2\\
  \hline
  4 & & 3\\
  \hline
  5 & & 3 \\
  \hline
  6 & & 3\\
  \hline
  7 & & 16\\
  \hline
  8 & & 16\\
  \hline
  9 & & 16\\
  \hline
  10 & & 16\\
  \hline
  11 & & 16\\
  \hline
  \text{Bonus} & & 10\\
  \hline
  \text{Total} & & 100\\
  \hline
\end{array}
$$

\newpage

\theoremstyle{definition}
\newtheorem{thm}{}
\newtheorem{defn}{Definition}

\section{Definitions}
\begin{thm}[3 Points]\label{ex1}
  Fill in the blanks with the correct factorizations.
  \begin{enumerate}[(a)]
  \item
    $\displaystyle{A^2 - B^2 =\ \line(1,0){100}}.$
    \vspace{.3in}
  \item
    $\displaystyle{A^2 + 2AB + B^2 =\ \line(1,0){100}}.$
    \vspace{.3in}
  \item
    $\displaystyle{A^2 - 2AB + B^2 =\ \line(1,0){100}}.$
    \vspace{.3in}
  \end{enumerate}
\end{thm}
\vspace{.5in}
\begin{thm}[6 Points]\label{ex2}
  Let $a, b$ be non-zero real numbers and $m, n$ integers.
  Fill in the blanks
  \vspace{.25in}
  \begin{multicols}{2}
    \begin{enumerate}
    \item
      $\displaystyle{a^0 =\ \line(1,0){100}}$
      \vspace{.4in}
    \item
      $\displaystyle{a^{-n} =\ \line(1,0){100}}$
      \vspace{.3in}
    \item
      $\displaystyle{a^m \cdot a^n =\ \line(1,0){100}}$
    \item
      $\displaystyle{\frac{a^m}{a^n}\ \line(1,0){100}}$
      \vspace{.25in}
    \item
      $\displaystyle{\left(a \cdot b\right)^n\ \line(1,0){100}}$
      \vspace{.25in}
    \item
      $\displaystyle{\left(\frac{a}{b}\right)^n\ \line(1,0){100}}$
    \end{enumerate}
  \end{multicols}
\end{thm}
\vspace{.5in}
\begin{thm}[2 Points]\label{ex3}
  Given a Quadratic Equation, $ax^2 + bx + c = 0$, the solutions are given by the Quadratic Formula.  State the Quadratic Formula.
  \vspace{1in}
\end{thm}

\newpage

\begin{thm}[3 Points]\label{ex4}
  Fill in the blanks:\\
  \begin{center}
    To make $x^2 + bx$ a perfect square, add\ \line(1,0){100}.
    This gives the perfect square
    \vspace{.5in}
    $$x^2 + bx +\ \line(1,0){100} = \left(x +\ \line(1,0){100}\right)^2.$$
  \end{center}
\end{thm}
\vspace{.5in}
\begin{thm}[3 Points]\label{ex5}
  Fill in the blanks:\\
  \begin{center}
    An equation in variables $x$ and $y$ defines the variable $y$ as function of the variable $x$ if\\
    \vspace{.2in}
    each value of\ \line(1,0){40}\ corresponds to exactly\ \line(1,0){40}\ value of\ \line(1,0){40}.
  \end{center}
  \vspace{1in}
\end{thm}
\vspace{.5in}
\begin{thm}[3 Points]\label{ex6}
  Let $f(x)$ be a function of the variable $x$, and assume that $a \leq b$ are real numbers.
  State the formula for the net change between the inputs $x = a$ and $x = b$.
\end{thm}

\newpage
\section{Problems}
\begin{thm}[16 Points]\label{ex7}
  Consider the equation
  $$y^2 + 2x = 6.$$
  \begin{enumerate}[(a)]
    \item
      Does this equation define $y$ as function of $x$?  
      \emph{Briefly} justify why or why not.
      If it does, then give the value of $y$ when $x = 2$.
      \vspace{2in}
    \item
      Does this equation define $x$ as function of $y$?  \emph{Briefly} justify why or why not.
      If it does, then give the value of $x$ when $y = 2$.
      \vspace{2in}
  \end{enumerate}
\end{thm}

\begin{thm}[16 Points]\label{ex8}
  Let $f(x) = x^2 + x + 1$.  Compute the net change between $x = 1$ and $x = 4$.
\end{thm}

\newpage

\begin{thm}[16 Points]\label{ex9}
  Add the following rational expressions and simplify the result,
  $$\frac{6}{9 - x^2} + \frac{-1}{3 + x}.$$
\end{thm}

\newpage

\begin{thm}[16 Points]\label{ex10}
  Solve the equation
  $$2x^2 - 2x - 24 = 0$$
  for $x$.
  \vspace{3in}
\end{thm}

\begin{thm}[16 Points]\label{ex11}
  Find the domain of the function
  $$f(x) = \sqrt{x^2 - 2x - 3}.$$
  Express the solution using interval notation and graph the domain on the number line.
\end{thm}

\newpage

\begin{thm}[Bonus - 10 Points]\label{bonus}
  Consider the function
  $$p(x) = x^2 + 5x + 6.$$
  \begin{enumerate}[(a)]
  \item
    Find two numbers, $a < b$, for which the net change of $p$ from $a$ to $b$ is 6.
    \vspace{4in}
  \item
    Find two numbers, $c < d$, for which the net change of $p$ from $c$ to $d$ is -6.
    \newline
    [Hint: Your method from part (a) should also give you these values.]
  \end{enumerate}
\end{thm}
\end{document}
