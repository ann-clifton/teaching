\documentclass[12pt]{amsart}
\usepackage{amsmath,amsthm,amssymb,amsfonts,enumerate,tikz-cd,fancyhdr}
\openup 5pt
\author[Blake Farman]{Blake Farman\\University of South Carolina}
\title[Exam 02]{Math 111\\ Exam 02}
\date{November 2, 2017}
\pdfpagewidth 8.5in
\pdfpageheight 11in
\usepackage[margin=1in]{geometry}

\renewcommand{\qedsymbol}{}

\begin{document}
\maketitle

\begin{center}
  \fbox{\fbox{\parbox{5.5in}{\centering
        Answer the questions in the spaces provided on the
        question sheets and turn them in at the end of the class period.
        If you require extra space, use the back of the page and indicate that you have done so.
        
        Unless otherwise stated, all supporting work is required.
        Unsupported or otherwise mysterious answers will \textbf{not receive credit.}
        
        You may use a calculator, but you may \textbf{not} use a Computer Algebra System (CAS)
        or any other electronic device whatsoever, \textbf{including cell phones.}
        By writing your name on the line below, you indicate that you have read and understand these directions.}}}
\end{center}

\vspace{0.2in}
\makebox[\textwidth]{Name:\enspace\hrulefill}
\vspace{0.2in}

$$
\begin{array}{|c|c|c|}
  \hline
  \text{Problem} & \text{Points Earned} & \text{Points Possible}\\
  \hline
  1 & & 4\\
  \hline
  2 & & 6\\
  \hline
  3 & & 5\\
  \hline
  4 & & 3\\
  \hline
  5 & & 2 \\
  \hline
  6 & & 16\\
  \hline
  7 & & 16\\
  \hline
  8 & & 16\\
  \hline
  9 & & 16\\
  \hline
  10 & & 16\\
  \hline
%  \text{Bonus} & & 5\\
%  \hline
  \text{Total} & & 100\\
  \hline
\end{array}
$$

\newpage

%\theoremstyle{plain}
\theoremstyle{definition}
\newtheorem{thm}{}
\newtheorem{lem}{Lemma}
\newtheorem{defn}{Definition}

\section{Definitions}

\begin{thm}[4 Points]\label{ex1}
  \begin{enumerate}[(a)]
  \item
    State the Point-Slope form of a line passing through the point $(x_0, y_0)$ with slope $m$.
    \vspace{2in}
  \item
    State the Slope-Intercept form of a line with slope $m$ and $y$-intercept $b$.
    \vspace{2in}
  \end{enumerate}
\end{thm}

\begin{thm}[6 Points]\label{ex2}
  Let $f(x)$ be a function.
  State the average rate of change of $f$ between $x = a$ and $x = b$.
\end{thm}

\newpage

\begin{thm}[5 Points]\label{ex3}
  Let $f(x)$ be an exponential function and let $a$ be the growth/decay factor.
  Express the growth/decay rate, $r$, in terms of $a$.
  \vspace{1in}
\end{thm}

\begin{thm}[3 Points]\label{ex4}
  \begin{enumerate}[(a)]
  \item
    State the general form of an exponential function.
    \vspace{1in}
  \item
    When does such a function model exponential growth?
    \vspace{1in}
  \item
    When does such a function model exponential decay?
    \vspace{1in}
  \end{enumerate}
\end{thm}

\begin{thm}[2 Points]
  Consider the two lines $f(x) = m_1x + b_2$ an $g(x) = m_2x + b_2$.
  \begin{enumerate}[(a)]
  \item
    When are $f$ and $g$ parallel?
    \vspace{1in}
  \item
    When are $f$ and $g$ perpendicular?
    \vspace{1in}
  \end{enumerate}
\end{thm}

\newpage
\section{Problems}

\begin{thm}[16 Points]\label{ex5}
  In the following problems, use the given information to find the equation of the line in slope-intercept form.
  \begin{enumerate}[(a)]
  \item
    The line passing through the points $(-2,3)$ and $(5,-18)$.
    \vspace{1in}
  \item
    The line passing through the point $(3, -2)$ and parallel to the line $2y - 6x = 8$.
    \vspace{1in}
  \item
    The line passing through the origin (that is, the point $(0,0)$) and perpendicular to the line $4y - x = 8$.
    \vspace{1in}
  \end{enumerate}
  \vspace{1in}
\end{thm}


\begin{thm}[16 Points]\label{ex10}
  Consider the two lines $f(x) = x + 2$ and $g(x) = 3x + 4$.
  Find the point (that is, the $(x,y)$ pair) where these two lines intersect.
  \vspace{2in}
\end{thm}

\newpage

\begin{thm}[16 Points]\label{ex9}
  Let $f(x) = x^2 - 2$.
  \begin{enumerate}[(a)]
  \item
    Compute the average rate of change for $f$ between $x = 2$ and $x = 5$.
    \vspace{2in}
  \item
    Give the Point-Slope form of the line that passes through $(2, f(2))$ and $(5, f(5))$.
    \vspace{2in}
  \item
    Give the Slope-Intercept form of the line that passes through $(2, f(2))$ and $(5, f(5))$.
  \end{enumerate}
\end{thm}

\newpage

\begin{thm}[16 Points]\label{ex7}
  Alice is hosting an event.  She is renting a facility, which costs $\$150$, and providing refreshments, which cost $\$7$ per guest.
  \begin{enumerate}[(a)]
  \item
    Find a function, $C$, that models the total cost of the event if $x$ people attend.
    \vspace{1in}
  \item
    Sketch a graph of $C$.
    \vspace{2in}
  \item
    Evaluate $C(10)$ and $C(15)$.  What do these numbers represent?
    \vspace{1in}
  \item
    If the total cost for the event was $\$500$, how many people attended?
    \vspace{1in}
  \end{enumerate}
\end{thm}

\newpage

\begin{thm}[16 Points]
  A population of size $32$ grows by $25\%$ every day.
  \begin{enumerate}[(a)]
  \item
    Give the daily growth factor for this population.
    \vspace{1in}
  \item
    Give an exponential model for the size of the population after $t$ days.
    \vspace{1.5in}
  \item
    Determine the size of the population after 2 days.
    
    \noindent[Hint: Express the growth factor as a fraction, rather than a decimal, and this will be very easy to compute.]
    \vspace{1.5in}
  \end{enumerate}
\end{thm}

%\newpage

%\begin{thm}[Bonus - 5 Points]\label{bonus}
%  Let $f(x) = x^2 - 1$.
%  Show that the average rate of change of $f$ between $x = a$ and $x = b$ is always $a + b$.
%\end{thm}

\end{document}
