\documentclass[10pt]{amsart}
\usepackage{amsmath,amsthm,amssymb,amsfonts,enumerate,hyperref}
\openup 5pt
\author{Blake Farman\\University of South Carolina}
\title{Syllabus\\Math 111-006\\Fall 2017}
\date{August 24, 2017}
\pdfpagewidth 8.5in
\pdfpageheight 11in
\usepackage[margin=1in]{geometry}

\begin{document}
\maketitle

\section*{Contact Information}
\noindent
\subsection*{E-mail:} \href{mailto:farmanb@math.sc.edu}{farmanb@math.sc.edu}
\subsection*{Office:} LeConte 317N
\subsection*{Office Hours:}
I will be available in my office every Tuesday/Thursday, 4:00 pm - 5:30 pm.
Additionally, I am available by appointment if these times are not amenable.
You should utilize this time to ask questions about homework, clarify concepts, etc. as needed.

\section*{Course Information}
\noindent
\subsection*{Lectures:}
Tuesday/Thursday,  6:00 pm - 7:15 pm in Callcot, Room 003.

\subsection*{Learning Outcomes:} Upon successful completion of this course, students should be able to:
\begin{itemize}
\item
  Recall basic mathematical terms related to linear, quadratic, exponential, and logarithmic functions and express these terms in correct context;
\item
  Apply the methods of algebra to solve applications involving intercepts, rates of change, inequalities, systems of equations, and interest growth;
\item
  Verbally interpret relationships in data given as graphs, tables, and equations and express functions given in verbal context as a graph, table, or equation.
\end{itemize}
\subsection*{Pre-Requisites:} Qualification through placement.
\subsection*{Text:}
The required text for this course is\\
\begin{center}
  {\it College Algebra: Concepts and Contexts}, $1^{\text{st}}$ Edition, Stewart, Redlin, Watson, Panman, 2011.  \\ISBN 9781424089208.
\end{center}

\noindent
The university bookstore offers two options for obtaining this text:
\begin{itemize}
\item
  A bundle that includes a hard copy of the text and a WebAssign access key,
\item
  A WebAssign access key that grants access to a digital copy of the text.
\end{itemize}
You may choose either option, however, be aware that it is {\bf expected} that you will read the text outside of lecture.

\subsection*{Course Website:} The URL for the course website is
\begin{center}
  \url{http://people.math.sc.edu/farmanb/courses/111/F17/index.html}
\end{center}
Here you can find a digital copy of the syllabus, lecture notes, test dates, and other various announcements.

\newpage

\section*{Coursework}
\noindent
\subsection*{Homework:}
Homework assignments are available on WebAssign.
All WebAssign homework assignments are currently available and may be accessed using the class key
\begin{center}
  {\bf sc 8392 8850}
\end{center}
It is important that you create a WebAssign account and register for the course as soon as possible.

The assignments are separated into three groups corresponding to the material on each of the exams.
The problems are chosen to highlight the core concepts on each exam and mastery of these homework sets serve as a good indicator for exam performance.
As such, you should ensure that you fully understand the material on these homework sets; that is, upon completion of the homework set, you should be capable of completing similar problems without the aid of the text or the various tools provided by WebAssign.

The homework sets in each group are due the day before the corresponding exam.
Late work will {\bf not} be accepted, and you are solely responsible for ensuring that these assignments are completed on time.
These assignments are relatively long, so you should be working through them as we cover the material.
Do {\bf not} leave these assignments until the last minute.

\subsection*{Exams:}
There will be three in-class exams.
The exams are tentatively scheduled as follows:
\begin{center}
  \begin{tabular}{ll}
    Exam 1: & Thursday, September 28, 2017,\\
    Exam 2: & Thursday, November 2, 2017,\\
    Exam 3: & Thursday, November 30, 2017.\\
  \end{tabular}\\
\end{center}
Though unexpected, any deviations from this schedule will be announced during lecture and reflected on the course website.
\subsubsection*{Missed Assessments:}
There will {\bf not} be any make-up exams or quizzes.
If you miss one exam, your final exam grade will replace the missing exam grade.
\textbf{This policy is intended only for exams missed due to illness, accidents, etc.  
  It does NOT mean that your lowest exam grade will be dropped.}
Any further missed exams will receive a grade of zero.
\subsection*{Final Exam:} There will be a cumulative final exam on Tuesday, December 12, 2017 at 7:30 pm.
The date and time of the final exam is determined by the registrar and \textbf{may not be changed} except under extreme circumstances.
\textbf{Scheduling your travel home before the day of the final exam does not constitute a valid excuse for rescheduling the final exam.}
Failure to appear at the final exam will result in a zero on the final exam.

\newpage

\section*{Grading}
\subsection*{Scale:}
Grades will be assigned on the following scale:
\begin{center}
  \begin{tabular}{ll}
    A: &90-100\%,\\
    B: & 80-89\%,\\
    C: & 70-79\%,\\
    D: & 60-69\%,\\
    F: & $<$ 60\%.\\
  \end{tabular}\\
\end{center}
\subsection*{Weights:}
Final grades will be calculated with the following weights:
\begin{center}
  \begin{tabular}{lr}
    Homework: & 30\%,\\
    Quizzes: &10\%,\\
    Exams: & 30\%,\\
    Final Exam: & 30\%.\\
  \end{tabular}\\
\end{center}

\section*{Math Tutoring Center}
\noindent
The mathematics department at the University of South Carolina offers free tutoring to UofSC students in LeConte 105.
It is staffed with talented graduate students (including myself) who will answer questions about MATH 111, 115, 122, 141, 142, and 170.
No appointment is necessary--just stop by during the hours listed on the schedule at
\begin{center}
  \url{http://math.sc.edu/math-tutoring-center}.
\end{center}

\section*{Student Success Center}
\noindent
In partnership with UofSC faculty, the Student Success Center (SSC) offers a number of programs to assist you in better understanding your course material and to aid you on your path to success. 
SSC programs are facilitated by professional staff, graduate students, and trained undergraduate peer leaders who have previously excelled in their courses.
Resources available to students include: 
\begin{itemize}
\item 
  \textbf{Peer Tutoring:} make a one-on-one appointment with a Peer Tutor by going to \url{www.sc.edu/success}. 
  If a course is not on the semester’s supported-course list, there is a process for requesting assistance. 
  The full schedule of days/times/locations for drop-in and Online Tutoring hours can also be viewed on the SSC website.
\item
  \textbf{Supplemental Instruction (SI):} attend SI Sessions to focus on the most difficult content being covered in class. 
  SI Leaders are assigned to specific sections of courses and hold three weekly study sessions that can serve as “built-in study time.” 
  The schedule is posted on the SSC website each week and will also be communicated by the SI Leader.
\item
  \textbf{Peer Success Consultations:} make a one-on-one Success Consultation with a Peer Consultant to work on developing study skills, setting goals, and connecting to a variety of campus resources. 
  Your instructor may communicate with the SSC via Success Connect, an online referral system, regarding your progress in the course. 
  If contacted by the SSC, please schedule a Success Consultation. 
  Success Connect referrals are not punitive and any information shared by your professor is confidential and subject to FERPA regulations.
\end{itemize}
SSC services are offered to all UofSC undergraduates at no additional cost.
You are invited to call the Student Success Hotline at (803) 777-1000, visit \url{www.sc.edu/success}, or come to the SSC in the Thomas Cooper Library (Mezzanine Level) to check schedules and make appointments. 

\newpage

\section*{Expectations}
\noindent
\subsection*{Academic Integrity:} 
\textit{As a Carolinian...I will practice personal and academic integrity.}\\

\noindent 
The University of South Carolina expects high standards in all areas from its students.
The University, as well as the faculty, staff, alumni, and students believe strongly in the Honor Code.
This Code requires acceptance of certain responsibilities and agreement by all students to abide by the spirit of the Honor Code upon entering the University of South Carolina.
In order that you may better understand the required responsibilities, the general University community codes are outlined below.

\begin{enumerate}
\item
  It shall be the responsibility of every faculty member, student, administrator and staff member of the University community to uphold and maintain the academic standards and integrity of the University of South Carolina.
\item
  Any member of the University community, who has reasonable grounds to believe that an infraction of the Honor Code has occured, has an obligation to report the alleged violation.
  Violation of any of the following standards subjects the student to disciplinary action: improper collaboration, cheating, lying, bribery, and plagiarism.
\end{enumerate}

\noindent
Your enrollment in this class signifies your willingness to accept these responsibilities and uphold the Honor Code of the University of South Carolina.
Please review the Honor Code via
\begin{center}
  \url{http://www.housing.sc.edu/academicintegrity}.
\end{center}

\noindent
Any deviation from this expectation \textbf{will result in a grade of F in the course} and disciplinary action through the Office of Academic Integrity.
\subsection*{Attendance:} 
Students are obligated to complete all assigned work promptly, to attend class regularly, and to participate in whatever class discussion may occur.\\

\noindent
The following events or circumstances are potentially excusable absences:
\begin{itemize}
\item
  participation in an authorized University activity (such as musical performances, academic competitions, or varsity athletic events in which the student plays a formal role in a University sanctioned event),
\item
  required participation in military duties,
\item
  mandatory admission interviews for professional or graduate school which cannot be rescheduled,
\item
  participation in legal proceedings or administrative duties that require a student's presence,
\item
  death or major illness in a student’s immediate family,
\item
  illness of a dependent family member
\item
  religious holy day if listed on \url{www.interfaithcalendar.org},
\item
  illness that is too severe or contagious for the student to attend class,
\item
  weather-related emergencies.
\end{itemize}
For more information, see the University Attendance Policy:
\begin{center}
  \url{http://bulletin.sc.edu/content.php?catoid=36\&navoid=3738}.
\end{center}

\end{document}
