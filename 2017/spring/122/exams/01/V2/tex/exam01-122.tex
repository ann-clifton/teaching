\documentclass[12pt]{amsart}
\usepackage{amsmath,amsthm,amssymb,amsfonts,enumerate,mymath,tikz-cd,fancyhdr,multicol}
\openup 5pt
\author{Blake Farman\\University of South Carolina}
\title{Math 122\\Exam 01}
\date{February 7, 2017}
\pdfpagewidth 8.5in
\pdfpageheight 11in
\usepackage[margin=1in]{geometry}

\renewcommand{\qedsymbol}{}
\theoremstyle{definition}
\newtheorem{thm}{}
\newtheorem{lem}{Lemma}
\theoremstyle{definition}
\newtheorem{defn}{Definition}
\begin{document}
\maketitle

\begin{center}
  \fbox{\fbox{\parbox{5.5in}{\centering
        Answer the questions in the spaces provided on the
        question sheets and turn them in at the end of the class period.
        If you require extra space, use the back of the page and indicate that you have done so.
        
        Unless otherwise stated, all supporting work is required.
        Unsupported or otherwise mysterious answers will {\bf not receive credit.}
        You may use a calculator {\bf without a CAS} if you like, but a calculator is not necessary.
        By writing your name on the line below, you acknowledge that you have read and understand these directions.}}}
\end{center}

\vspace{0.2in}
\makebox[\textwidth]{Name:\enspace\hrulefill}
\vspace{0.2in}

$$
\begin{array}{|c|c|c||c|c|c|}
  \hline
  \text{Definitions} & \text{Points Earned} & \text{Points Possible} & \text{Problems} & \text{Points Earned} & \text{Points Possible}\\
  \hline
  1 & & 2 & 1 & & 6\\
  \hline
  2 & & 5 & 2 & & 9\\
  \hline
  3 & & 5 & 3 & & 12\\
  \hline
  4 & & 5 & 4 & & 15\\
  \hline
  5 & & 6 & 5 & & 15\\
  \hline
  \text{Subtotal} & & 23 & 6 & & 20\\
  \hline
  & & & \text{Subtotal} & & 77\\
  \hline
  & & & \text{Total} & & 100\\
  \hline
\end{array}
$$

\newpage

\section{Definitions}
\begin{thm}[2 Points]\label{ex1}
  \begin{enumerate}[(a)]
  \item
    State the Point-Slope form of a line passing through the point $(x_0, y_0)$ with slope $m$.
    \vspace{2in}
  \item
    State the Slope-Intercept form of a line with slope $m$ and $y$-intercept $b$.
    \vspace{2in}
  \end{enumerate}
\end{thm}

\begin{thm}[5 Points]
  Let $f$ be a function and let $a < b$ be given.
  State the average rate of change of $f$ on the interval $[a,b]$.
  \vspace{2in}
\end{thm}

\newpage

\begin{thm}[5 Points]
  Given a quantity $P$, state the relative change of the quantity from $P$ to $P^\prime$.
  \vspace{2in}
\end{thm}

\begin{thm}[5 Points]
  \begin{enumerate}[(a)]
  \item
    State the form of an exponential function of a variable $t$ with inital value $P_0$ and base $a$:
    \vspace{0.15in}
    $$P(t) =\ \line(1,0){100}.$$
  \item
    The relative rate of change of $P$ is 
    \vspace{0.3in}
    $$r =\ \line(1,0){100}.$$
    [Hint: If you don't recall the formula, this is just the relative change from $P(t)$ to $P(t + 1)$.]\\
  \item
    The function $P$ models 
    \begin{itemize}
    \item
      \vspace{0.15in}
      exponential growth when $r$ is \line(1,0){100},
      \vspace{0.15in}
    \item
      exponential decay when $r$ is \line(1,0){100}.
    \end{itemize}
  \item
    The continuous growth/decay rate is
    \vspace{.15in}
    $$k =\ \line(1,0){100}.$$
  \end{enumerate}
\end{thm}

\begin{thm}[6 Points]\label{ex2}
  Let $0 < x, 0 < y$ be given.
  Fill in the blanks:
  \vspace{.15in}
  \begin{multicols}{2}
    \begin{enumerate}[(i)]
    \item
      $\displaystyle{\ln(1) =\ \line(1,0){100}}$,
      \vspace{.15in}
    \item
      $\displaystyle{\ln(xy) =\ \line(1,0){100}}$,
      \vspace{.15in}
      \item
        $\displaystyle{\ln\left(\frac{x}{y}\right) =\ \line(1,0){100}}$,
        \vspace{.15in}
      \item
        $\displaystyle{\ln(x^r) =\ \line(1,0){100}}$,
        \vspace{.15in}
      \item
        $\displaystyle{\ln\left(e^x\right) =\ \line(1,0){100}}$,
        \vspace{.15in}
      \item
        $\displaystyle{e^{\ln(x)} =\ \line(1,0){100}}$.
    \end{enumerate}
  \end{multicols}
\end{thm}

\newpage

\section{Problems}
\setcounter{thm}{0}
\begin{thm}[6 Points]\label{ex5}
  \begin{enumerate}[(a)]
  \item
    Find the slope of the line passing through the points $\left(3,\frac{1}{2}\right)$ and $(2,1)$.
    \vspace{2in}
  \item
    Write the equation of this line in Point-Slope Form.
    \vspace{2in}
  \item
    Write the equation of this line in Slope-Intercept Form.
    \vspace{2in}
  \item
    Sketch a graph of $f(x)$.
    Label the $x$-intercept and the $y$-intercept.
  \end{enumerate}
\end{thm}

\newpage
\begin{thm}[9 Points]\label{ex9}
    Let $f(x) = -x^2 + 1$.
  \begin{enumerate}[(a)]
  \item
    Compute the average rate of change for $f$ between $x = 3$ and $x = 5$.
    \vspace{2in}
  \item
    Give the Point-Slope form of the line that passes through $(3, f(3))$ and $(5, f(5))$.
    \vspace{2in}
  \item
    Give the Slope-Intercept form of the line that passes through $(3, f(3))$ and $(5, f(5))$.
  \end{enumerate}
\end{thm}

\newpage

\begin{thm}[12 Points]\label{ex9}
  A biologist observes a population with initial size $9$.
  In two years, the biologist returns to observe the population again and finds that there are $81$.
  \begin{enumerate}[(a)]
  \item
    Find an exponential function for the size of the population as a function of $t$ years since the initial observation.
    \vspace{2in}
  \item
    Does the function from part (a) model growth or decay?
    \vspace{2in}
  \item
    Use the model from part (a) to determine how many years it will take for the size of the population to reach 243.
  \end{enumerate}
\end{thm}

\newpage

\begin{thm}[15 Points]
  A bank is offering an account that pays $7\%$ interest compounded continuously.
  If you decide to invest money in this account, how long will it take for your initial investment to double?
  Give an {\bf exact} answer.
\end{thm}
\newpage

\begin{thm}[15 Points]
  Sketch a graph of the function
  $$f(x) = 2x^2 - 8x + 6.$$
  Label any $x$-intercepts, the $y$-intercept, and the vertex.
\end{thm}
\newpage

\begin{thm}[20 Points]\label{ex9}
  A company hosts a weekly event.
  They find that $25$ people attend at a ticket price of $\$30$, and $15$ people attend at a ticket price of $\$50$.
  Assuming this relationship is linear, determine the ticket price that will generate the highest revenue.
\end{thm}
\end{document}
