\documentclass[12pt]{amsart}
\usepackage{amsmath,amsthm,amssymb,amsfonts,enumerate,mymath,tikz-cd,fancyhdr,multicol}
\openup 5pt
\author{Blake Farman\\University of South Carolina}
\title{Math 122\\Final Exam}
\date{April 27, 2017}
\pdfpagewidth 8.5in
\pdfpageheight 11in
\usepackage[margin=1in]{geometry}

\renewcommand{\qedsymbol}{}
\theoremstyle{definition}
\newtheorem{thm}{}
\newtheorem{lem}{Lemma}
\theoremstyle{definition}
\newtheorem{defn}{Definition}

\newcommand{\ddx}[1]{\frac{\operatorname{d}}{\operatorname{d}\!#1}}
\newcommand{\dx}[1]{\!\!\;\operatorname{d}\!#1}

\begin{document}
\maketitle

\begin{center}
  \fbox{\fbox{\parbox{5.5in}{\centering
        Answer the questions in the spaces provided on the
        question sheets and turn them in at the end of the class period.
        If you require extra space, use the back of the page and indicate that you have done so.
        
        Unless otherwise stated, all supporting work is required.
        Unsupported or otherwise mysterious answers will {\bf not receive credit.}
        You may use a calculator {\bf without a CAS} if you like, but a calculator is not necessary.
        By writing your name on the line below, you acknowledge that you have read and understand these directions.}}}
\end{center}

\vspace{0.2in}
\makebox[\textwidth]{Name:\enspace\hrulefill}
\vspace{0.2in}

$$
\begin{array}{|c|c|c||c|c|c|}
  \hline
  \text{Definitions} & \text{Points Earned} & \text{Points Possible} & \text{Problems} & \text{Points Earned} & \text{Points Possible}\\
  \hline
  1 & & 3 & 1 & & 16\\
  \hline
  2 & & 3 & 2 & & 16\\
  \hline
  3 & & 3 & 3 & & 16\\
  \hline
  4 & & 6 & 4 & & 16\\
  \hline
  5 & & 2 & 5 & & 16\\
  \hline
  6 & & 3 & \text{Subtotal} & & 80\\
  \hline
  \text{Subtotal} & & 20 & \text{Bonus} & & 5\\
  \hline
  & & & \text{Total} & & 100\\
  \hline
\end{array}
$$

\newpage

\section{Definitions}

Throughout this section, let $f$ and $g$ be differentiable functions.
Fill in the blanks.
\begin{thm}[3 Points]
  \begin{enumerate}[(a)]
    Let $a$ be a constant.
  \item
    \vspace{.05in}
    $$\ddx{x}\left(a f(x)\right)\ =\ \line(1,0){250}$$
    \vspace{.05in}
  \item
    \vspace{.05in}
    $$\ddx{x}\left(f(x) + g(x)\right)\ =\ \line(1,0){250}$$ 
    \vspace{.05in}
  \item
    \vspace{.05in}
    $$\ddx{x}\left(f(x) - g(x)\right)\ =\ \line(1,0){250}$$ 
  \end{enumerate}
\end{thm}

\vspace{.5in}

\begin{thm}[3 Points]
  \begin{enumerate}[(a)]
  \item
    For $n$ a number,
    \vspace{.05in}
    $$\ddx{x}\left(x^n\right)\ =\ \line(1,0){250}$$
    \vspace{.05in}
  \item
    \vspace{.05in}
    $$\ddx{x}\ln(x)\ =\ \line(1,0){250}$$
    \vspace{.05in}
  \item
    \vspace{.05in}
    $$\ddx{x}e^x\ =\ \line(1,0){250}$$
    \vspace{.05in}
  \end{enumerate}
\end{thm}

\newpage

\begin{thm}[3 Points]
  Write the formula for each of the following derivatives.
  \vspace{.05in}
  \begin{enumerate}[(a)]
  \item
    $$\ddx{x}\left(f(x)g(x)\right)$$
    \vspace{2in}
  \item
    $$\ddx{x}\left(\frac{f(x)}{g(x)}\right)$$
    \vspace{2in}
  \item
    $$\ddx{x}\left(f \circ g(x)\right)$$    
  \end{enumerate}
\end{thm}

\newpage

\begin{thm}[6 Points]
  If $F^\prime(t)$ is a continuous function on the interval $[a,b]$, then 
  $$\int_a^b F^\prime(t)\dx{t}\ =\ \line(1,0){250}$$
\end{thm}

\vspace{.15in}

\begin{thm}[4 Points]
  Assume that $\int f(x)\dx{x}$ and $\int g(x)\dx{x}$ exist.
  \begin{enumerate}[(a)]
  \item
    $$\int f(x) \pm g(x) \dx{x}\ =\ \line(1,0){250}$$
  \item
    Let $a$ be a number.
    $$\int af(x)\dx{x}\ =\ \line(1,0){250}$$
  \end{enumerate}
\end{thm}

\vspace{.15in}

\begin{thm}[5 Points]
  Let $n \neq -1$ be a fixed number.
  \begin{enumerate}[(a)]
  \item
    $$\int x^n \dx{x}\ =\ \line(1,0){250}$$
    \vspace{.15in}
  \item
    $$\int e^x \dx{x}\ =\ \line(1,0){250}$$
    \vspace{.15in}
  \item
    $$\int \frac{1}{x}\dx{x}\ =\ \line(1,0){250}$$
  \end{enumerate}
\end{thm}

\newpage

\section{Problems}
\setcounter{thm}{0}
\begin{thm}[16 Points]
  Compute the derivative of the following functions.
  \begin{enumerate}[(a)]
  \item
    $\displaystyle{f(x) = \frac{1}{3}x^3 + 2\sqrt{x} + e^x + 5\ln(x)}$
    \vspace{2in}
  \item
    $\displaystyle{f(x) = \ln((x+1)^2) + e^{x^2} + \sqrt{x^2 + 35}}$
    \vspace{2in}
  \item
    $\displaystyle{f(x) = x^2\ln(x) + (x+5)e^{5x}}$
    \vspace{2in}
  \item
    $\displaystyle{f(x) = \frac{x^2- 1}{x^2 + 1}}$
  \end{enumerate}
\end{thm}

\newpage
\begin{thm}[16 Points]
  Let $f(x) = \frac{2}{3}x^3 - x^2 - 12x$.
    \begin{enumerate}[(a)]
  \item
    Find the derivative of $f$.
    \vspace{1in}
  \item
    Find the critical points of $f$.\newline
    [Hint: Factoring after taking the derivative will make this task much easier.]
    \vspace{2in}
  \item
    Find any local maxima and local minima of $f$.
    Clearly indicate whether a point is a maximum or a minimum.
    \vspace{2in}
  \item
    Find any inflection points of $f$.\newline
    [Hint: Same as for part (b).]
  \end{enumerate}
\end{thm}

\newpage
\begin{thm}[16 Points]
  A company sells a product for \$30 each and the manufacturing costs can be modeled by the function
  $$C(q) = q^3 - 9q^2 + 45q + 15$$
  of $q$ units produced.
  \begin{enumerate}[(a)]
  \item
    What is the revenue as a function of $q$ units?
    \vspace{1in}
  \item
    What is the profit as a function of $q$ units?
    \vspace{1in}
  \item
    What is the marginal profit function (i.e. the derivative)?
    \vspace{1in}
  \item
      For each of the quantities below, determine whether the company should increase, decrease, or not change the production levels in order to maximize profit.
  Justify your answers using calculus.
  {\bf You will not receive credit for guess and check solutions.}
  \begin{multicols}{3}
  \begin{enumerate}[(a)]
    \item
      $q = 2$
    \item
      $q = 6$
    \item
      $q = 7$
    \end{enumerate}
  \end{multicols}
  \end{enumerate}
\end{thm}

\newpage
\begin{thm}[16 Points]
  Compute the following indefinite integrals.
  \begin{enumerate}[(a)]
  \item
    $\displaystyle{\int{\left(\frac{2}{x} + 4x^3 + 4e^x\right)}\dx{x}}$
    \vspace{2in}
  \item
    $\displaystyle{\int(2x + 3)e^{x^2 + 3x + 7}\dx{x}}$
    \vspace{2in}
  \item
    $\displaystyle{\int \frac{x^2 + 2}{x^3 + 6x}\dx{x}}$
    \vspace{2in}
  \item
    $\displaystyle{\int \frac{x}{\sqrt{x^2 + 3}}\dx{x}}$
  \end{enumerate}
\end{thm}
\newpage

\begin{thm}[16 Points]
  Use the Fundamental Theorem of Calculus to compute the following definite integrals.
  \begin{enumerate}[(a)]
  \item
    $\displaystyle{\int_3^5 (x + 2)\dx{x}}$
    \vspace{2in}
  \item
    $\displaystyle{\int_1^3 \left(3x^2 + 2x + 7\right) \dx{x}}$
    \vspace{2in}
  \item
    $\displaystyle{\int_0^{\ln(3)}e^x\dx{x}}$
    \vspace{2in}
  \item
    $\displaystyle{\int_1^e \frac{2\ln(x)}{x}\dx{x}}$
  \end{enumerate}
\end{thm}

\newpage

\begin{thm}[Bonus]
  Write down a function that has a local minimum at $x = -1$, a local maximum at $x = 0$, and a local minimum at $x = 1$.
  Justify why the function you wrote down satisfies this criteria.\newline
  [Hint: You don't need to get too fancy here.  This can be done with a polynomial with integer coefficients.  The easiest way to do this is to start by defining what the derivative {\it should} be, then use an integral to get the function you want.]
\end{thm}
\end{document}
