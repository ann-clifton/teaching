\documentclass[12pt]{amsart}
\usepackage{amsmath,amsthm,amssymb,amsfonts,enumerate,mymath,tikz-cd,fancyhdr,multicol}
\openup 5pt
\author{Blake Farman\\University of South Carolina}
\title{Math 122\\Exam 02}
\date{March 16, 2017}
\pdfpagewidth 8.5in
\pdfpageheight 11in
\usepackage[margin=1in]{geometry}

\renewcommand{\qedsymbol}{}
\theoremstyle{definition}
\newtheorem{thm}{}
\newtheorem{lem}{Lemma}
\theoremstyle{definition}
\newtheorem{defn}{Definition}

\newcommand{\ddx}[1]{\frac{\operatorname{d}}{\operatorname{d}\!#1}}
\newcommand{\dx}[1]{\!\!\;\operatorname{d}\!#1}

\begin{document}
\maketitle

\begin{center}
  \fbox{\fbox{\parbox{5.5in}{\centering
        Answer the questions in the spaces provided on the
        question sheets and turn them in at the end of the class period.
        If you require extra space, use the back of the page and indicate that you have done so.
        
        Unless otherwise stated, all supporting work is required.
        Unsupported or otherwise mysterious answers will {\bf not receive credit.}
        You may use a calculator {\bf without a CAS} if you like, but a calculator is not necessary.
        By writing your name on the line below, you acknowledge that you have read and understand these directions.}}}
\end{center}

\vspace{0.2in}
\makebox[\textwidth]{Name:\enspace\hrulefill}
\vspace{0.2in}

$$
\begin{array}{|c|c|c||c|c|c|}
  \hline
  \text{Derivative Rules} & \text{Points Earned} & \text{Points Possible} & \text{Problems} & \text{Points Earned} & \text{Points Possible}\\
  \hline
  1 & & 3 & 1 & & 5\\
  \hline
  2 & & 3 & 2 & & 10\\
  \hline
  3 & & 3 & 3 & & 12\\
  \hline
  4 & & 5 & 4 & & 12\\
  \hline
  5 & & 2 & 5 & & 10\\
  \hline
  6 & & 2 & 6 & & 12\\
  \hline
  7 & & 1 & 7 & & 20\\
  \hline
  \text{Subtotal} & & 19 & \text{Subtotal} & & 81\\
  \hline
  & & & \text{Total} & & 100\\
  \hline
\end{array}
$$

\newpage

\section{Derivative Rules}
Throughout this section, let $f$ and $g$ be differentiable functions.
Fill in the blanks.
\begin{thm}[3 Points]
  \begin{enumerate}[(a)]
    Let $a$ be a constant.
  \item
    \vspace{.05in}
    $$\ddx{x}\left(a f(x)\right)\ =\ \line(1,0){250}$$
    \vspace{.05in}
  \item
    \vspace{.05in}
    $$\ddx{x}\left(f(x) + g(x)\right)\ =\ \line(1,0){250}$$ 
    \vspace{.05in}
  \item
    \vspace{.05in}
    $$\ddx{x}\left(f(x) - g(x)\right)\ =\ \line(1,0){250}$$ 
  \end{enumerate}
\end{thm}

\vspace{.5in}

\begin{thm}[3 Points]
  \begin{enumerate}[(a)]
  \item
    For $n$ a number,
    \vspace{.05in}
    $$\ddx{x}\left(x^n\right)\ =\ \line(1,0){250}$$
    \vspace{.05in}
  \item
    \vspace{.05in}
    $$\ddx{x}\ln(x)\ =\ \line(1,0){250}$$
    \vspace{.05in}
  \item
    \vspace{.05in}
    $$\ddx{x}e^x\ =\ \line(1,0){250}$$
    \vspace{.05in}
  \end{enumerate}
\end{thm}

\newpage

\begin{thm}[3 Points]
  Write the formula for each of the following derivatives.
  \vspace{.05in}
  \begin{enumerate}[(a)]
  \item
    $$\ddx{x}\left(f(x)g(x)\right)$$
    \vspace{2in}
  \item
    $$\ddx{x}\left(\frac{f(x)}{g(x)}\right)$$
    \vspace{2in}
  \item
    $$\ddx{x}\left(f \circ g(x)\right)$$    
  \end{enumerate}
\end{thm}

\newpage
For each of the following questsions, circle the correct answer.
\begin{thm}[5 Points]
  Assume that $f$ is a function such that $f^\prime(x)$ and $f^{\prime\prime}(x)$ are defined for all $x$.
  \begin{enumerate}[(a)]
  \item
    A point $p$ is a critical point of $f$ if
    \begin{multicols}{2}
      \begin{enumerate}[(i)]
      \item
        $f^\prime(p) = 0$
        \vspace{.05in}
      \item
        $f^\prime(p) > 0$,
      \item
        $f^\prime(p) < 0$,
        \vspace{.05in}
      \item
        $f(p) = 0$.
      \end{enumerate}
    \end{multicols}
  \item
    $f$ is increasing on an interval if
    \begin{multicols}{2}
      \begin{enumerate}[(i)]
      \item
        $f^\prime < 0$ on that interval,
        \vspace{.05in}
      \item
        $f > 0$ on that interval,
      \item
        $f^\prime > 0$ on that interval,
        \vspace{.05in}
      \item
        $f < 0$ on that interval.
      \end{enumerate}
    \end{multicols}
    \item
    $f$ is decreasing on an interval if
    \begin{multicols}{2}
      \begin{enumerate}[(i)]
      \item
        $f^\prime < 0$ on that interval,
        \vspace{.05in}
      \item
        $f > 0$ on that interval,
      \item
        $f^\prime > 0$ on that interval,
        \vspace{.05in}
      \item
        $f < 0$ on that interval.
      \end{enumerate}
    \end{multicols}
  \item
    $f$ is concave down on an interval if
    \begin{multicols}{2}
      \begin{enumerate}[(i)]
      \item
        $f^{\prime\prime} = 0$ on that interval,
        \vspace{.05in}
      \item
        $f^{\prime\prime} < 0$ on that interval,
      \item
        $f^{\prime\prime} > 0$ on that interval,
        \vspace{.05in}
      \item
        $f^\prime = 0$ on that interval.
      \end{enumerate}
    \end{multicols}
  \item
    $f$ is concave up on an interval if
    \begin{multicols}{2}
      \begin{enumerate}[(i)]
      \item
        $f^{\prime\prime} = 0$ on that interval,
        \vspace{.05in}
      \item
        $f^{\prime\prime} < 0$ on that interval,
      \item
        \vspace{.05in}
        $f^{\prime\prime} > 0$ on that interval,
      \item
        $f^\prime = 0$ on that interval.
      \end{enumerate}
    \end{multicols}
  \end{enumerate}
\end{thm}

\newpage

\begin{thm}[2 Points]
  The first derivative test says that a critical point, $p$, of $f$ is a
  \begin{enumerate}[(a)]
  \item
    local maximum if
    \begin{multicols}{2}
      \begin{enumerate}[(i)]
      \item
        $f^\prime$ changes from negative to positive at $p$,
        \vspace{.05in}
      \item
        $f^\prime$ changes from positive to negative at $p$,
      \item
        $f$ changes from negative to positive at $p$,
        \vspace{.05in}
      \item
        $f$ changes from positive to negative at $p$.
      \end{enumerate}
    \end{multicols}
  \item
    local minimum if
    \begin{multicols}{2}
      \begin{enumerate}[(i)]
        \item
        $f^\prime$ changes from negative to positive at $p$,
        \vspace{.05in}
      \item
        $f^\prime$ changes from positive to negative at $p$,
      \item
        $f$ changes from negative to positive at $p$,
        \vspace{.05in}
      \item
        $f$ changes from positive to negative at $p$.
      \end{enumerate}
    \end{multicols}
  \end{enumerate}
\end{thm}

\begin{thm}[2 Points]
  The second derivative test says that a critical point, $p$, of $f$ is a
  \begin{enumerate}[(a)]
  \item
    local maximum if
    \begin{multicols}{2}
      \begin{enumerate}[(i)]
      \item
        $f^{\prime\prime}$ changes from negative to positive at $p$,
        \vspace{.05in}
      \item
        $f^{\prime\prime}$ changes from positive to negative at $p$,
      \item
        $f^{\prime\prime}(p) < 0$,
        \vspace{.05in}
      \item
        $f^{\prime\prime}(p) > 0$.
      \end{enumerate}
    \end{multicols}
  \item
    local minimum if
    \begin{multicols}{2}
      \begin{enumerate}[(i)]
        \item
        $f^{\prime\prime}$ changes from negative to positive at $p$,
        \vspace{.05in}
      \item
        $f^{\prime\prime}$ changes from positive to negative at $p$,
      \item
        $f^{\prime\prime}(p) < 0$,
        \vspace{.05in}
      \item
        $f^{\prime\prime}(p) > 0$.
      \end{enumerate}
    \end{multicols}
  \end{enumerate}
\end{thm}

\begin{thm}[1 Point]
  Suppose that $f^{\prime\prime}(p)\ = 0$.
  We say that $p$ is an inflection point of $f$ if
  \begin{multicols}{2}
    \begin{enumerate}[(i)]
    \item
      $f^{\prime}(p) = 0$,
      \vspace{.05in}
    \item
      $f(p) = 0$,
    \item
      $f^{\prime\prime}(p)$ changes sign at $p$,
      \vspace{.05in}
    \item
      $f$ changes sign at $p$.
    \end{enumerate}
  \end{multicols}
\end{thm}

\newpage

\section{Problems}
\setcounter{thm}{0}
\begin{thm}[5 Points]
  Find the derivative of the following functions.
  \begin{enumerate}[(a)]
  \item
    $\displaystyle{f(x) = 3x + 7}$
    \vspace{1in}
  \item
    $\displaystyle{g(x) = 5x^2 + 2x + 1}$
    \vspace{1in}
  \item
    $\displaystyle{h(x) = 12x^3 + 13x^2}$
    \vspace{1in}
  \item
    $\displaystyle{r(x) = \frac{1}{3}x^3 + 2}$
    \vspace{1in}
  \item
    $\displaystyle{s(x) = \sqrt{x} + 3}$
    \vspace{1in}
  \end{enumerate}
\end{thm}

\newpage
\begin{thm}[10 Points]
  Find the derivative of the following functions.
  \begin{enumerate}[(a)]
  \item
    $\displaystyle{(x + 7)^{25}}$
    \vspace{1in}
  \item
    $\displaystyle{e^{\frac{1}{2}x^2 + 2x + 1}}$
    \vspace{1in}
  \item
    $\displaystyle{\ln(2x^2 + 7)}$
    \vspace{1in}
  \item
    $\displaystyle{\sqrt{x^2 + 1}}$
    \vspace{1in}
  \item
    $\displaystyle{6e^{5x} + e^{-x^2}}$
  \end{enumerate}
\end{thm}

\newpage

\begin{thm}[12 Points]
  Differentiate the following functions
  \begin{enumerate}[(a)]
  \item
    $\displaystyle{xe^{-2x}}$
    \vspace{2in}
  \item
    $\displaystyle{x\ln(x)}$
    \vspace{2in}
  \item
    $\displaystyle{(x^2 + 3)e^x}$
  \end{enumerate}
\end{thm}

\newpage

\begin{thm}[12 Points]
Differentiate the following functions
\begin{enumerate}[(a)]
\item
  $\displaystyle{\frac{x + 1}{x - 1}}$
  \vspace{2in}
\item
  $\displaystyle{\frac{x}{e^x}}$
  \vspace{2in}
\item
  $\displaystyle{\frac{x}{\ln(x)}}$
\end{enumerate}
\end{thm}
\newpage

\begin{thm}[10 Points]
  Let $f(x) = 10x^4 - 4x^5$.
  \begin{enumerate}[(a)]
  \item
    Find the derivative of $f$.
    \vspace{1in}
  \item
    Find the critical points of $f$.\newline
    [Hint: Factoring after taking the derivative will make this task much easier.]
    \vspace{2in}
  \item
    Find any local maxima and local minima of $f$.
    Clearly indicate whether a point is a maximum or a minimum.
    \vspace{2in}
  \item
    Find any inflection points of $f$.\newline
    [Hint: Same as for part (b).]
  \end{enumerate}
\end{thm}

\newpage

\begin{thm}[12 Points]
  Find the global maximum and global minimum of $f(x) = 10x^4 - 4x^5$ on the interval $[1,3]$.
\end{thm}
\newpage
\begin{thm}[20 Points]
  A company sells a product for \$21 each and the manufacturing costs can be modeled by the function
  $$C(q) = \frac{1}{3}q^{3} - 2q^{2} + 100$$
  of $q$ units produced.
  For each of the quantities below, determine whether the company should increase, decrease, or not change the production levels in order to maximize profit.
  Justify your answers using calculus.
  {\bf You will not receive credit for guess and check solutions.}
  \begin{multicols}{3}
  \begin{enumerate}[(a)]
    \item
      $q = 3$
    \item
      $q = 7$
    \item
      $q = 9$
    \end{enumerate}
  \end{multicols}
\end{thm}
\end{document}
