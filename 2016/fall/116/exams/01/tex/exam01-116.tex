\documentclass[12pt]{amsart}
\usepackage{amsmath,amsthm,amssymb,amsfonts,enumerate,mymath,tikz-cd,fancyhdr,multicol}
\openup 5pt
\author{Blake Farman\\University of South Carolina}
\title{Math 116\\Exam 01}
\date{October 31, 2016}
\pdfpagewidth 8.5in
\pdfpageheight 11in
\usepackage[margin=1in]{geometry}

\renewcommand{\qedsymbol}{}
\theoremstyle{definition}
\newtheorem{thm}{}
\newtheorem{lem}{Lemma}
\theoremstyle{definition}
\newtheorem{defn}{Definition}
\begin{document}
\maketitle

\begin{center}
  \fbox{\fbox{\parbox{5.5in}{\centering
        Answer the questions in the spaces provided on the
        question sheets and turn them in at the end of the class period.
        If you require extra space, use the back of the page and indicate that you have done so.
        
        Unless otherwise stated, all supporting work is required.
        Unsupported or otherwise mysterious answers will {\bf not receive credit.}
        You may {\it not} use any calculators.}}}
\end{center}

\vspace{0.2in}
\makebox[\textwidth]{Name:\enspace\hrulefill}
\vspace{0.2in}

$$
\begin{array}{|c|c|c|}
  \hline
  \text{Problem} & \text{Points Earned} & \text{Points Possible}\\
  \hline
  1 & & 6\\
  \hline
  2 & & 1\\
  \hline
  3 & & 1\\
  \hline
  4 & & 2\\
  \hline
  5 & & 18 \\
  \hline
  6 & & 18\\
  \hline
  7 & & 18\\
  \hline
  8 & & 18\\
  \hline
  9 & & 18\\
  \hline
  \text{Total} & & 100\\
  \hline
\end{array}
$$

\newpage

\section{Definitions}

\begin{thm}[6 Points]\label{ex2}
  Let $a, b$ be non-zero real numbers and $m, n$ rational numbers.
  Fill in the blanks
  \begin{multicols}{2}
    \begin{enumerate}[(i)]
    \item
      $\displaystyle{a^0 =\ \line(1,0){60}}$,
      \vspace{.15in}
    \item
      $\displaystyle{\left(\frac{a}{b}\right)^{-n} =\ \line(1,0){60}}.$
      \vspace{.15in}
    \item
      $\displaystyle{a^m \cdot a^n =\ \line(1,0){60}}$
      \vspace{.15in}
    \item
      $\displaystyle{\frac{a^m}{a^n} =\ \line(1,0){60}}$
      \vspace{.15in}
    \item
      $\displaystyle{\left(a \cdot b\right)^n =\ \line(1,0){60}}$
      \vspace{.15in}
    \item
      $\displaystyle{\left(\frac{a}{b}\right)^n =\ \line(1,0){60}}$
    \end{enumerate}
  \end{multicols}
\end{thm}

\begin{thm}[1 Points]\label{ex3}
  State the Quadratic Formula.
  That is, state the solutions of the equation $ax^2 + bx + c = 0$.
  \vspace{1in}
\end{thm}

\begin{thm}[1 Points]\label{ex4}
  Fill in the blanks:\\
  \begin{center}
    To make $x^2 + bx$ a perfect square, add and subtract $\fbox{\raisebox{16px}{\hspace{16px}}}$\,.
    This gives
    $$x^2 + bx + \fbox{\raisebox{16px}{\hspace{16px}}} - \fbox{\raisebox{16px}{\hspace{16px}}} = \left(x + \fbox{\raisebox{16px}{\hspace{16px}}}\,\right)^2 - \fbox{\raisebox{16px}{\hspace{16px}}}.$$
  \end{center}
\end{thm}

\begin{thm}[2 Points]\label{ex1}
  \begin{enumerate}[(a)]
  \item
    State the Point-Slope form of a line passing through the point $(x_0, y_0)$ with slope $m$.
    \vspace{1in}
  \item
    State the Slope-Intercept form of a line with slope $m$ and $y$-intercept $b$.
  \end{enumerate}
\end{thm}
\newpage
\section{Exercises}

\begin{thm}[18 Points]\label{ex5}
  In the following problems, use the given information to find the equation of the line in Slope-Intercept Form and then graph the lines.
  \begin{enumerate}[(a)]
  \item
    The line passing through the point $\left(\frac{1}{2}, 3\right)$ and parallel to the line $4y + 8x = 16$.
    \vspace{3in}
  \item
    The line passing through the  point $\left(\frac{1}{3},5\right)$ and perpendicular to the line $3y - x = 9$.
    \vspace{2in}
  \end{enumerate}
  \vspace{1in}
\end{thm}

\newpage
\begin{thm}[18 Points]\label{ex9}
  Find all solutions, real and complex, to the equation
  $$x^2 + x + 1 = 0.$$
  \vspace{2in}
\end{thm}

\begin{thm}[18 Points]\label{ex10}
  \begin{enumerate}[(a)]
  \item
    Complete the square for the function $f(x) = x^2 - x - 6$.
    \vspace{1in}
  \item
    Solve $f(x) = 0$.
    \vspace{1in}
  \item
    Use the information from parts (a) and (b) to sketch a graph of $f(x)$.
    Label the $y$-intercept, any $x$-intercept(s), and the vertex.
    \vspace{2in}
  \end{enumerate}
\end{thm}

\newpage
\begin{thm}[18 Points]\label{ex9}
  Find all solutions, both real and complex, of the equation
  $$x + \sqrt{x} + 1 = 0.$$
  \vspace{3in}
\end{thm}

\begin{thm}[18 Points]\label{ex9}
  Find the simultaneous solutions to the following system
  $$\left\{\begin{array}{rcl}
    y &=& x^2 + 2x + 3,\\
    y &=& x + 5.
  \end{array}\right.$$
  \vspace{2in}
\end{thm}
\end{document}
