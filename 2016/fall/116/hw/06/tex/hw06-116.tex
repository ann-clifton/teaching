\documentclass[12pt]{amsart}
\usepackage{amsmath,amsthm,amssymb,amsfonts,enumerate,mymath,tikz-cd,fancyhdr}
\openup 5pt
\author{Blake Farman\\University of South Carolina}
\title{Math 116\\Homework 06}
\date{November 16, 2015}
\pdfpagewidth 8.5in
\pdfpageheight 11in
\usepackage[margin=1in]{geometry}

\theoremstyle{plain}
\newtheorem{thm}{}
\renewcommand{\qedsymbol}{}

\begin{document}
\maketitle

\section*{6.2}
\renewcommand{\exp}[1]{\operatorname{e}^{#1}}

\setcounter{thm}{1}
\begin{thm}
  Sketch
  \begin{enumerate}[(a)]
  \item
    $\displaystyle{y = -\exp{-x}}$
  \item
    $\displaystyle{y = -2\exp{-x}}$
  \item
    $\displaystyle{y = \exp{-x} + 1}$
  \item
    $\displaystyle{y = 3 - \exp{x}}$
  \item
    $\displaystyle{y = 2 - 3\exp{x}}$
  \end{enumerate}
\end{thm}

\setcounter{thm}{5}

\begin{thm}
  Simplify
  \begin{enumerate}[(a)]
  \item
    $\displaystyle{\left(\exp{-x}\right)^2}$
  \item
    $\displaystyle{\sqrt{\exp{2x}}}$
  \item
    $\displaystyle{\frac{\exp{x} + 1}{\exp{2x} - 1}}$
  \end{enumerate}
\end{thm}

\newpage

\section*{7.1}

\setcounter{thm}{1}
\begin{thm}
  Given the functions:\\
  $f(x) = x^2 + 1$, $g(x) = \sin(x)$, $s(t) = 2t - 3$, find the following composition functions:
  \begin{enumerate}[(a)]
  \item
    $\displaystyle{f(g(x))}$
  \item
    $\displaystyle{f(s(t))}$
  \item
    $\displaystyle{g(s(t))}$
  \item
    $\displaystyle{g(f(x))}$
  \item
    $\displaystyle{g(g(x))}$
  \end{enumerate}
\end{thm}

\setcounter{thm}{3}
\begin{thm}
  Suppose that $f(x) = x^3 + 4x$, $g(x) = \sqrt{x + 1}$, and $h(x) = \cos(x)$.
  Find: 
  \begin{enumerate}[(a)]
  \item
    $\displaystyle{f(g(h(x)))}$
  \item
    $\displaystyle{f(h(g(x)))}$
  \end{enumerate}
\end{thm}

\newpage
\section*{7.4}

In Exercises 2 and 6, find inverses, if they exist, of the given functions.  If they do not exist, explain why.
\setcounter{thm}{1}

\begin{thm}
  $\displaystyle{k(x) = \frac{x}{x+1}}$
\end{thm}

\setcounter{thm}{5}
\begin{thm}
  $f(w) = \displaystyle{\frac{w^2}{w^2 + 1}}$
\end{thm}

\newpage
\section*{8.2}

\setcounter{thm}{5}
\begin{thm}
  Solve $\displaystyle{\log_3(x - 3) = 2}$.
\end{thm}

\setcounter{thm}{7}
\begin{thm}
  Solve $\displaystyle{\log_9(x^2) = \frac{1}{2}}$
\end{thm}

\newpage
\section*{8.3}

\setcounter{thm}{5}
\begin{thm}
  Solve $\displaystyle{\log_2(x^2) - \log_2(3x - 8) = 2}$
\end{thm}

\setcounter{thm}{9}
\begin{thm}
  Solve $\displaystyle{\log(x) - \log(x-1) - 1 = 0}$
\end{thm}

\newpage
\section*{8.4}

\setcounter{thm}{7}
\begin{thm}
  Solve $\displaystyle{\exp{x^2 + 4x - 5} = 1}$.
\end{thm}

\setcounter{thm}{13}
\begin{thm}
  Solve $\displaystyle{\ln(x) - \ln(\sqrt{x}) - \frac{1}{2} = 0}$
\end{thm}

\end{document}
