\documentclass[12pt]{amsart}
\usepackage{amsmath,amsthm,amssymb,amsfonts,enumerate,mymath,tikz-cd,fancyhdr}
\openup 5pt
\author{Blake Farman\\University of South Carolina}
\title{Math 170\\Final Exam}
\date{May 5, 2014}
\pdfpagewidth 8.5in
\pdfpageheight 11in
\usepackage[margin=1in]{geometry}

\renewcommand{\qedsymbol}{}

\begin{document}
\maketitle

\begin{center}
  \fbox{\fbox{\parbox{5.5in}{\centering
        Answer the questions in the spaces provided on the
        question sheets and turn them in at the end of the class period.
        If you require extra space, use the back of the page and indicate that you have done so.
        
        Unless otherwise stated, all supporting work is required.
        Unsupported or otherwise mysterious answers will {\bf not receive credit.}}}}
\end{center}

\vspace{0.2in}
\makebox[\textwidth]{Name:\enspace\hrulefill}
\vspace{0.2in}

$$
\begin{array}{|c|c|c|}
  \hline
  \text{Problem} & \text{Points Earned} & \text{Points Possible}\\
  \hline
  1 & & 20\\
  \hline
  2 & & 20\\
  \hline
  3 & & 20\\
  \hline
  4 & & 20\\
  \hline
  5 & & 20 \\
  \hline
 % 6 & & 20 \\
%  \hline
%  \text{Bonus} & & 10\\
%  \hline
  \text{Total} & & 100\\
  \hline
\end{array}
$$

\newpage

\theoremstyle{plain}
\newtheorem{thm}{}
\newtheorem{lem}{Lemma}
\theoremstyle{definition}
\newtheorem{defn}{Definition}

\section{Problems}

\begin{thm}[10 Points]\label{ex4}
  Let $S$ be the set of all cars.
  Let $A$ be the subset of all cars that are blue.
  Let $B$ be the subset of all cars that are Volvos.
  \begin{enumerate}[(a)]
  \item
    In words, what does the set $A \cup B$ represent?
%    \vspace{1.5in}
  \item
    In words, what does the set $A \cap B$ represent?
%    \vspace{1.5in}
  \item
    In words, what do the sets $S \setminus A$ and $S \setminus B$ represent?
%    \vspace{1.5in}
  \item
    In words, what does the set $\left(S \setminus A\right) \cap \left(S \setminus B\right)$ represent?
 %   \vspace{1.5in}
  \item
    In words, what does the set $\left(S \setminus A\right) \cup \left(S \setminus B\right)$ represent?
  \end{enumerate}
  \begin{proof}[Solution]
    \begin{enumerate}[(a)]
    \item
      The set of all cars that are either blue or a Volvo.
    \item
      The set of all blue volvos.
    \item
      $S\setminus A$ is the set of all cars that are not blue.
      $S\setminus B$ is the set of all cars that are not volvos.
    \item
      The set of all cars that are not blue Volvos.
    \item
      The set of all cars that are either not blue or not a Volvo.
    \end{enumerate}
  \end{proof}
\end{thm}

\newpage

\begin{thm}[10 Points]\label{ex3}
  How many three letter sequences can be made using the five letters 
  \begin{center}q, u, a, k, e?\end{center}
  \begin{proof}[Solution]
    $$P(5,3) = \frac{5!}{(5 - 3)!} = 5 \cdot 4 \cdot 3 = 60.$$
  \end{proof}
    
\end{thm}

\newpage

\noindent Let $U = \{A,\, B,\, C, D,\, E,\, F,\, G\}$.
Let 
$$X = \{A,\, B,\, C\},$$
$$Y = \{A,\, B,\, E,\, F,\, G\}, \text{and}$$
$$Z = \{A,\, D,\, E,\, F,\, G\}.$$
\begin{thm}[10 Points]\label{ex4}
  Compute the following sets.
  \begin{enumerate}[(a)]
  \item
    $X \cap Y$,
%    \vspace{1in}
  \item
    $X \cup Z$,
%    \vspace{1in}
  \item
    The complement of $Z$ in $U$.
%    \vspace{1in}
  \end{enumerate}
  \begin{proof}[Solution]
    \begin{enumerate}[(a)]
    \item
      $X \cap Y = \{A, B\}$.
    \item
      $X \cup Z = \{A,B,C,D,E,F,G\} = U$.
    \item
      $U \setminus Z = \{B,C\}$.
    \end{enumerate}
  \end{proof}
\end{thm}

\newpage

\begin{thm}[10 Points]
  \begin{enumerate}[(a)]
  \item
    What is the cardinality of $X \times Z$?
%    \vspace{1in}
  \item
    What is the cardinality of $Y \cup Z$?
%    \vspace{1in}
  \end{enumerate}
  \begin{proof}[Solution]
    \begin{enumerate}[(a)]
    \item
      $\abs{X \times Z} = \abs{X}\cdot \abs{Z} = 3 \cdot 5 = 15$.
    \item
      $\abs{Y \cup Z} = \abs{Y} + \abs{Z} - \abs{Y \cap Z} = 5 + 5 - 4 = 6$.
    \end{enumerate}
  \end{proof}
\end{thm}
\newpage

\begin{thm}[10 Points]\label{ex9}
  Use a truth table to prove the following logical equivalences.
  \begin{enumerate}[(a)]
  \item
    $$p \vee q \equiv \neg p \Rightarrow q.$$
%    \vspace{2in}
  \item
    $$p \wedge q \equiv \neg\left(p \Rightarrow \neg q\right).$$
%    \vspace{2in}
  \end{enumerate}
  \begin{proof}[Solution]
    \begin{enumerate}[(a)]
    \item
      $$\begin{array}{c|c|c|c|c|c}
        p & q & p \vee q & \neg p & q & \neg p \Rightarrow q\\
        \hline
        T & T & T & F & T & T\\
        T & F & T & F & F & T\\
        F & T & T & T & T & T\\
        F & F & F & T & F & F
        \end{array}$$
      The logical equivalence follows because the third and sixth columns are the same.
      \item
        $$\begin{array}{c|c|c|c|c|c|c}
        p & q & p \wedge q & p & q & p \Rightarrow \neg q & \neg(p \Rightarrow \neg q)\\
        \hline
        T & T & T & T & F & F & T\\
        T & F & F & T & T & T & F\\
        F & T & F & F & F & T & F\\
        F & F & F & F & T & T & F
      \end{array}$$
        The logical equivalence follows because the third and seventh columns are the same.
    \end{enumerate}
  \end{proof}
\end{thm}

\newpage
\begin{thm}[15 Points]\label{ex2}
  Solve the system of equations
  \begin{eqnarray*}
    x + 2y + z &=& 0\\
    3x + 7y + z &=& 0.
  \end{eqnarray*}
  If there is no solution, simply write 'no solution.'  If the system is dependent, express your answer in terms of $z$.
%  \vspace{2in}
  \begin{proof}[Solution]
    Row reducing the associated augmented matrix gives the matrix
    $$\left(\begin{array}{cccc}
      1 & 0 & 5 & 0\\
      0 & 1 & -2 & 0
    \end{array}\right)$$
    which translates to two equations
    \begin{eqnarray*}
      x + 5z &=& 0\\
      y - 2z &=& 0
    \end{eqnarray*}
    Since we are asked to present our solution in terms of $z$, we solve the first for $x$ and the second for $y$.
    The first gives
    $$x = -5z,$$
    the second gives
    $$y = 2z,$$
    and the solutions are therefore
    $$\left\{(-5z, 2z, z) \;\mid\; z \in \R\right\}.$$
  \end{proof}
\end{thm}

\newpage

\begin{thm}[15 Points]\label{ex3}
  Solve the system of equations
  \begin{eqnarray*}
    x + 3y - z &=& 2\\
    2x + 5y - z &=& 2\\
    2x + 8y - 2z &=& 6
  \end{eqnarray*}
  If there is no solution, simply write 'no solution.'  If the system is dependent, express your answer in terms of $x$.
  \begin{proof}[Solution]
    The associated augmented matrix row reduces to the matrix
    $$\left(\begin{array}{cccc}
      1 & 0 & 0 & -2\\
      0 & 1 & 0 & 1\\
      0 & 0 & 1 & -1
    \end{array}\right)$$
    so the solution is $(-2,1,-1)$.
  \end{proof}
\end{thm}

\newpage

\begin{thm}[10 Points]
  Use {\bf matrix inversion} to solve the given equation
  $$\left(\begin{array}{rr}
    1 & 1 \\
    -3 & -2
  \end{array}\right) 
  \left(\begin{array}{r}
    x \\
    y
  \end{array}\right) = 
  \left(\begin{array}{r}
    5 \\
    7
  \end{array}\right)$$
  for $x$ and $y$.
  %\begin{eqnarray*}
  %    -4x + y &=& 4\\
  %    -4x - 3y &=& 4.
  %  \end{eqnarray*}
  \begin{proof}[Solutions]
    The determinant of 
    $$\left(\begin{array}{rr}
    1 & 1 \\
    -3 & -2
  \end{array}\right)$$
    is 
    $$1 \cdot -2 - 1 \cdot -3 = -2 + 3 = 1$$
    so the inverse is
    $$\left(\begin{array}{rr}
      -2 & -1 \\
      3 & 1
  \end{array}\right).$$
    The solution is therefore
    $$\left(\begin{array}{r}
    x \\
    y
  \end{array}\right) = 
    \left(\begin{array}{rr}
      -2 & -1 \\
      3 & 1
  \end{array}\right)
    \left(\begin{array}{r}
    5 \\
    7
  \end{array}\right) = \left(\begin{array}{r}
    -17 \\
    22
  \end{array}\right)$$
  \end{proof}
\end{thm}
\newpage


\begin{thm}[10 Points]\label{ex1}
  A bag contains five red marbles, two green marbles, one lavender marble, one yellow marble, and three orange marbles.
  \begin{enumerate}[(a)]
  \item
    How many sets of four marbles have only red marbles?
%    \vspace{2in}
  \item
    How many sets of four marbles do not have any orange marbles and have {\bf at most} two red marbles?
%    \vspace{2in}
  \end{enumerate}
  
  \begin{proof}[Solution]
    
    \begin{enumerate}[(a)]
    \item
      There are
      $${5 \choose 4} = \frac{5!}{(5 - 4)!4!} = \frac{5!}{4!} = 5$$
      ways to choose 4 marbles from the 5 red marbles.
    \item
      Note that without the orange marbles, there are 9 total marbles.
      There are three cases.
      Case 1: There are no red marbles.
      Without the red marbles, there are then 4 marbles from which to choose.
      Hence there is  
      $${4 \choose 4} = 1$$
      way to do this.
      
      Case 2: There is one red marble.
      There are
      $${5 \choose 1} = 5$$
      ways to choose a red marble and
      $${4 \choose 3} = 4$$
      ways to choose the other three marbles.
      Hence there are $5 \cdot 4 = 20$ ways to choose a set of four with one red marble.
      
      Case 3: There are two red marbles.
      There are 
      $${5 \choose 2} = 10$$
      ways to choose 2 red marbles and 
      $${4 \choose 2} = 6$$
      ways to choose the other two marbles.
      Hence there are $10 \cdot 6 = 60$ ways to choose a set of four with two red marbles.
      
      The number of ways to choose a set of four marbles with at most 2 red marbles is therefore
      $$1 + 20 + 60 = 81$$
    \end{enumerate}
  \end{proof}
\end{thm}

\newpage

%\begin{thm}[Bonus - 10 Points]\label{bonus}
%  Let 
%  $$A = \left(\begin{array}{cc} 
%    a & b\\
%    c & d
%  \end{array}\right)$$
%  be a matrix with entries real numbers.
%  You may assume that $a \neq 0$ and $ad - bc \neq 0$.
%  Use Gauss-Jordan row reduction on the appropriate augmented matrix to compute the inverse of $A$.
%  NOTE:  I would like you to explicitly compute the inverse of $A$, not simply write down the matrix $A^{-1}$.
%\end{thm}

%\newpage
%\section{Useful Formulae}

%\begin{itemize}
%  \newcommand{\INT}{\operatorname{INT}}
%  \newcommand{\PV}{\operatorname{PV}}
%  \newcommand{\FV}{\operatorname{FV}}
%  \newcommand{\PMT}{\operatorname{PMT}}
%\item
%  $\INT = \FV - \PV$
%\item
%  $\FV = \PV(1 + rt)$
%\item
%  $\FV = \PV \left(1 + \frac{r}{m}\right)^{mt}$
%\item
%  $r_\text{eff} = \displaystyle{\left( 1 + \frac{r_\text{nom}}{m}\right)^m - 1}$
%\item
%  $\FV = \PMT \displaystyle{\frac{(1 + i)^n - 1}{i}},\, \text{where}\ i = \frac{r}{m}\ \text{and}\ n = mt$
%\item
%  $\PV = \PMT \displaystyle{\frac{1 - (1 + i)^{-n}}{i}},\, \text{where}\ i = \frac{r}{m}\ \text{and}\ n = mt$
%\end{itemize}
\end{document}
