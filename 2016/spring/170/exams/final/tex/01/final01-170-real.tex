\documentclass[12pt]{amsart}
\usepackage{amsmath,amsthm,amssymb,amsfonts,enumerate,mymath,tikz-cd,fancyhdr}
\openup 5pt
\author{Blake Farman\\University of South Carolina}
\title{Math 170\\Final Exam}
\date{May 2, 2016}
\pdfpagewidth 8.5in
\pdfpageheight 11in
\usepackage[margin=1in]{geometry}

\renewcommand{\qedsymbol}{}

\begin{document}
\maketitle

\begin{center}
  \fbox{\fbox{\parbox{5.5in}{\centering
        Answer the questions in the spaces provided on the
        question sheets and turn them in at the end of the class period.
        If you require extra space, use the back of the page and indicate that you have done so.
        
        Unless otherwise stated, all supporting work is required.
        Unsupported or otherwise mysterious answers will {\bf not receive credit.}}}}
\end{center}

\vspace{0.2in}
\makebox[\textwidth]{Name:\enspace\hrulefill}
\vspace{0.2in}

$$
\begin{array}{|c|c|c|}
  \hline
  \text{Problem} & \text{Points Earned} & \text{Points Possible}\\
  \hline
  1 & & 10\\
  \hline
  2 & & 10\\
  \hline
  3 & & 10\\
  \hline
  4 & & 10\\
  \hline
  5 & & 10 \\
  \hline
  6 & & 15 \\
  \hline
  7 & & 15 \\
  \hline
  8 & & 10 \\
  \hline
  9 & & 10 \\
  \hline
  \text{Bonus} & & 10\\
  \hline
  \text{Total} & & 100\\
  \hline
\end{array}
$$

\newpage

\theoremstyle{plain}
\newtheorem{thm}{}
\newtheorem{lem}{Lemma}
\theoremstyle{definition}
\newtheorem{defn}{Definition}

\section{Problems}

\begin{thm}[10 Points]\label{ex4}
  Let $S$ be the set of all dogs.
  Let $A$ be the subset of all dogs that are labradors.
  Let $B$ be the subset of all dogs that are yellow.
  \begin{enumerate}[(a)]
  \item
    In words, what does the set $A \cup B$ represent?
    \vspace{1.5in}
  \item
    In words, what does the set $A \cap B$ represent?
    \vspace{1.5in}
  \item
    In words, what do the sets $S \setminus A$ and $S \setminus B$ represent?
    \vspace{1.5in}
  \item
    In words, what does the set $\left(S \setminus A\right) \cap \left(S \setminus B\right)$ represent?
    \vspace{1.5in}
  \item
    In words, what does the set $\left(S \setminus A\right) \cup \left(S \setminus B\right)$ represent?
  \end{enumerate}
\end{thm}

\newpage

\begin{thm}[10 Points]\label{ex3}
  How many three letter sequences without repetition can be made using the letters
  \begin{center}c, o, l, o, r, a, d, o?\end{center}
\end{thm}

\newpage

\noindent Let $U = \{A,\, B,\, C, D,\, E,\, F,\, G\}$.
Let 
$$X = \{B,\, D,\, E\},$$
$$Y = \{A,\, B,\, E,\, F,\, G\}, \text{and}$$
$$Z = \{A,\, E,\,F,\}.$$
Use these sets to answer questions 3 and 4.
\begin{thm}[10 Points]\label{ex4}
  Compute the following sets.
  \begin{enumerate}[(a)]
  \item
    $X \cap Y$,
    \vspace{1in}
  \item
    $X \cup Z$,
    \vspace{1in}
  \item
    The complement of $Z$ in $U$.
    \vspace{1in}
  \end{enumerate}
\end{thm}

\begin{thm}[10 Points]
  \begin{enumerate}[(a)]
  \item
    What is the cardinality of $X \times Z$?
    \vspace{1in}
  \item
    What is the cardinality of $Y \cup Z$?
    \vspace{1in}
  \end{enumerate}
\end{thm}
\newpage

\begin{thm}[10 Points]\label{ex9}
  Use a truth table to prove the following logical equivalences.
  \begin{enumerate}[(a)]
  \item
    $$q \wedge (p \Rightarrow q) \equiv q$$
    %V2: $$q \wedge (\neg q \Rightarrow \neg p) \equiv q$$
    \vspace{2in}
  \item
    $$p \wedge \neg q \equiv \neg(p \Rightarrow q)$$
    %V2: $$\neg p \vee q \equiv p \Rightarrow q$$
    \vspace{2in}
  \end{enumerate}
\end{thm}

\newpage
\begin{thm}[15 Points]\label{ex2}
  Solve the system of equations
  \begin{eqnarray*}
    x + 3y + 5z &=& 7\\
    2x + 4y + 6z &=& 8.
  \end{eqnarray*}
  If there is no solution, simply write 'no solution.'  If the system is dependent, express your answer in terms of $z$.
  \vspace{2in}
\end{thm}

\newpage

\begin{thm}[15 Points]\label{ex3}
  Solve the system of equations
  \begin{eqnarray*}
    x + 4y + 2z &=& 1\\
    2y + z &=& 2\\
    3x + 5y + 3z &=& 3
  \end{eqnarray*}
  If there is no solution, simply write 'no solution.'  If the system is dependent, express your answer in terms of $x$.
\end{thm}

\newpage

\begin{thm}[10 Points]
  Use {\bf matrix inversion} to solve the given equation
  $$\left(\begin{array}{rr}
    2 & 1 \\
    3 & 2
  \end{array}\right) \left(\begin{array}{r}
    x \\
    y
  \end{array}\right) = \left(\begin{array}{r}
    5 \\
    7
  \end{array}\right)$$
  for $x$ and $y$.
  %\begin{eqnarray*}
  %    -4x + y &=& 4\\
  %    -4x - 3y &=& 4.
  %  \end{eqnarray*}
  
\end{thm}
\newpage


\begin{thm}[10 Points]\label{ex1}
  A bag contains five red marbles, two green marbles, one lavender marble, one yellow marble, and three orange marbles.
  \begin{enumerate}[(a)]
  \item
    How many sets of four marbles have only green and orange marbles?
    \vspace{2in}
  \item
    How many sets of four marbles do not have any red marbles and have {\bf at most} two orange marbles?
    \vspace{2in}
  \end{enumerate}
\end{thm}

\newpage

%\begin{thm}[Bonus - 10 Points]\label{bonus}
%  Let 
%  $$A = \left(\begin{array}{cc} 
%    a & b\\
%    c & d
%  \end{array}\right)$$
%  be a matrix with entries real numbers.
%  You may assume that $a \neq 0$ and $ad - bc \neq 0$.
%  Use Gauss-Jordan row reduction on the appropriate augmented matrix to compute the inverse of $A$.
%  NOTE:  I would like you to explicitly compute the inverse of $A$, not simply write down the matrix $A^{-1}$.
%\end{thm}

%\newpage
%\section{Useful Formulae}

%\begin{itemize}
%  \newcommand{\INT}{\operatorname{INT}}
%  \newcommand{\PV}{\operatorname{PV}}
%  \newcommand{\FV}{\operatorname{FV}}
%  \newcommand{\PMT}{\operatorname{PMT}}
%\item
%  $\INT = \FV - \PV$
%\item
%  $\FV = \PV(1 + rt)$
%\item
%  $\FV = \PV \left(1 + \frac{r}{m}\right)^{mt}$
%\item
%  $r_\text{eff} = \displaystyle{\left( 1 + \frac{r_\text{nom}}{m}\right)^m - 1}$
%\item
%  $\FV = \PMT \displaystyle{\frac{(1 + i)^n - 1}{i}},\, \text{where}\ i = \frac{r}{m}\ \text{and}\ n = mt$
%\item
%  $\PV = \PMT \displaystyle{\frac{1 - (1 + i)^{-n}}{i}},\, \text{where}\ i = \frac{r}{m}\ \text{and}\ n = mt$
%\end{itemize}
\end{document}
