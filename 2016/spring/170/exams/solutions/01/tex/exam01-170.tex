\documentclass[12pt]{amsart}
\usepackage{amsmath,amsthm,amssymb,amsfonts,enumerate,mymath,tikz-cd,fancyhdr}
\openup 5pt
\author{Blake Farman\\University of South Carolina}
\title{Math 170\\ Exam 01}
\date{February 26, 2016}
\pdfpagewidth 8.5in
\pdfpageheight 11in
\usepackage[margin=1in]{geometry}

\renewcommand{\qedsymbol}{}

\begin{document}
\maketitle

\begin{center}
\fbox{\fbox{\parbox{5.5in}{\centering
      Answer the questions in the spaces provided on the
      question sheets and turn them in at the end of the class period.
      If you require extra space, use the back of the page and indicate that you have done so.
      
      Unless otherwise stated, all supporting work is required.
      Unsupported or otherwise mysterious answers will {\bf not receive credit.}}}}
\end{center}

\vspace{0.2in}
\makebox[\textwidth]{Name: Solutions}
\vspace{0.2in}

$$
\begin{array}{|c|c|c|}
  \hline
  \text{Problem} & \text{Points Earned} & \text{Points Possible}\\
  \hline
  1 & & 10\\
  \hline
  2 & & 20\\
  \hline
  3 & & 12\\
  \hline
  4 & & 18\\
  \hline
  5 & & 20 \\
  \hline
  6 & & 20 \\
  \hline
  \text{Total} & & 100\\
  \hline
\end{array}
$$

\theoremstyle{plain}
\newtheorem{thm}{}
\newtheorem{lem}{Lemma}
\theoremstyle{definition}
\newtheorem{defn}{Definition}

\newpage

%\section{Problems}

\begin{thm}[10 Points]\label{ex1}
  Let $S$ be the set of all students at the University of South Carolina.
  Let $A$ be the subset of all students taking Math 170 this semester.
  Let $B$ be the subset of all students majoring in business.
  \begin{enumerate}[(a)]
  \item
    In words, what does the set $A \cup B$ represent?
  \item
    In words, what does the set $A \cap B$ represent?
  \item
    In words, what do the sets $S \setminus A$ and $S \setminus B$ represent?
  \item
    In words, what does the set $\left(S \setminus A\right) \cap \left(S \setminus B\right)$ represent?
  \item
    In words, what does the set $\left(S \setminus A\right) \cup \left(S \setminus B\right)$ represent?
  \end{enumerate}
  \begin{proof}[Solution]
    \begin{enumerate}[(a)]
    \item
      The set $A \cup B$ is the set of all students at USC who are taking Math 170 this semester or are majoring in business (or both).
    \item
      The set $A \cap B$ is the set of all students at USC who are taking Math 170 this semester and are majoring in business.
    \item
      The set $S \setminus A$ is the set of students at USC who are not taking Math 170 this semester.
      The set $S \setminus B$ is the set of students at USC who are not majoring in business.
    \item
      The set $\left(S \setminus A\right) \cap \left(S \setminus B\right)$ is the set of students at USC who are not taking Math 170 this semester and are not majoring in business.
    \item
      The set $\left(S \setminus A\right) \cup \left(S \setminus B\right)$ is the set of students at USC who are not taking Math 170 this semester or are not majoring in business (or neither).
    \end{enumerate}
  \end{proof}
\end{thm}

\newpage
\begin{thm}[20 Points]\label{ex2}
  A bag contains five red marbles, two green marbles, one lavender marble, one yellow marble, and three orange marbles.
  The marbles are all distinguishable.
  \begin{enumerate}[(a)]
  \item
    How many sets of four marbles include none of the red ones?
  \item
    How many sets of four marbles include exactly one red marble?
  \end{enumerate}  
  \begin{proof}[Solution]
    First note that there are 
      $$5 + 2 + 1 + 1 + 3 = 12$$ 
      marbles in the bag and the sets are unordered.
    \begin{enumerate}[(a)]
    \item
      Since we don't want any red marbles in our set, we may remove the red marbles from the bag immediately, before we try to count the number of sets with four marbles, none of them red.
      Hence we have reduced the question to the following:
      \begin{center}
        {\it How many ways are there to choose four marbles from a set of seven?}
      \end{center}
      Since the sets are unordered, there are 
      $${7 \choose 4} = \frac{7!}{(7-4)! \cdot 4!} 
      = \frac{7!}{3! \cdot 4!} 
      = \frac{7 \cdot 6 \cdot 5 \cdot 4!}{3!\cdot 4!}
      = \frac{7 \cdot 6 \cdot 5}{3!}
      = \frac{7 \cdot 6 \cdot 5}{3 \cdot 2 \cdot 1}
      = \frac{7 \cdot 6 \cdot 5}{6}
      = 7 \cdot 5
      = 35$$
      ways to to choose a set of four marbles, none of them red.
    \item
      To solve this problem, we break it down into a sequence of choices.
      We note that we must always choose one of the five red marbles to be in our set; we may as well make this our first step.
      Once we have chosen one red marble, we must choose three other marbles from the remaining seven.
      There are 
      $${5 \choose 1} = \frac{5!}{(5 - 1)!\cdot1!}
      = \frac{5!}{4!\cdot 1}
      = \frac{5 \cdot 4!}{4!}
      = 5$$
      ways to choose the red marble and 
      $${7 \choose 3} = \frac{7!}{(7-4)! \cdot 4!}
      = \frac{7!}{3! \cdot 4!}
      = \frac{7!}{4! \cdot 3!}
      = {7 \choose 4}
      = 35$$
      ways to choose the remaining three marbles.
      Therefore by the Multiplication Principle there are 
      $$5 \cdot 35 = 175$$
      ways to choose a set of four marbles with exactly one red marble.
    \end{enumerate}
  \end{proof}
\end{thm}

\newpage

\begin{thm}[12 Points]\label{ex3}
  How many three letter sequences can be made using the six letters 
  \begin{center}a, e, f, r, u, d?\end{center}
  \begin{proof}[Solution]
    We first observe that the three lettered sequences are ordered; for example, the sequences ``red'' and ``der'' are {\it not} the same.
    Since there are no repeated letters, this tells us that we are counting the number of 3-permutations of a set with six elements.
    Therefore there are
    $$P(6,3) = \frac{6!}{(6 - 3)!} = \frac{6!}{3!} = \frac{6 \cdot 5 \cdot 4 \cdot 3!}{3!} = 6 \cdot 5 \cdot 4 = 120$$
    ways to create a three letter sequence with these six letters.
  \end{proof}
\end{thm}

\newpage

\noindent Let $U = \{A, B, C, D, E, F, G\}$.
  Let 
  $X = \{A,C,E\},$
  $Y = \{B, D, E, F, G\},\, \text{and}$
  $Z = \{B, C, E\}.$
  Use these sets to answer problems \ref{ex4} and \ref{ex5}.
\begin{thm}[18 Points]\label{ex4}
  Compute
  \begin{enumerate}[(a)]
  \item
    $X \cap Y$,
  \item
    $X \cup Z$,
  \item
    The complement of $Z$ in $U$, $U \setminus Z$.
  \end{enumerate}
  \begin{proof}[Solution]
    \begin{enumerate}[(a)]
    \item
      Recall that $X \cap Y$ is the set of common elements of $X$ and $Y$, so
      $$X \cap Y = \left\{ E\right\}.$$
    \item
      Recall that $X \cup Z$ is the set of elements that are either in $X$ or $Z$ (or both), so 
      $$X \cup Z = \left\{A,B,C,E\right\}.$$
    \item
      Recall that $U \setminus Z$ is the set of elements of $U$ that are not in $Z$, so
      $$U \setminus Z = \left\{A, D,F,G\right\}.$$
    \end{enumerate}
  \end{proof}
\end{thm}

\newpage

\begin{thm}[20 Points]\label{ex5}
  \begin{enumerate}[(a)]
  \item
    What is the cardinality of $X \times Z$?
  \item
    What is the cardinality of $Y \cup Z$?
  \end{enumerate}
  
  \begin{proof}[Solution]
    \begin{enumerate}[(a)]
    \item
      The cardinality of the cartesian product is given by the formula
      $$\abs{X \times Z} = \abs{X} \cdot \abs{Z} = 3 \cdot 3 = 9.$$
    \item
      The cardinality of the union is given by the formula
      $$\abs{Y \cup Z} = \abs{Y} + \abs{Z} - \abs{Y \cap Z} = 5 + 3 - 2 = 6.$$
    \end{enumerate}
  \end{proof}
\end{thm}

\newpage

\begin{thm}[20 Points]\label{ex9}
  Use a truth table to prove the following logical equivalences.
  \begin{enumerate}[(a)]
  \item
    $$\neg\left(p \vee q\right) \equiv \neg p \wedge \neg q.$$
  \item
    $$\neg\left(p \wedge q \right) \equiv \neg p \vee \neg q.$$
  \end{enumerate}
  
  \begin{proof}[Solution]
    \begin{enumerate}[(a)]
    \item
      The truth table for both propositions is given by
      \begin{center}
        \begin{tabular}{c|c|c|c|c|c|c}
          $p$ & $q$ & $p \vee q$ & $\neg(p \vee q)$ & $\neg p$ & $\neg q$ & $\neg p \wedge \neg q$\\
          \hline
          T & T & T & F & F & F & F\\
          T & F & T & F & F & T & F\\
          F & T & T & F & T & F & F\\
          F & F & F & T & T & T & T
        \end{tabular}
      \end{center}
      Comparing the columns for $\neg\left(p \vee q\right)$ and $\neg p \wedge \neg q$, we see they are the same and therefore are logically equivalent.
    \item
      The truth table for both propositions is given by
      \begin{center}
        \begin{tabular}{c|c|c|c|c|c|c}
          $p$ & $q$ & $p \wedge q$ & $\neg(p \wedge q)$ & $\neg p$ & $\neg q$ & $\neg p \vee \neg q$\\
          \hline
          T & T & T & F & F & F & F\\
          T & F & F & T & F & T & T\\
          F & T & F & T & T & F & T\\
          F & F & F & T & T & T & T
        \end{tabular}
      \end{center}
      Comparing the columns for $\neg\left(p \wedge q\right)$ and $\neg p \vee \neg q$, we see they are the same and therefore are logically equivalent.
    \end{enumerate}
  \end{proof}
\end{thm}

\newpage

%\begin{thm}[Bonus - 10 Points]\label{bonus}
%  Write out Modus Tollens symbolically.
%  State the conditions for an argument to be valid and then prove that Modus Tollens is a valid argument.
%  Can you give an example of how Modus Tollens is used?
%\end{thm}
\end{document}
