\documentclass[12pt]{amsart}
\usepackage{amsmath,amsthm,amssymb,amsfonts,enumerate,mymath,tikz-cd,fancyhdr}
\openup 5pt
\author{Blake Farman\\University of South Carolina}
\title{Math 170\\Exam 02}
\date{April 18, 2016}
\pdfpagewidth 8.5in
\pdfpageheight 11in
\usepackage[margin=1in]{geometry}

\renewcommand{\qedsymbol}{}

\begin{document}
\maketitle

\begin{center}
  \fbox{\fbox{\parbox{5.5in}{\centering
        Answer the questions in the spaces provided on the
        question sheets and turn them in at the end of the class period.
        If you require extra space, use the back of the page and indicate that you have done so.
        
        Unless otherwise stated, all supporting work is required.
        Unsupported or otherwise mysterious answers will {\bf not receive credit.}}}}
\end{center}

\vspace{0.2in}
\makebox[\textwidth]{Name:\enspace\hrulefill}
\vspace{0.2in}

\theoremstyle{plain}
\newtheorem{thm}{}
\newtheorem{lem}{Lemma}
\theoremstyle{definition}
\newtheorem{defn}{Definition}
$$
\begin{array}{|c|c|c|}
  \hline
  \text{Problem} & \text{Points Earned} & \text{Points Possible}\\
  \hline
  1 & & 20\\
  \hline
  2 & & 20\\
  \hline
  3 & & 20\\
  \hline
  4 & & 20\\
  \hline
  5 & & 20 \\
  \hline
 % 6 & & 20 \\
%  \hline
%  \text{Bonus} & & 10\\
%  \hline
  \text{Total} & & 100\\
  \hline
\end{array}
$$

\newpage

\section{Problems}

\begin{thm}[20 Points]\label{ex1}
  Consider the system of equations
  \begin{eqnarray*}
    2x + 3y &=& 1\\
    x + 2y &=& 2.
  \end{eqnarray*}
  \begin{enumerate}[(a)]
  \item\label{1.a}
    Write down the {\bf matrix equation} (not the augmented matrix) associated to this system.
    \vspace{1.5in}
  \item
    What is the inverse of the coefficient matrix?
    \vspace{2in}
  \item
    Use your answers from the previous parts to solve the system.
  \end{enumerate}
\end{thm}

\newpage

\begin{thm}[20 Points]\label{ex2}
  Solve the system
  \begin{eqnarray*}
    2x -3y &=& 3\\
    -x + 2y &=& 4
  \end{eqnarray*}
\end{thm}

\newpage

\begin{thm}[20 Points]\label{ex3}
  Solve the system of equations
  \begin{eqnarray*}
    x + 3y + z &=& 4\\
    3x + 3z &=& 12.
  \end{eqnarray*}
  If there is no solution, simply write 'no solution.'  If the system is dependent, express your answer in terms of $x$, where $y = y(x)$ and $z = z(x)$.
\end{thm}

\newpage

\begin{thm}[20 Points]\label{ex4}
  Solve the system of equations
  \begin{eqnarray*}
    x + z &=& 4\\
    2y    &=& 4\\
    3x - 3z &=& -6
  \end{eqnarray*}
  If there is no solution, simply write 'no solution.'  If the system is dependent, express your answer in terms of $x$, where $y = y(x)$ and $z = z(x)$.
\end{thm}


\newpage

\begin{thm}[20 Points]\label{ex5}
  Consider the system of equations
  \begin{eqnarray*}
    x + y &=& 1\\
    2x - y + z &=& 2\\
    -4x + y - 2z &=& 3.
  \end{eqnarray*}
  Given  
  $$\left(\begin{array}{ccc}
    1 & 1 & 0\\
    2 & -1 & 1\\
    -4 & 1 & -2
  \end{array}\right)^{-1} = \left(\begin{array}{ccc}
    1 & 2 & 1\\
    0 & -2 & -1\\
    -2 & -5 & -3
  \end{array}\right),$$
  solve the system.
\end{thm}

%\newpage

%\begin{thm}[Bonus - 10 Points]\label{bonus}
%\end{thm}

\end{document}
