\documentclass[12pt]{amsart}
\usepackage{amsmath,amsthm,amssymb,amsfonts,enumerate,mymath,tikz-cd,fancyhdr}
\openup 5pt
\author{Blake Farman\\University of South Carolina}
\title{Math 170\\Practice}
\date{April 13, 2016}
\pdfpagewidth 8.5in
\pdfpageheight 11in
\usepackage[margin=1in]{geometry}

\renewcommand{\qedsymbol}{}

\begin{document}
\maketitle

\begin{center}
  \fbox{\fbox{\parbox{5.5in}{\centering
        Answer the questions in the spaces provided on the
        question sheets and turn them in at the end of the class period.
        If you require extra space, use the back of the page and indicate that you have done so.
        
        Unless otherwise stated, all supporting work is required.
        Unsupported or otherwise mysterious answers will {\bf not receive credit.}}}}
\end{center}

\vspace{0.2in}
\makebox[\textwidth]{Name:\enspace\hrulefill}
\vspace{0.2in}

\theoremstyle{plain}
\newtheorem{thm}{}
\newtheorem{lem}{Lemma}
\theoremstyle{definition}
\newtheorem{defn}{Definition}
$$
\begin{array}{|c|c|c|}
  \hline
  \text{Problem} & \text{Points Earned} & \text{Points Possible}\\
  \hline
  1 & & 20\\
  \hline
  2 & & 20\\
  \hline
  3 & & 20\\
  \hline
  4 & & 20\\
  \hline
  5 & & 20 \\
  \hline
 % 6 & & 20 \\
%  \hline
%  \text{Bonus} & & 10\\
%  \hline
  \text{Total} & & 100\\
  \hline
\end{array}
$$

\newpage

\section{Problems}

\begin{thm}[20 Points]\label{ex1}
  Consider the system of equations
  \begin{eqnarray*}
    2x + 3y &=& 4\\
    5x + 7y &=& 4.
  \end{eqnarray*}
  \begin{enumerate}[(a)]
  \item\label{1.a}
    Write down the {\bf matrix equation} (not the augmented matrix) associated to this system.
    %\vspace{1.5in}
  \item
    What is the inverse of the coefficient matrix?
    %\vspace{2in}
  \item
    Use your answers from the previous parts to solve the system.
  \end{enumerate}
  \begin{proof}[Solution]
    \begin{enumerate}[(a)]
    \item
      The associated matrix equation is
      $$\left(\begin{array}{cc}
        2 & 3\\
        5 & 7
      \end{array}
      \right)
      \left(\begin{array}{c}
        x\\
        y
      \end{array}
      \right)
      =
      \left(\begin{array}{c}
        4\\
        4
      \end{array}
      \right)$$
    \item
      The inverse of the coefficient matrix
      $$A = \left(\begin{array}{cc}
        2 & 3\\
        5 & 7
      \end{array}
      \right)$$
      is
      $$A^{-1} = \frac{1}{2\cdot 7 - 3 \cdot 5} \cdot \left(\begin{array}{ccc}
            7 & -3\\
            -5 & 2
        \end{array}
      \right) = 
      -1 \cdot \left(\begin{array}{ccc}
            7 & -3\\
            -5 & 2
        \end{array}
      \right)
      = \left(\begin{array}{ccc}
            -7 & 3\\
            5 & -2
        \end{array}
      \right).$$
    \item
      The solution to the system is given by 
      $$\left(\begin{array}{ccc}
            -7 & 3\\
            5 & -2
        \end{array}
      \right)
      \left(\begin{array}{c}
        4\\
        4
      \end{array}
      \right) 
      = 
      \left(\begin{array}{c}
        -16\\
        12
      \end{array}
      \right).$$
      That is, the solution to the system is the point in $\mathbb{R}^2$ with coordinates (-16,12).
    \end{enumerate}
    
    Note that the reason this is the solution is the following: Multiply both sides of the equation from part (\ref{1.a}) by $A^{-1}$ on the left 
    $$\left(\begin{array}{ccc}
            -7 & 3\\
            5 & -2
        \end{array}
      \right)\left(\begin{array}{cc}
        2 & 3\\
        5 & 7
      \end{array}
      \right)
      \left(\begin{array}{c}
        x\\
        y
      \end{array}
      \right)
      =
      \left(\begin{array}{ccc}
            -7 & 3\\
            5 & -2
        \end{array}
      \right)\left(\begin{array}{c}
        4\\
        4
      \end{array}
      \right) = \left(\begin{array}{c}
      -16\\
      12\end{array}\right).$$
      Carrying out the multiplication $A^{-1} \cdot A$ gives the 2x2 identity matrix, so we have the equation
      $$\left(\begin{array}{cc}
        1 & 0\\
        0 & 1
      \end{array}
      \right)\left(\begin{array}{c}
        x\\
        y
      \end{array}
      \right) = 
      \left(\begin{array}{c}
        x\\
        y
      \end{array}
      \right)
      = \left(\begin{array}{c}
      -16\\
      12\end{array}\right).$$
  \end{proof}
\end{thm}

\newpage

\begin{thm}[20 Points]\label{ex2}
  Solve the system
  \begin{eqnarray*}
    2x + 3y &=& 9\\
    5x + 7y &=& 19
  \end{eqnarray*}
  \begin{proof}[Solution]
    Note that we've already done most of the work in the previous problem.
    The associated matrix equation is 
          $$\left(\begin{array}{cc}
        2 & 3\\
        5 & 7
      \end{array}
      \right)
      \left(\begin{array}{c}
        x\\
        y
      \end{array}
      \right)
      =
      \left(\begin{array}{c}
        9\\
        19
      \end{array}
      \right),$$
      so the solution is
      $$\left(\begin{array}{ccc}
            -7 & 3\\
            5 & -2
        \end{array}
      \right)
      \left(\begin{array}{c}
        9\\
        19
      \end{array}
      \right) 
      = 
      \left(\begin{array}{c}
        -6\\
        7
      \end{array}
      \right).$$
      
  \end{proof}
\end{thm}

\newpage

\begin{thm}[20 Points]\label{ex3}
  Solve the system of equations
  \begin{eqnarray*}
    x - y + 7z &=& 4\\
    x - y + 8z &=& 3.
  \end{eqnarray*}
  If there is no solution, simply write 'no solution.'  If the system is dependent, express your answer in terms of $x$, where $y = y(x)$ and $z = z(x)$.
  \begin{proof}[Solution]
    The augmented matrix associated to this system is
    $$\left(\begin{array}{cccc}
      1 & -1 & 7 & 4\\
      1 & -1 & 8 & 3
    \end{array}
    \right).$$
    Performing Gauss-Jordan row reduction, we obtain the augmented matrix
    $$\left(\begin{array}{cccc}
      1 & -1 & 0 & 11\\
      0 & 0 & 1 & -1
    \end{array}
    \right).$$
    This is a dependent system, and the reduced augmented matrix gives us the two equations
    \begin{eqnarray*}
      x - y &=& 11\\
      z &=& -1
    \end{eqnarray*}
    Since we are asked to write our solutions in terms of $x$, we solve the first equation for $y$ in terms of $x$:
    $$y = x - 11$$
    and so we have that the set of solutions is
    $$\left\{(x, x - 11, -1)  \;\mid\; x\ \text{is a real number}\right\}.$$
  \end{proof}
\end{thm}

\newpage

\begin{thm}[20 Points]\label{ex4}
  Solve the system of equations
  \begin{eqnarray*}
    2x - y &=& 0\\
    x + y + z &=& 18\\
    x - z &=& 2
  \end{eqnarray*}
  If there is no solution, simply write 'no solution.'  If the system is dependent, express your answer in terms of $x$, where $y = y(x)$ and $z = z(x)$.
  \begin{proof}[Solution]
    The associated augmented matrix is
    $$\left(\begin{array}{cccc}
      2 & -1 & 0 & 0\\
      1 & 1 & 1 & 18\\
      1 & 0 & -1 & 2
    \end{array}\right)$$
    which reduces by Guass-Jordan row reduction to
    $$\left(\begin{array}{cccc}
      1 & 0 & 0 & 5\\
      0 & 1 & 0 & 10\\
      0 & 0 & 1 & 3
    \end{array}\right).$$
    Therefore the solution to the system is the point in $\mathbb{R}^3$ with coordinates
    $$(5,10,3).$$
  \end{proof}
\end{thm}


\newpage

\begin{thm}[20 Points]\label{ex5}
  Consider the system of equations
  \begin{eqnarray*}
    2x + z &=& 0\\
    2x + y - z &=& 1\\
    3x + y - z &=& 2.
  \end{eqnarray*}
  Given  
  $$\left(\begin{array}{ccc}
    2 & 0 & 1\\
    2 & 1 & -1\\
    3 & 1 & -1
  \end{array}\right)^{-1} = \left(\begin{array}{ccc}
  0 & -1 & 1\\
  1 & 5 & -4\\
  1 & 2 & -2
  \end{array}\right),$$
  solve the system.
  \begin{proof}[Solution]
    Since we are given the inverse of the coefficient matrix, our solution is
    $$\left(\begin{array}{c}
      x\\
      y\\
      z
    \end{array}\right) = 
    \left(\begin{array}{ccc}
      0 & -1 & 1\\
      1 & 5 & -4\\
      1 & 2 & -2
  \end{array}\right)
    \left(\begin{array}{c}
      0\\
      1\\
      2
    \end{array}\right) = 
    \left(\begin{array}{c}
      1\\
      -3\\
      -2
    \end{array}\right).$$
    That is, the solution is the point in $\mathbb{R}^3$ with coordinates
    $$(1, -3, -2).$$
  \end{proof}
\end{thm}

%\newpage

%\begin{thm}[Bonus - 10 Points]\label{bonus}
%\end{thm}

\end{document}
