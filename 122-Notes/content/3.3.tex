\documentclass[Lecture.tex]{subfiles}
\begin{document}
\section{3.3: The Chain Rule}
\begin{frame}{The Chain Rule}
  \begin{thm}
    Let $f$ and $g$ be differentiable functions such that $f\circ g(x)$ is well-defined.
    \onslide<2->{The derivative of the composition is given by
    $$(f\circ g)^\prime(x) = \left(f^\prime \circ g(x)\right) \cdot g^\prime(x).$$}
  \end{thm}
\end{frame}

\begin{frame}{The Derivative of Arbitrary Exponentials}
  Let $P(t) = P_0a^t$.\\
  \onslide<2->{
    Let $f(t) = P_0e^t$ and let $g(t) = \ln(a)t$ so
    $$(f \circ g)(t) \onslide<3->{= f(\ln(a)t)} \onslide<4->{= P_0e^{\ln(a)t}} \onslide<5->{= P_0(e^{\ln(a)})^t}\onslide<6->{ = P_0a^t}\onslide<7->{ = P(t).}$$
  }
  \onslide<8->{
    Hence 
    \begin{eqnarray*}
      \onslide<8->{P^\prime(t) &=& f^\prime \circ g(t) \cdot g^\prime(t)\\}
      \onslide<9->{&=& P_0e^{\ln(a)t} \cdot \ln(a)\\}
      \onslide<10->{&=& P_0a^t \cdot \ln(a)\\}
      \onslide<11->{&=& \ln(a)P(t).}
    \end{eqnarray*}
  }
\end{frame}

\begin{frame}{Example}
  Differentiate $(x + 5)^2$.
  \begin{minipage}[t]{\linewidth}
      \onslide<2->{First we identify this function as a composition.}
      \onslide<3->{If we let $f(x) = \underline{\onslide<4->{x^2}}$ and $g(x) = \underline{\onslide<4->{x + 5}}$, then $f \circ g(x) = \left(x+5\right)^2$.}
      \begin{itemize}
        \item<5->
          $f^\prime(x) = 2x.$
        \item<6->
          $g^\prime(x) = 
          \onslide<7->{\ddx{x}(x + 5) =} 
          \onslide<8->{\ddx{x}(x) + \ddx{x}(5) =}
          \onslide<9->{1 + 0 =}
          \onslide<10->{1.}$
        \item<11->
          $f^\prime \circ g(x) = 
          \onslide<12->{f^\prime\left(g(x)\right) =} 
          \onslide<13->{f^\prime(x + 5) =}
          \onslide<14->{2(x + 5) =}
          \onslide<15->{2x + 10.}$
        \item<16->
          Therefore by the Chain Rule
          \begin{eqnarray*}
            \ddx{x}(x + 5)^2 &=& 
            \onslide<17->{\ddx{x}\left(f \circ g(x)\right) =} 
            \onslide<18->{\left(f^\prime\circ g(x)\right)\cdot g^\prime(x)\\}
            \onslide<19->{&=& (2x + 10) \cdot 1\\}
            \onslide<20->{&=& 2x + 10.}
          \end{eqnarray*}
      \end{itemize}
  \end{minipage}
\end{frame}

\begin{frame}{Example}
  Differentiate $e^{3x}$.
  \begin{minipage}[t]{\linewidth}
      \onslide<2->{First we identify this function as a composition.}
      \onslide<3->{If we let $f(x) = \underline{\onslide<4->{e^x}}$ and $g(x) = \underline{\onslide<4->{3x}}$, then $f \circ g(x) = e^{3x}$.}
      \begin{itemize}
        \item<5->
          $f^\prime(x) = e^x.$
        \item<6->
          $g^\prime(x) = 
          \onslide<7->{\ddx{x}(3x) =} 
          \onslide<8->{3\ddx{x}(x)=}
          \onslide<9->{3(1) =}
          \onslide<10->{3.}$
        \item<11->
          $f^\prime \circ g(x) = 
          \onslide<12->{f^\prime\left(g(x)\right) =} 
          \onslide<13->{f^\prime(3x) =}
          \onslide<14->{e^{3x}.}$
        \item<15->
          Therefore by the Chain Rule
          \begin{eqnarray*}
            \ddx{x}\left(e^{3x}\right) &=& 
            \onslide<16->{\ddx{x}\left(f \circ g(x)\right) =} 
            \onslide<17->{\left(f^\prime\circ g(x)\right)\cdot g^\prime(x)\\}
            \onslide<18->{&=& e^{3x} \cdot 3\\}
            \onslide<19->{&=& 3e^{3x}.}
          \end{eqnarray*}
      \end{itemize}
  \end{minipage}
\end{frame}

\begin{frame}{Example}
  Differentiate $\left(\ln\left(2t^2 + 3\right)\right)^2$.
  \begin{minipage}[t]{\linewidth}
    \onslide<2->{If we let $f(t) = \underline{\onslide<3->{t^2}}$ and $g(t) = \underline{\onslide<3->{\ln\left(2t^2 + 3\right)}}$, then \\$f \circ g(t) = \left(\ln\left(2t^2 + 3\right)\right)^2.$}\\
    \only<4-17>{
      \begin{minipage}[t]{\linewidth}
        To differentiate $g$, we need to use the Chain Rule!\\
        \onslide<5->{If we let $h(t) = \underline{\onslide<6->{\ln(t)}}$ and $k(t) = \underline{\onslide<6->{2t^2 + 3}}$, then $h \circ k(t) = \left(\ln\left(2t^2 + 3\right)\right)^2$.}
        \begin{itemize}
        \item<7->
          $h^\prime(t) = \frac{1}{t}.$
        \item<8->
          $k^\prime(t) = 
          \onslide<9->{\ddx{t}(2t^2 + 3) =} 
          \onslide<10->{\ddx{t}(2t^2) + \ddx{t}(3))=}
          \onslide<11->{2\ddx{t}(t^2) + 0 =}
          \onslide<12->{4t.}$
        \item<13->
          $h^\prime \circ k(t) = 
          \onslide<14->{h^\prime\left(k(t)\right) =} 
          \onslide<15->{h^\prime(2t^2 + 3) =}
          \onslide<16->{\frac{1}{2t^2 + 3}.}$
        \item<17->
          Therefore by the Chain Rule, 
          $$g^\prime(t) = \frac{1}{2t^2 + 3} \cdot 4t = \frac{4t}{2t^2 + 3}.$$
      \end{itemize}
      \end{minipage}
    }
    \onslide<18->{So we have:}
    \begin{itemize}
      \item<19->
        $g^\prime(t) = \frac{4t}{2t^2 + 3}$
      \item<20->
        $f^\prime(t) = 2t$
      \item<21->
        $f^\prime \circ g(t) =$
        \onslide<22->{$f^\prime(ln(2t^2 + 3)) = $}
        \onslide<23->{$2ln(2t^2 + 3)$}
    \end{itemize}
    \begin{eqnarray*}
      \onslide<24->{\ddx{t}\left(\ln\left(2t^2 + 3\right)\right)^2 &=&}
      \onslide<25->{\left(f^\prime \circ g(t)\right) \cdot g^\prime(t)\\}
      \onslide<26->{&=& 2ln(2t^2 + 3) \cdot \frac{4t}{2t^2 + 3}\\}
      \onslide<27->{&=& \frac{8t\ln(2t^2 + 3)}{2t^2 + 3}.}
    \end{eqnarray*}
  \end{minipage}
\end{frame}

\begin{frame}{Remark}
  The last problem is a specific example of iterated use of the chain rule.
  \onslide<3->{We could decompose our function as the composition of the three functions $f(t) = t^2$, $g(t) = ln(t)$, $h(t) = 2t^2 + 3$:}
  \begin{eqnarray*}
    \onslide<4->{\ddx{t}\left(f \circ (g \circ h)(t)\right) &=&}
    \onslide<5->{(f^\prime \circ (g \circ h)(t)) \cdot \ddx{t}(g \circ h(t))\\}
    \onslide<6->{&=& \left(f^\prime \circ (g \circ h)(t)\right) \cdot (g^\prime \circ h(t)) \cdot h^\prime(t)\\}
    \onslide<7->{&=& f^\prime(\ln(2t^2 + 3)) \cdot g^\prime(2t^2 + 3) \cdot 4t\\}
    \onslide<8->{&=& 2\ln(2t^2 + 3) \cdot \frac{1}{2t^2 + 3} \cdot 4t\\}
    \onslide<9->{&=& \frac{8t\ln(2t^2 + 3)}{2t^2 + 3}.}
  \end{eqnarray*}
\end{frame}

\begin{frame}{Example}
  \begin{itemize}
  \item<1->
    The amount of gas, $G$, in gallons, consumed by a car depends on the distance, $s$, traveled in miles, which in turn depends on the time traveled, $t$.
  \item<2->
    If the car consumes 0.05 gallons for each mile traveled and the car is traveling 30 mph, then how fast is the gas being consumed?
  \item<3->
    \begin{eqnarray*}
      \onslide<3->{\frac{\operatorname{d}}{\operatorname{dt}}(G \circ s(t))} 
      \onslide<4->{&=& \left(G^\prime \circ s(t)\right) \cdot s^\prime(t)\\}
        \onslide<5->{&=& 0.05\frac{\text{gal}}{\text{mile}} \cdot 30 \frac{\text{miles}}{\text{hour}}\\}
        \onslide<6->{&=& 1.5 \frac{\text{gal}}{\text{hour}}.}
      \end{eqnarray*}
  \end{itemize}
\end{frame}
\end{document}
