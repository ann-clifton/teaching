\documentclass[Lecture.tex]{subfiles}
\begin{document}

\section{2.1: Instantaneous Rate of Change}
\begin{frame}{Definition}
  \begin{defn}
    The {\it instantaneous rate of change} of $f$ at $a$ is defined to be the limit of the average rates of change of $f$ over successively smaller intervals around $a$.\\
  This is also known as the {\it derivative of $f$ at $a$}.
  \end{defn}
\end{frame}

\begin{frame}{Example}
  The quadratic
  $$s(t) = -4.9t^2 + 9.8t$$
  models the position of an object thrown vertically into the air with an initial velocity of $9.8$ m/s.
  \onslide<2->{What is the instantaneous rate of change at the vertex, where $t = 1$?}
\end{frame}

\begin{frame}{Example (Cont.)}
  Here are some values:
  \begin{center}
    \begin{tabular}{ll}
      \onslide<1->{t & $\frac{f(t) - f(1)}{t - 1}$\\}
      \onslide<2->{0 & 4.9\\}
      \onslide<3->{0.9 & $\approx$ 0.49\\}
      \onslide<4->{0.99 & $\approx$ 0.049\\}
      \onslide<5->{0.999 & $\approx$ 0.0049\\}
      \onslide<6->{0.9999 & $\approx$ 0.00049\\}
      \onslide<7->{0.99999 & $\approx$ 0.000049\\}
      \onslide<8->{0.999999 & $\approx$ 0.0000049\\}
    \end{tabular}
  \end{center}
  \onslide<9->{So, we would guess that the instantaneous rate of change is 0 at $t = 1$.}
\end{frame}

\begin{frame}{Animation}
  \animategraphics[loop,controls,scale=0.3]{12}{limit/limit-}{0}{99}
\end{frame}
\end{document}
