\documentclass[12pt]{amsart}
\usepackage{amsmath,amsthm,amssymb,amsfonts,enumerate,mymath,tikz-cd,fancyhdr}
\openup 5pt
\author{Blake Farman\\University of South Carolina}
\title{Math 170: Final Exam}
\date{May 5, 2014}
\pdfpagewidth 8.5in
\pdfpageheight 11in
\usepackage[margin=1in]{geometry}

\renewcommand{\qedsymbol}{}

\begin{document}
\maketitle

\begin{center}
\fbox{\fbox{\parbox{5.5in}{\centering
Answer the questions in the spaces provided on the
question sheets and turn them in at the end of the class period. 
Unless otherwise stated, all supporting work is required.}}}
\end{center}

\vspace{0.2in}
\makebox[\textwidth]{Name:\enspace\hrulefill}
\vspace{0.2in}

\theoremstyle{plain}
\newtheorem{thm}{}
\newtheorem{lem}{Lemma}
\theoremstyle{definition}
\newtheorem{defn}{Definition}

\section{Problems}

\noindent In problem 1, use the given information to set up the appropriate equation.
You need not carry out the computation.
On the last page you will find a selection of potentially useful formulae.
\begin{thm}[10 Points]\label{ex3}
  Your local bank offers an account that pays $6\%$ per year with monthly compounding interest.
  You wish to set up an account that will pay $\$10,000$ monthly over the course of $10$ years.
  How much must you deposit into the account to reach your goal?
\end{thm}

\newpage

\noindent Let $U = \{A,\, B,\, C, D,\, E,\, F,\, G\}$.
  Let 
  $$X = \{A,\, B,\, G\},$$
  $$Y = \{A,\, B,\, D,\, E,\, F,\, G\}, \text{and}$$
  $$Z = \{A,\, D,\, F,\, G\}.$$
\begin{thm}[10 Points]\label{ex4}
  Compute the following sets.
  \begin{enumerate}[(a)]
  \item
    $X \cap Y$,
    \vspace{1in}
  \item
    $X \cup Z$,
    \vspace{1in}
  \item
    The complement of $Z$ in $U$.
    \vspace{1in}
  \end{enumerate}
\end{thm}

\begin{thm}[10 Points]
  \begin{enumerate}[(a)]
  \item
    What is the cardinality of $X \times Z$?
    \vspace{1in}
  \item
    What is the cardinality of $Y \cup Z$?
    \vspace{1in}
  \end{enumerate}
\end{thm}
\newpage

\begin{thm}[10 Points]\label{ex9}
  Use a truth table to prove the following logical equivalences.
  \begin{enumerate}[(a)]
  \item
    $$p \Rightarrow q \equiv \neg p \vee q.$$
    \vspace{2in}
  \item
    $$p \Rightarrow q \equiv \neg q \Rightarrow \neg p.$$
    \vspace{2in}
  \end{enumerate}
\end{thm}

\newpage
\begin{thm}[15 Points]\label{ex2}
  Use Gauss-Jordan row reduction to solve the system of equations
  \begin{eqnarray*}
    3x - 3y + 21z &=& 0\\
    4x - 4y + 32z &=& 0.
  \end{eqnarray*}
  If there is no solution, simply write 'no solution.'  If the system is dependent, express your answer in terms of $x$.
  \vspace{2in}
\end{thm}

\begin{thm}[15 Points]\label{ex3}
  Use Gauss-Jordan row reduction to solve the system of equations
  \begin{eqnarray*}
    2x + 4y + 2z &=& 0\\
    -2x - 2y + 2z &=& 0\\
    2x + 6y + 9z  &=& 0
  \end{eqnarray*}
  If there is no solution, simply write 'no solution.'  If the system is dependent, express your answer in terms of $x$.
\end{thm}

\newpage

\begin{thm}[10 Points]\label{ex4}
  Compute the matrix product,
  $$\left(\begin{array}{cc}
    2 & 1\\
    1 & 1
  \end{array}\right)
  \left(\begin{array}{cc}
    1 & 2\\
    2 & 1
  \end{array}\right).$$
  \vspace{2in}
\end{thm}

\begin{thm}[10 Points]
  Use {\bf matrix inversion} to solve the equation
  $$\left(\begin{array}{rr}
    1 & 1 \\
    -3 & -2
  \end{array}\right) \left(\begin{array}{r}
    x \\
    y
  \end{array}\right) = \left(\begin{array}{r}
    3 \\
    5
  \end{array}\right)$$
  for $x$ and $y$.
\end{thm}
\newpage

\begin{thm}[10 Points]\label{ex1}
  A bag contains five red marbles, two green marbles, one lavender marble, one yellow marble, and three orange marbles.
  \begin{enumerate}[(a)]
  \item
    How many sets of four marbles have three orange marbles?
    \vspace{2in}
  \item
    How many sets of four marbles do not have any red marbles and have {\bf at most} two orange marbles?
    \vspace{2in}
  \end{enumerate}
\end{thm}

\newpage

\begin{thm}[Bonus - 10 Points]\label{bonus}
  Let 
  $$A = \left(\begin{array}{cc} 
    a & b\\
    c & d
  \end{array}\right)$$
  be a matrix with entries non-zero real numbers.
  Use Gauss-Jordan row reduction on the appropriate augmented matrix to compute the inverse of $A$ and explain why $A$ is invertible only if
  $$\det{A} = ad - bc \neq 0.$$
  NOTE:  I would like you to explicitly compute the inverse of $A$, not simply write down the matrix $A^{-1}$.
\end{thm}

\newpage
\section{Useful Formulae}

\begin{itemize}
  \newcommand{\INT}{\operatorname{INT}}
  \newcommand{\PV}{\operatorname{PV}}
  \newcommand{\FV}{\operatorname{FV}}
  \newcommand{\PMT}{\operatorname{PMT}}
\item
  $\INT = \FV - \PV$
\item
  $\FV = \PV(1 + rt)$
\item
  $\FV = \PV \left(1 + \frac{r}{m}\right)^{mt}$
\item
  $r_\text{eff} = \displaystyle{\left( 1 + \frac{r_\text{nom}}{m}\right)^m - 1}$
\item
  $\FV = \PMT \displaystyle{\frac{(1 + i)^n - 1}{i}},\, \text{where}\ i = \frac{r}{m}\ \text{and}\ n = mt$
\item
  $\PV = \PMT \displaystyle{\frac{1 - (1 + i)^{-n}}{i}},\, \text{where}\ i = \frac{r}{m}\ \text{and}\ n = mt$
\end{itemize}
\end{document}
