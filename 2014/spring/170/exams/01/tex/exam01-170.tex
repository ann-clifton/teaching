\documentclass[12pt]{amsart}
\usepackage{amsmath,amsthm,amssymb,amsfonts,enumerate,mymath,tikz-cd,fancyhdr}
\openup 5pt
\author{Blake Farman\\University of South Carolina}
\title{Math 170: Exam 01}
\date{February 26, 2014}
\pdfpagewidth 8.5in
\pdfpageheight 11in
\usepackage[margin=1in]{geometry}

\renewcommand{\qedsymbol}{}

\begin{document}
\maketitle

\begin{center}
\fbox{\fbox{\parbox{5.5in}{\centering
Answer the questions in the spaces provided on the
question sheets and turn them in at the end of the class period. 
Unless otherwise stated, all supporting work is required.}}}
\end{center}

\vspace{0.2in}
\makebox[\textwidth]{Name:\enspace\hrulefill}
\vspace{0.2in}

\theoremstyle{plain}
\newtheorem{thm}{}
\newtheorem{lem}{Lemma}
\theoremstyle{definition}
\newtheorem{defn}{Definition}

\section{Problems}

\noindent In problems 1-3, use the given information to set up the appropriate equation.
You need not carry out the computation.
On the last page you will find a selection of potentially useful formulae.
\begin{thm}[10 Points]\label{ex1}
  You deposit $\$1,000$ in an account that pays $6\%$ annual interest, compounded quarterly.
  \begin{enumerate}[(a)]
  \item
    What will be the balance of the account after 4 years?
    \vspace{1.5in}
  \item
    How much interest will have accrued?
  \end{enumerate}
\end{thm}

\newpage
\begin{thm}[18 Points]\label{ex2}
  Your local bank offers an account that pays $6\%$ per year with quarterly compounding interest.
  You wish to save $\$75,000$ over the course of 20 years.
  How much must you deposit into the account each quarter to reach your goal?
  \vspace{4in}
\end{thm}

\begin{thm}[18 Points]\label{ex3}
  You wish to set up an account that will pay $\$500$ per month over the course of 20 years.
  If the account earns $3\%$ per year, how much money must be deposited into the account?
\end{thm}

\newpage

\noindent Let $U = \{\text{Burton},\, \text{Ride},\,\text{Forum}, \text{V\"{o}lkl},\, \text{LibTech},\, \text{Gnu},\, \text{Rome}\}$.
  Let 
  $X = \{\text{Burton},\, \text{Ride},\,\text{Forum}\},$
  $Y = \{\text{Forum}, \text{V\"{o}lkl},\, \text{LibTech},\, \text{Gnu},\, \text{Rome}\}, \text{and}$
  $Z = \{\text{Ride},\,\text{Forum}, \text{V\"{o}lkl}\}.$
\begin{thm}[18 Points]\label{ex4}
  Compute
  \begin{enumerate}[(a)]
  \item
    $X \cap Y$,
    \vspace{1in}
  \item
    $X \cup Z$,
    \vspace{1in}
  \item
    The complement of $Z$ in $U$, $Z^c$.
    \vspace{1in}
  \end{enumerate}
\end{thm}

\begin{thm}[18 Points]
  \begin{enumerate}[(a)]
  \item
    What is the cardinality of $X \times Z$?
    \vspace{1in}
  \item
    What is the cardinality of $Y \cup Z$?
    \vspace{1in}
  \end{enumerate}
\end{thm}
\newpage

\begin{thm}[18 Points]\label{ex9}
  Use a truth table to prove the following logical equivalences.
  \begin{enumerate}[(a)]
  \item
    $$\neg\left(p \vee q\right) \equiv \neg p \wedge \neg q.$$
    \vspace{2in}
  \item
    $$\neg\left(p \wedge q \right) \equiv \neg p \vee \neg q.$$
    \vspace{2in}
  \end{enumerate}
\end{thm}

\begin{thm}[Bonus - 10 Points]\label{bonus}
  Write out Modus Tollens symbolically.
  State the conditions for an argument to be valid and then prove that Modus Tollens is a valid argument.
  Can you give an example of how Modus Tollens is used?
\end{thm}
\newpage
\section{Useful Formulae}

\begin{itemize}
  \newcommand{\INT}{\operatorname{INT}}
  \newcommand{\PV}{\operatorname{PV}}
  \newcommand{\FV}{\operatorname{FV}}
  \newcommand{\PMT}{\operatorname{PMT}}
\item
  $\INT = \PV r t$
\item
  $\FV = \PV(1 + rt)$
\item
  $\FV = \PV \left(1 + \frac{r}{m}\right)^{mt}$
\item
  $r_\text{eff} = \displaystyle{\left( 1 + \frac{r_\text{nom}}{m}\right)^m - 1}$
\item
  $\FV = \PMT \displaystyle{\frac{(1 + i)^n - 1}{i}},\, \text{where}\ i = \frac{r}{m}\ \text{and}\ n = mt$
\item
  $\PV = \PMT \displaystyle{\frac{1 - (1 + i)^{-n}}{i}},\, \text{where}\ i = \frac{r}{m}\ \text{and}\ n = mt$
\end{itemize}
\end{document}
