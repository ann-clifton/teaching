\documentclass[10pt]{amsart}
\usepackage{amsmath,amsthm,amssymb,amsfonts,enumerate,hyperref}
\openup 5pt
\author{Blake Farman\\University of South Carolina}
\title{Syllabus\\Math 115-011: Fall 2014}
\date{August 21, 2014}
\pdfpagewidth 8.5in
\pdfpageheight 11in
\usepackage[margin=1in]{geometry}

\begin{document}
\maketitle

\section*{Contact Information}
\noindent
\begin{tabular}{p{1.4in}p{5in}}
  {\bf E-mail:} &\href{mailto:farmanb@math.sc.edu}{farmanb@math.sc.edu}.\\
  {\bf Office:} & LeConte 400A.\\
  {\bf Office Hours:} & Tuesday/Thursday, 2:50 pm - 4:20 pm.\\
\end{tabular}

\section*{Course Information}
\noindent
\begin{tabular}{p{1.4in}p{5in}}
  {\bf Lectures:} &
  Monday/Wednesday,  3:55 pm - 5:10 pm in LeConte, Room 101.\\
  &Tuesday/Thursday, 4:25 pm - 5:15 pm in LeConte, Room 101.\\
  {\bf Learning Outcomes:} &Upon successful completion of this course, students should be able to:
	\begin{itemize}
	\item
	Recall basic mathematical terms related to linear, quadratic, exponential, logarithmic, inverse, polynomial, rational, circular, and trigonometric functions and express these terms in correct context;
	\item
	Apply the methods of algebra to solve applications involving intercepts, rates of change, inequalities, systems of equations, rational functions, and interest growth;
	\item
	Understand trigonometric functions – including basic relationships, domain, range, graphs, unit circle, identities including Pythagorean, reciprocal, even-odd identities, and formulas including sum/difference, half-angle, and double-angle formulas.
	\end{itemize}\\
  {\bf Pre-Requisites:} &Qualification through placement.\\
  {\bf Text:} & {\it Precalculus}, $2^{\text{nd}}$ Custom Edition for the University of South Carolina, Dugopolski, Pearson, 2014.  ISBN 9781269748155.\\
  {\bf Course Website:} & \url{http://people.math.sc.edu/farmanb/courses/115/F14}\\
  & Homework, test dates, and other various announcements made in class will also be posted here.\\
\end{tabular}

\section*{Coursework}
\noindent
\begin{tabular}{p{1.4in}p{5in}}
  {\bf Homework:} & Regular homework assignments will be assigned on MyMathLab.
  Late work will {\bf not} be accepted, except under extenuating circumstances (see Attendance).\\
  {\bf Exams:} & There will be four in class exams.
  The exact dates of the exams will be announced in class, but are tentatively scheduled as follows:\\
  & Exam 1: Thursday, September 11, 2014,\\
  & Exam 2: Thursday, October 2 , 2014.\\
  & Exam 3: Thursday, October 30, 2014\\
  & Exam 4: Tuesday, November 25, 2014\\
  {\bf Final Exam:} & There will be a cumulative final exam on Monday, December 8, 2014 at 4:00 pm.\\
\end{tabular}
\section*{Grading}
\begin{tabular}{p{1.4in}p{5in}}
  {\bf Scale:} & Grades will be assigned on the following scale:\\
  & \begin{tabular}{ll}
      A: &90-100\%,\\
      B: & 80-89\%,\\
      C: & 70-79\%,\\
      D: & 60-69\%,\\
      F: & $<$ 60\%.\\
    \end{tabular}\\
  {\bf Weights:} & Final grades will be calculated with the following weights:\\
  & \begin{tabular}{lr}
      Homework: & 20\%,\\
      Quizzes: &10\%,\\
      Exams: & 40\%,\\
      Final Exam: & 30\%.\\
    \end{tabular}\\
\end{tabular}
\section*{Expectations}
\noindent
\begin{tabular}{p{1.4in}p{5in}}
  {\bf Academic Integrity:} & Students are expected to act in accordance with the {\it University of South Carolina Honor Code}, 
  which can be found here: \url{http://www.housing.sc.edu/academicintegrity}.\\
  {\bf Attendance:} & See \url{http://bulletin.sc.edu/content.php?catoid=36\&navoid=3738}
\end{tabular}
\end{document}
