\documentclass[12pt]{amsart}
\usepackage{amsmath,amsthm,amssymb,amsfonts,enumerate}
\openup 5pt
\author{Blake Farman\\University of South Carolina}
\title{Math 115\\Exam 04}
\date{November 24, 2014}
\pdfpagewidth 8.5in
\pdfpageheight 11in
\usepackage[margin=1in]{geometry}

\renewcommand{\qedsymbol}{}

\begin{document}
\maketitle

\begin{center}
  \fbox{\fbox{\parbox{5.5in}{\centering
        Answer the questions in the spaces provided on the
        question sheets and turn them in at the end of the class period. 
        Unless otherwise stated, all supporting work is required.
        You may {\bf not} use any calculators.}}}
\end{center}

\vspace{0.2in}
\makebox[\textwidth]{Name:\enspace\hrulefill}
\vspace{0.2in}

$$
\begin{array}{|c|c|c|}
  \hline
  \text{Problem} & \text{Points Earned} & \text{Points Possible}\\
  \hline
  1 & & 20\\
  \hline
  2 & & 20\\
  \hline
  3 & & 20\\
  \hline
  4 & & 20\\
  \hline
  5 & & 20 \\
  \hline
  %\text{Bonus} & & 10\\
  %\hline
  \text{Total} & & 100\\
  \hline
\end{array}
$$

\newpage

\theoremstyle{plain}
\newtheorem{thm}{}
\newtheorem{lem}{Lemma}
\theoremstyle{definition}
\newtheorem{defn}{Definition}

\section{Problems}

\begin{thm}
  Find the period, frequency, phase shift, and amplitude of $$y = 4\sin(2x - \pi) - 1,$$ then graph one period.
\end{thm}

\newpage

\begin{thm}
  Find the period, frequency, phase shift, and amplitude of $$y = 3\cos\left(\frac{x}{2}\right) + 2,$$ then graph one period.
\end{thm}

\newpage

\begin{thm}
  Show that
  $$\tan(\arccos(x)) = \frac{\sqrt{1 - x^2}}{x}.$$
\end{thm}

\newpage
\begin{thm}
  Determine whether the following functions are even, odd, or neither.
  {\bf Justify your answers.}
  \begin{enumerate}[(a)]
  \item
    $f(x) = \sin(x^2)$
    \vspace{2in}
  \item
    $g(x) = 1 + \sec(x)$
    \vspace{2in}
  \item
    $h(x) = x\cos(x)$
    \vspace{2in}
  \item
    $k(x) = \sin(x) - 1$
  \end{enumerate}
  
\end{thm}

\newpage

\begin{thm}
  Given $\sin(\pi/4) = \cos(\pi/4) = \sqrt{2}/2$, $\cos(\pi/6) = \sqrt{3}/2$, and $\sin(\pi/6) = 1/2$, compute the values of each of the following expressions.
  You will find a list of potentially useful trigonometric identities on the following page.
  [Hint: $1/4$ and $1/6$ have common denominator $12$]
  \begin{enumerate}[(a)]
  \item
    $\displaystyle{\sin\left(\frac{5\pi}{12}\right)}$.
    \vspace{3in}
  \item
    $\displaystyle{\cos\left(\frac{5\pi}{12}\right)}$.
    \vspace{3in}
  \end{enumerate}
\end{thm}

\newpage

\section{Useful Formulae}

\begin{itemize}
\item
  $\sin^2(\theta) + \cos^2(\theta) = 1$.
\item
  $\tan^2(\theta) + 1 = \sec^2(\theta)$.
\item
  $\cot^2(\theta) + 1 = \csc^2(\theta)$.
\item
  $\cos(2\theta) = \cos^2(\theta) - \sin^2(\theta)$.
\item
  $\cos(2\theta) = 1 - 2\sin^2(\theta)$.
\item
  $\cos(2\theta) = 2\cos^2(\theta) - 1$.
\item
  $\sin(2\theta) = 2\sin(\theta)\cos(\theta)$.
\item
  $\sin(\theta + \varphi) = \sin(\theta)\cos(\varphi) + \sin(\varphi)\cos(\varphi)$.
\item
  $\cos(\theta + \varphi) = \cos(\theta)(\cos(\varphi) - \sin(\varphi)\cos(\varphi)$.
\item
  $\displaystyle{\sin\left(\frac{\theta}{2}\right) = \pm \sqrt{\frac{1 - \cos(\theta)}{2}}}$
\item
  $\displaystyle{\cos\left(\frac{\theta}{2}\right) = \pm \sqrt{\frac{1 + \cos(\theta)}{2}}}$
\item
  $2\sin(\theta)\cos(\varphi) = \sin(\theta - \varphi) + \sin(\theta + \varphi)$.
\item
  $2\sin(\theta)\sin(\varphi) = \cos(\theta - \varphi) - \cos(\theta + \varphi)$.
\item
  $2\cos(\theta)\cos(\varphi) = \cos(\theta - \varphi) + \cos(\theta + \varphi)$.
\end{itemize}
\end{document}
