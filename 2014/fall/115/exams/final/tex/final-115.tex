\documentclass[12pt]{amsart}
\usepackage{amsmath,amsthm,amssymb,amsfonts,enumerate}
\openup 5pt
\author{Blake Farman\\University of South Carolina}
\title{Math 115\\Final Exam}
\date{December 8, 2014}
\pdfpagewidth 8.5in
\pdfpageheight 11in
\usepackage[margin=1in]{geometry}

\renewcommand{\qedsymbol}{}

\begin{document}
\maketitle

\begin{center}
  \fbox{\fbox{\parbox{5.5in}{\centering
        Answer the questions in the spaces provided on the
        question sheets and turn them in at the end of the class period.
        If you require extra space, use the back of the page and indicate that you have done so.
        
        Unless otherwise stated, all supporting work is required.
        Unsupported or otherwise mysterious answers will {\bf not} receive credit.
        
        You may {\bf not} use any calculators or other electronic devices.}}}
\end{center}

\vspace{0.2in}
\makebox[\textwidth]{Name:\enspace\hrulefill}
\vspace{0.2in}

$$
\begin{array}{|c|c|c|}
  \hline
  \text{Problem} & \text{Points Earned} & \text{Points Possible}\\
  \hline
  1 & & 10\\
  \hline
  2 & & 10\\
  \hline
  3 & & 10\\
  \hline
  4 & & 10\\
  \hline
  5 & & 10 \\
  \hline
  6 & & 10\\
  \hline
  7 & & 10\\
  \hline
  8 & & 10\\
  \hline
  9 & & 10\\
  \hline
  10 & & 10\\
  \hline
  \text{Bonus} & & 5\\
  \hline
  \text{Total} & & 100\\
  \hline
\end{array}
$$

\newpage

\theoremstyle{plain}
\newtheorem{thm}{}
\newtheorem{lem}{Lemma}
\theoremstyle{definition}
\newtheorem{defn}{Definition}

\section{Problems}

\begin{thm}
  Let $f(x) =  -2x^2 + 8x - 5$.
  \begin{enumerate}[(a)]
  \item
    Find the solutions to $f(x) = 0$.
    \vspace{2in}
  \item
    Write $f(x)$ in vertex form.
    \vspace{2in}
  \item
    Find the coordinates of the $y$-intercept for $f(x)$.
    \vspace{2in}
  \item
    Use parts (a)-(c) to sketch a graph of $f(x)$.
  \end{enumerate}
\end{thm}

\newpage

\begin{thm}
    Find and simplify the difference quotient for $f(x) = 2x^2 + x$.
\end{thm}

\newpage

\begin{thm}
  Graph the function
  $$f(x) = \left\{
  \begin{array}{ccc}
    -x - 3 & \text{if} & x \leq -3,\\
    \sqrt{9 - x^2} & \text{if} & -3 < x < 3,\\
    x - 3 & \text{if} & 3 \leq x.
  \end{array}
  \right.
  $$
\end{thm}

\newpage

\begin{thm}
  Solve the inequality
  $$-4x^2 + 3x + 1 \geq 0$$
  for $x$.
\end{thm}

\newpage

\begin{thm}
  Solve the equation
  $$\ln(x - 2) = \ln\left(\frac{1}{x - 4}\right)$$
  for $x$.
\end{thm}
\newpage

\begin{thm}
  Find the period, frequency, phase shift, and amplitude of $$y = 2\sin(3x + \pi) + 1,$$ then graph one period.
\end{thm}

\newpage

\begin{thm}
  Find the period, frequency, phase shift, and amplitude of $$y = -3\cos\left(x - \frac{\pi}{2}\right) - 3,$$ then graph one period.
\end{thm}

\newpage

\begin{thm}
  Show that
  $$\sin(\arctan(x)) = \frac{x}{1 + x^2}.$$
\end{thm}

\newpage
\begin{thm}
  Determine whether the following functions are even, odd, or neither.
  {\bf Justify your answers.}
  \begin{enumerate}[(a)]
  \item
    $f(x) = \sin(x^3)$
    \vspace{2in}
  \item
    $\displaystyle{h(x) = \cos\left(x^3\right)}$
    \vspace{2in}
  \item
    $g(x) = 1 + \csc(x)$
    \vspace{2in}
  \item
    $k(x) = \cos(x) - 1$
  \end{enumerate}
  
\end{thm}

\newpage

\begin{thm}
  You are given that $\cos(\pi/3) = 1/2$, $\sin(\pi/3) = \sqrt{3}/2$, $\cos(\pi/2) = 0$, and $\sin(\pi/2) = 1$.
  Using the list of trigonometric identities on the last page, compute the values of each of the following expressions.
  {\bf Do not} simply write down values you have memorized from the unit circle as you will receive no credit for doing so.
  [Hint: $1/3$ and $1/2$ have common denominator $6$]
  \begin{enumerate}[(a)]
  \item
    $\displaystyle{\sin\left(\frac{5\pi}{6}\right)}$.
    \vspace{3in}
  \item
    $\displaystyle{\cos\left(\frac{5\pi}{6}\right)}$.
    \vspace{3in}
  \end{enumerate}
\end{thm}

\newpage

\begin{thm}[Bonus]
  Solve the following system algebraically and then sketch both equations on the same coordinate system.
  Label the points of intersection.
  \begin{eqnarray*}
    y &=& x^2 - 7x + 18\\
    y &=& x + 3
  \end{eqnarray*}
\end{thm}

\newpage

\section{Useful Formulae}

\begin{itemize}
\item
  $\sin^2(\theta) + \cos^2(\theta) = 1$.
\item
  $\tan^2(\theta) + 1 = \sec^2(\theta)$.
\item
  $\cot^2(\theta) + 1 = \csc^2(\theta)$.
\item
  $\cos(2\theta) = \cos^2(\theta) - \sin^2(\theta)$.
\item
  $\cos(2\theta) = 1 - 2\sin^2(\theta)$.
\item
  $\cos(2\theta) = 2\cos^2(\theta) - 1$.
\item
  $\sin(2\theta) = 2\sin(\theta)\cos(\theta)$.
\item
  $\sin(\theta + \varphi) = \sin(\theta)\cos(\varphi) + \sin(\varphi)\cos(\theta)$.
\item
  $\cos(\theta + \varphi) = \cos(\theta)\cos(\varphi) - \sin(\theta)\sin(\varphi)$.
\item
  $\displaystyle{\sin\left(\frac{\theta}{2}\right) = \pm \sqrt{\frac{1 - \cos(\theta)}{2}}}$
\item
  $\displaystyle{\cos\left(\frac{\theta}{2}\right) = \pm \sqrt{\frac{1 + \cos(\theta)}{2}}}$
\item
  $2\sin(\theta)\cos(\varphi) = \sin(\theta - \varphi) + \sin(\theta + \varphi)$.
\item
  $2\sin(\theta)\sin(\varphi) = \cos(\theta - \varphi) - \cos(\theta + \varphi)$.
\item
  $2\cos(\theta)\cos(\varphi) = \cos(\theta - \varphi) + \cos(\theta + \varphi)$.
\end{itemize}
\end{document}
