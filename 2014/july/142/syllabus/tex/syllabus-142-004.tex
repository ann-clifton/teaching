\documentclass[10pt]{amsart}
\usepackage{amsmath,amsthm,amssymb,amsfonts,enumerate,mymath,hyperref}
\openup 5pt
\author{Blake Farman\\University of South Carolina}
\title{Syllabus\\Math 142-004: July 2014}
\date{July 7, 2014}
\pdfpagewidth 8.5in
\pdfpageheight 11in
\usepackage[margin=1in]{geometry}

\begin{document}
\maketitle

\section*{Contact Information}
\noindent
\begin{tabular}{p{1.4in}p{5in}}
  {\bf E-mail:} &\href{mailto:farmanb@math.sc.edu}{farmanb@math.sc.edu}.\\
  {\bf Office:} & LeConte 400A.\\
  {\bf Office Hours:} & Monday/Wednesday/Friday, 1:00 pm - 2:00 pm.\\
\end{tabular}

\section*{Course Information}
\noindent
\begin{tabular}{p{1.4in}p{5in}}
  {\bf Lectures:} &
  Monday - Friday,  10:05 am - 12:05 pm in LeConte, Room 121.\\
  {\bf Recitations:} &
  Monday/Wednesday/Friday, 2:50pm - 4:50pm in LeConte, Room 121.\\
  {\bf Lab:} &
  Tuesday/Thursday, 2:50 pm - 4:50 pm in LeConte, Room 102.\\
  {\bf Learning Outcomes:} & A student who successfully completes Calculus II (MATH 142) should continue to develop as an independent learner 
  with the ability to approach problems from a conceptual viewpoint, to utilize more than one idea in a single problem, and to apply appropriate 
  calculus skills to problems in context.
  
  In particular, the successful student will master concepts and gain skills needed to solve problems related to 
  techniques of integration, 
  improper integrals, 
  applications of integration, 
  convergence of sequences and series,
  power series,
  Taylor and Maclaurin series,
  applications of Taylor polynomials,
  polar coordinates,
  area and length in polar coordinates.\\
  {\bf Pre-Requisites:} & Qualification through placement or a grade of C or better in MATH 141.\\
  {\bf Text:} & {\it Calculus, Early Transcendentals,} Sixth Edition, James Stewart, Cengage Learning.  ISBN 9781111077952.\\
  {\bf Course Website:} & \url{http://www.math.sc.edu/~farmanb/courses/142/J14}\\
  & Test dates and other various announcements made in class will also be posted here.\\
\end{tabular}

\section*{Coursework}
\noindent
\begin{tabular}{p{1.4in}p{5in}}
  {\bf Homework:} & Regular homework assignments will be posted on WebAssign.
  Homework is expected to be completed on time.
  Late work will {\bf not} be accepted, except under extenuating circumstances (see Attendance).\\
  {\bf Exams:} & There will be three in class exams.
  The exact dates of the exams will be announced in class, but are tentatively scheduled for the end of each week.\\
  & Exam 1: Friday, July 11, 2014,\\
  & Exam 2: Friday, July 18, 2014.\\
  & Exam 3: Friday, July 25, 2014.\\
  {\bf Final Exam:} & There will be a 2.5 hour cumulative final exam on Friday, August 1, 2014 at 2:50 pm.\\
\end{tabular}

\section*{Grading}
\noindent
\begin{tabular}{p{1.4in}p{5in}}
  {\bf Scale:} & Grades will be assigned on the following scale:\\
  & \begin{tabular}{ll}
      A: &90-100\%,\\
      B: & 80-89\%,\\
      C: & 70-79\%,\\
      D: & 60-69\%,\\
      F: & $<$ 60\%.\\
    \end{tabular}\\
  {\bf Weights:} & Final grades will be calculated with the following weights:\\
  & \begin{tabular}{lr}
      Homework: & 30\%,\\
      Exams: & 30\%,\\
      Final Exam: & 30\%.\\
      Lab: & 10\%.\\
    \end{tabular}\\
\end{tabular}
\section*{Expectations}
\noindent
\begin{tabular}{p{1.4in}p{5in}}
  {\bf Academic Integrity:} & Students are expected to act in accordance with the {\it University of South Carolina Honor Code}, 
  which can be found here: \url{http://www.housing.sc.edu/academicintegrity}.\\
  {\bf Attendance:} & See \url{http://bulletin.sc.edu/content.php?catoid=36&navoid=3738}
\end{tabular}
\end{document}
