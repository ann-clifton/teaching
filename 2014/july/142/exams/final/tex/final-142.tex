\documentclass[12pt]{amsart}
\usepackage{amsmath,amsthm,amssymb,amsfonts,enumerate,mymath,tikz-cd,fancyhdr}
\openup 5pt
\author{Blake Farman\\University of South Carolina}
\title{Math 142: Final Exam}
\date{August 1, 2014}
\pdfpagewidth 8.5in
\pdfpageheight 11in
\usepackage[margin=1in]{geometry}

\renewcommand{\qedsymbol}{}

\begin{document}
\maketitle

\begin{center}
  \fbox{\fbox{\parbox{5.5in}{\centering
        Answer the questions in the spaces provided on the
        question sheets and turn them in at the end of the class period. 
        Unless otherwise stated, all supporting work is required.
        It is advised, although not required, that you check your answers.
        You may {\bf not} use any calculators.}}}
\end{center}

\vspace{0.2in}
\makebox[\textwidth]{Name:\enspace\hrulefill}
\vspace{0.2in}

\theoremstyle{plain}
\newtheorem{thm}{}
\newtheorem{lem}{Lemma}
\theoremstyle{definition}
\newtheorem{defn}{Definition}

$$
\begin{array}{|c|c|c|}
  \hline
  \text{Problem} & \text{Points Earned} & \text{Points Possible}\\
  \hline
  1 & & 10\\
  \hline
  2 & & 10\\
  \hline
  3 & & 10\\
  \hline
  4 & & 10\\
  \hline
  5 & & 10 \\
  \hline
  6 & & 10 \\
  \hline
  7 & & 10 \\
  \hline
  8 & & 10 \\
  \hline
  9 & & 10 \\
  \hline
  10 & & 10 \\
  \hline
  \text{Bonus} & & 10\\
  \hline
  \text{Total} & & 100\\
  \hline
\end{array}
$$

\newpage

\section{Problems}

\subsection{Applications of Integration}

%\begin{thm}[10 Points]
%Find the area between the curves $\displaystyle{y = \sin\left(\frac{\pi x}{2}\right)}$ and $\displaystyle{y = x}$.
%\end{thm}

%\newpage

\begin{thm}[10 Points]
  Find the volume of the solid obtained by rotating the region bounded by the curves $f(x) = 4(x - 2)^2$ and $g(x) = x^2 - 4x + 7$ about the $y$-axis.
\end{thm}

\newpage

\subsection{Integration}

For each of the following problems, decide which method of integration is appropriate and compute the given integrals.
You will find some useful trigonmetric identities on the last page.
If you need more space for a problem, use the back of the page.

%\begin{thm}[10 Points]
%  Compute the following integrals.
%  \begin{enumerate}[(a)]
%  \item\label{1.a}
%    $\displaystyle{\int \theta\cos(\theta^2)d\theta}$.
%    \vspace{3in}
%  \item
%    $\displaystyle{\int \theta^3\cos(\theta^2)d\theta}$.
%  \end{enumerate}
%\end{thm}

%\newpage

\begin{thm}[10 Points]
  Compute $\displaystyle{\int\sin^3(\theta)\cos^2(\theta)d\theta}$.
\end{thm}

\newpage

\begin{thm}[10 Points]
  Compute $\displaystyle{\int \frac{\ln(x)}{x^2}dx}$.
\end{thm}

\newpage

\begin{thm}[10 Points]
  Compute $\displaystyle{\int\frac{dx}{x^2\sqrt{x^2-9}}}$.
\end{thm}

\newpage

\begin{thm}[10 Points]
  Compute $\displaystyle{\int\frac{x}{x^2 - 5x + 6}}dx$.
\end{thm}

\newpage
\begin{thm}[10 Points]
  Explain why $\displaystyle{\int_0^3 \frac{dx}{\sqrt{x}}}$ is an improper integral, then evaluate it.
\end{thm}

\newpage

\subsection{Series and Sequences}
\begin{thm}[10 Points]
  Express $5.\overline{5} = 5.55555\ldots$ as a rational number. [Hint: Geometric Series.]
\end{thm}

\newpage
Test the following series for convergence.
You may use any of the tests we covered in class, however you {\bf must indicate which test you use}.
\begin{thm}[10 Points]
  $\displaystyle{\sum_{k=1}^\infty \frac{2^k k!}{(k+2)!}}.$
\end{thm}

\newpage

\begin{thm}[10 Points]
  $\displaystyle{\sum_{n=1}^\infty \frac{2^{2n}}{n^n}}$
\end{thm}

\newpage

\subsection{Power Series}

\begin{thm}[10 Points]
  Find the Taylor expansion of $\ln(x)$ about $a = 2$ and find the radius of convergence and the interval of convergence for the series you find.
\end{thm}

\newpage

\subsection{Bonus}

\begin{thm}[10 Points]
  Sketch the curve with polar equation $\displaystyle{r = \sin^2(2\theta)}$.
\end{thm}
\newpage

\section{Useful Formulae}

\begin{itemize}
\item
  $\sin^2(\theta) + \cos^2(\theta) = 1$.
\item
  $\tan^2(\theta) + 1 = \sec^2(\theta)$.
\item
  $\cot^2(\theta) + 1 = \csc^2(\theta)$.
\item
  $2\sin^2(\theta) = 1 - \cos(2\theta)$.
\item
  $2\cos^2(\theta) = 1 + \cos(2\theta)$.
\item
  $\sin(2\theta) = 2\sin(\theta)\cos(\theta)$.
\item
  $2\sin(\theta)\cos(\varphi) = \sin(\theta - \varphi) + \sin(\theta + \varphi)$.
\item
  $2\sin(\theta)\sin(\varphi) = \cos(\theta - \varphi) - \cos(\theta + \varphi)$.
\item
  $2\cos(\theta)\cos(\varphi) = \cos(\theta - \varphi) + \cos(\theta + \varphi)$.
\end{itemize}

\end{document}
