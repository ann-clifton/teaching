\documentclass[12pt]{amsart}
\usepackage{amsmath,amsthm,amssymb,amsfonts,enumerate,mymath,tikz-cd,fancyhdr}
\openup 5pt
\author{Blake Farman\\University of South Carolina}
\title{Math 142: Exam 01}
\date{July 11, 2014}
\pdfpagewidth 8.5in
\pdfpageheight 11in
\usepackage[margin=1in]{geometry}

\renewcommand{\qedsymbol}{}

\begin{document}
\maketitle

\begin{center}
  \fbox{\fbox{\parbox{5.5in}{\centering
        Answer the questions in the spaces provided on the
        question sheets and turn them in at the end of the class period. 
        Unless otherwise stated, all supporting work is required.
        It is advised, although not required, that you check your answers.
        You may {\bf not} use any calculators.}}}
\end{center}

\vspace{0.2in}
\makebox[\textwidth]{Name:\enspace\hrulefill}
\vspace{0.2in}

\theoremstyle{plain}
\newtheorem{thm}{}
\newtheorem{lem}{Lemma}
\theoremstyle{definition}
\newtheorem{defn}{Definition}

$$
\begin{array}{|c|c|c|}
  \hline
  \text{Problem} & \text{Points Earned} & \text{Points Possible}\\
  \hline
  1 & & 20\\
  \hline
  2 & & 20\\
  \hline
  3 & & 20\\
  \hline
  4 & & 20\\
  \hline
  5 & & 20 \\
  \hline
  \text{Bonus} & & 10\\
  \hline
  \text{Total} & & 100\\
  \hline
\end{array}
$$

\newpage

\section{Problems}

For each of the following problems, decide which method of integration is appropriate and compute the given integrals.
You will find some useful trigonmetric identities on the last page.
If you need more space for a problem, use the back of the page.

\begin{thm}[20 Points]
  Compute the following integrals.
  \begin{enumerate}[(a)]
  \item\label{1.a}
    $\displaystyle{\int \theta\cos(\theta^2)d\theta}$.
    \vspace{3in}
  \item
    $\displaystyle{\int \theta^3\cos(\theta^2)d\theta}$.
  \end{enumerate}
\end{thm}

\newpage

\begin{thm}[20 Points]
  Compute $\displaystyle{\int\frac{dx}{\sqrt{x^2 + 16}}}$.
\end{thm}

\newpage

\begin{thm}[20 Points]
  Compute $\displaystyle{\int\cos^2(\theta)\tan^3(\theta)d\theta}$.
\end{thm}

\newpage

\begin{thm}[20 Points]
  Compute $\displaystyle{\int 2x\tan(x^2)dx}$.
  %Hint: $\displaystyle{\tan(x) = \frac{\tan(x)\sec(x)}{\sec(x)}}$
\end{thm}

\newpage

\begin{thm}[20 Points]
  Compute $\displaystyle{\int\frac{x}{x^2 + x - 2}}dx$.
\end{thm}

\newpage

\begin{thm}[Bonus - 10 points]
  Compute $\displaystyle{\int \sqrt{\frac{1-x}{1+x}}}dx$.
\end{thm}

\newpage
\section{Useful Formulae}

\begin{itemize}
\item
  $\sin^2(\theta) + \cos^2(\theta) = 1$.
\item
  $\tan^2(\theta) + 1 = \sec^2(\theta)$.
\item
  $\cot^2(\theta) + 1 = \csc^2(\theta)$.
\item
  $2\sin^2(\theta) = 1 - \cos(2\theta)$.
\item
  $2\cos^2(\theta) = 1 + \cos(2\theta)$.
\item
  $\sin(2\theta) = 2\sin(\theta)\cos(\theta)$.
\item
  $2\sin(\theta)\cos(\varphi) = \sin(\theta - \varphi) + \sin(\theta + \varphi)$.
\item
  $2\sin(\theta)\sin(\varphi) = \cos(\theta - \varphi) - \cos(\theta + \varphi)$.
\item
  $2\cos(\theta)\cos(\varphi) = \cos(\theta - \varphi) + \cos(\theta + \varphi)$.
\end{itemize}

\end{document}
