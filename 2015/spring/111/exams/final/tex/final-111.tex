\documentclass[12pt]{amsart}
\usepackage{amsmath,amsthm,amssymb,amsfonts,enumerate,mymath,tikz-cd,fancyhdr}
\openup 5pt
\author{Blake Farman\\University of South Carolina}
\title{Math 111\\Final Exam}
\date{May 4, 2015}
\pdfpagewidth 8.5in
\pdfpageheight 11in
\usepackage[margin=1in]{geometry}

\renewcommand{\qedsymbol}{}

\begin{document}
\maketitle

\begin{center}
  \fbox{\fbox{\parbox{5.5in}{\centering
        Answer the questions in the spaces provided on the
        question sheets and turn them in at the end of the exam period.
        If you require extra space, use the back of the page and indicate that you have done so.
        
        Unless otherwise stated, all supporting work is required.
        Unsupported or otherwise mysterious answers will {\bf not receive credit.}
        
        You may use a calculator, but you may {\bf not} use a Computer Algebra System (CAS)
        or any other electronic device whatsoever, {\bf including cell phones.}}}}
\end{center}

\vspace{0.2in}
\makebox[\textwidth]{Name:\enspace\hrulefill}
\vspace{0.2in}

$$
\begin{array}{|c|c|c|c|c|c|}
  \hline
  \text{Problem} & \text{Points Earned} & \text{Points Possible} \\
  \hline
  1 & & 3 \\
  \hline
  2 & & 6 \\
  \hline
  3 & & 1 \\
  \hline
  4 & & 2 \\
  \hline
  5 & & 3 \\
  \hline
  6 & & 2 \\
  \hline
  7 & & 1 \\
  \hline
  8 & & 4 \\
  \hline
  9 & & 3 \\
  \hline
  10 & & 5\\
  \hline
  11 & & 5\\
  \hline
  12 & & 9\\
  \hline
  13 & & 9\\
  \hline
  14 & & 9\\
  \hline
  15 & & 20\\
  \hline
  16 & & 10\\
  \hline
  17 & & 8\\
  \hline
  \text{Total} & & 100\\
  \hline
\end{array}
$$

\theoremstyle{plain}
\newtheorem{thm}{}
\newtheorem{lem}{Lemma}
\theoremstyle{definition}
\newtheorem{defn}{Definition}

\newpage

\section{Definitions}
%1
\begin{thm}[3 Points]
  Fill in the blanks with the correct factorizations.
  \begin{enumerate}[(a)]
  \item
    $\displaystyle{A^2 - B^2 =\ \line(1,0){100}}.$
    \vspace{.3in}
  \item
    $\displaystyle{A^2 + 2AB + B^2 =\ \line(1,0){100}}.$
    \vspace{.3in}
  \item
    $\displaystyle{A^2 - 2AB + B^2 =\ \line(1,0){100}}.$
    \vspace{.3in}
  \end{enumerate}
\end{thm}

\vspace{1in}
%2
\begin{thm}[6 Points]
  Let $a, b$ be non-zero real numbers and $m, n$ integers.
  Fill in the blanks
  \vspace{.25in}
  \begin{enumerate}[(i)]
  \item
    $\displaystyle{a^0 =\ \line(1,0){100}}$,
    \vspace{.25in}
  \item
    $\displaystyle{a^{-n} =\ \line(1,0){100}}.$
    \vspace{.25in}
  \item
    $\displaystyle{a^m \cdot a^n =\ \line(1,0){100}}$
    \vspace{.25in}
  \item
    $\displaystyle{\frac{a^m}{a^n} = \ \line(1,0){100}}$
    \vspace{.25in}
  \item
    $\displaystyle{\left(a \cdot b\right)^n = \ \line(1,0){100}}$
    \vspace{.25in}
  \item
    $\displaystyle{\left(\frac{a}{b}\right)^n = \ \line(1,0){100}}$
  \end{enumerate}
\end{thm}

\newpage
%3
\begin{thm}[1 Point]
  Given an equation $ax^2 + bx + c = 0$, the solutions are given by the Quadratic Formula.  State the Quadratic Formula.
  \vspace{2in}
\end{thm}

%4
\begin{thm}[2 Points]
  \begin{enumerate}[(a)]
  \item
    State the Point-Slope form of a line passing through the point $(x_1, y_1)$ with slope $m$.
    \vspace{2in}
  \item
    State the Slope-Intercept form of a line with slope $m$ and $y$-intercept $b$.
    \vspace{2in}
  \end{enumerate}
\end{thm}

\newpage
%5
%\begin{thm}[1 Point]
%  If $f(x)$ is an exponential function with growth/decay factor $a$, express t%he growth/decay rate, $r$, in terms of the growth/decay factor.
%  \vspace{1in}
%\end{thm}

%6
\begin{thm}[3 Points]
  \begin{enumerate}[(a)]
  \item
    State the general form of an exponential function.
    \vspace{1in}
    \item
      When does such a function model exponential growth?
      \vspace{1in}
    \item
      When does such a function model exponential decay?
      \vspace{1in}
  \end{enumerate}
\end{thm}

%7
\begin{thm}[2 Points]
  Consider the two distinct lines $f(x) = m_1x + b_1$ and $g(x) = m_2x + b_2$.
  \begin{enumerate}[(a)]
  \item
    When are $f$ and $g$ parallel?
    \vspace{1in}
  \item
    When are $f$ and $g$ perpendicular?
    \vspace{1in}
  \end{enumerate}
\end{thm}

\newpage

%8
\begin{thm}[1 Point]
  Let $a$ be a fixed positive number.
  The base $a$ logarithm of $x$ is defined by
  $$\log_a(x) = y\  \text{if and only if}\ \ \line(1,0){80}.$$
\end{thm}

\vspace{1in}

%9
\begin{thm}[4 Points]
  Let $a$ be a positive number.  Fill in the blanks.
  \begin{enumerate}[(a)]
    \vspace{.25in}
  \item
    $\log_a(1) = \ \line(1,0){100}$.
    \vspace{.25in}
  \item
    $\log_a(a) = \ \line(1,0){100}$.
    \vspace{.25in}
  \item
    $\log_a(a^x) = \ \line(1,0){100}$.
    \vspace{.25in}
  \item
    $a^{\log_a(x)} = \ \line(1,0){100}$.
  \end{enumerate}
  \vspace{.5in}
\end{thm}

%10
\begin{thm}[3 Points]
  Let $0 < a$ and $C$ be fixed numbers.  Fill in the blanks.
  \vspace{.25in}
  \begin{enumerate}[(a)]
  \item
    $\log_a(xy) = \ \line(1,0){100}$.
    \vspace{.25in}
  \item
    $\log_a\left(\frac{x}{y}\right) = \ \line(1,0){100}$.
    \vspace{.25in}
  \item
    $\log_a(x^C) = \ \line(1,0){100}$.
  \end{enumerate}
  \vspace{.5in}
\end{thm}

\newpage

\section{Problems}

%11
\begin{thm}[4 Points]
  Add the following rational expressions and simplify the result,
  $$\frac{1}{x + \sqrt{3}} + \frac{1}{x - \sqrt{3}}.$$
  \vspace{3in}
\end{thm}

%12
\begin{thm}[5 Points]
  Consider the two lines $f(x) = 5x + 16$ and $g(x) = 8x + 7$.
  Find the point (that is, the $(x,y)$ pair) where these two lines intersect.
  \vspace{2in}
\end{thm}

\newpage

%13
\begin{thm}[9 Points]
  Let $f(x) = 3x^2 - 18x + 24.$
  \begin{enumerate}[(a)]
  \item
    Put $f(x)$ into standard form.
    \vspace{2in}
  \item
    Solve $f(x) = 0$.
    \vspace{2in}
  \item
    Use the information from parts (a) and (b) to sketch a graph of $f(x)$.
    To receive credit, you must label the $y$-intercept, any $x$-intercept(s), and the vertex.
    \vspace{2in}
  \end{enumerate}
\end{thm}

\newpage

%14
\begin{thm}[9 Points]
  In the following problems, use the given information to find the equation of the line in slope-intercept form.
  \begin{enumerate}[(a)]
  \item
    The line passing through the points $(4,20)$ and $(1,14)$.
    \vspace{2in}
  \item
    The line passing through the point $(6, 12)$ and parallel to the line in part (a).
    \vspace{2in}
  \item
    The line passing through $(4, 8)$ and perpendicular to the line in part (a).
    \vspace{1in}
  \end{enumerate}
\end{thm}

\newpage

%15
\begin{thm}[9 Points]
  A biologist observes a population with initial size $81$.
  In two years, the biologist returns to observe the population again and finds that only $9$ remain.
  \begin{enumerate}[(a)]
  \item
    Find an exponential model for the size of the population as a function of $t$ years.
    \vspace{1in}
  \item
    Does the function from part (a) model growth or decay?
    \vspace{1.5in}
  \item
    Use the model from part (a) to determine how many years it will take for the size of the population to reach 1.
    \vspace{1.5in}
  \end{enumerate}
\end{thm}

\newpage

%16
\begin{thm}[20 Points]
  \begin{enumerate}[(a)]
  \item
    Simplify the expression 
    $$\log_4(x + 3) + \log_4(x - 3).$$
    \vspace{2in}
  \item
    Solve the following equation for $x$
    $$\log_4(x + 3) + \log_4(x - 3) = 2$$
    \vspace{1in}
  \end{enumerate}
\end{thm}

%17
\begin{thm}[10 Points]
  Solve the following equation for $x$
  $$e^{x^2} = e^{-2x - 1}$$
\end{thm}

\newpage

%18
\begin{thm}[8 Points]
  Let $f(x) = \sqrt{x}$ and $g(x) = x^2 - 9$.
  \begin{enumerate}[(a)]
  \item
    Compute the composition $(f \circ g)(x)$.
    What is the domain of this function?
    \vspace{3.5in}
  \item
    Compute the composition $(g \circ f)(x)$.
    What is the domain of this function?
  \end{enumerate}
\end{thm}
\end{document}
