\documentclass[10pt]{amsart}
\usepackage{amsmath,amsthm,amssymb,amsfonts,enumerate,hyperref}
\openup 5pt
\author{Blake Farman\\University of South Carolina}
\title{Syllabus\\Math 111-006\\Spring 2015}
\date{January 12, 2015}
\pdfpagewidth 8.5in
\pdfpageheight 11in
\usepackage[margin=1in]{geometry}

\begin{document}
\maketitle

\section*{Contact Information}
\noindent
\begin{tabular}{p{1.4in}p{5in}}
  {\bf E-mail:} &\href{mailto:farmanb@math.sc.edu}{farmanb@math.sc.edu}.\\
  {\bf Office:} & LeConte 317N.\\
  {\bf Office Hours:} & Monday/Wednesday, 12:30 pm - 2:00 pm.\\
\end{tabular}

\section*{Course Information}
\noindent
\begin{tabular}{p{1.4in}p{5in}}
  {\bf Lectures:} &
  Monday/Wednesday/Friday,  10:50 am - 11:40 am in Gambrell, Room 124.\\
  {\bf SI:} &  Supplemental Instruction is available for this course to assist you in better understanding the course material. The SI Program provides peer-facilitated study sessions led by qualified and trained undergraduate SI Leaders who attend classes with students and encourage students to practice and discuss course concepts in sessions. Sessions are open to all students who want to improve their understanding of the material, as well as their grades. SI Sessions will focus on the most recent material covered in class. Each SI Leader holds three sessions per week. Your SI Leader is Ryan Dyckes and you can find her session schedule online at \url{http://www.sa.sc.edu/ssc/supplementalinstruction/si-schedule}. You can also contact the Student Success Center at (803) 777-1000 if you have questions about the SI Session schedule.\\
  {\bf Learning Outcomes:} &Upon successful completion of this course, students should be able to:
  \begin{itemize}
  \item
    Recall basic mathematical terms related to linear, quadratic, exponential, and logarithmic functions and express these terms in correct context;
  \item
    Apply the methods of algebra to solve applications involving intercepts, rates of change, inequalities, systems of equations, and interest growth;
  \item
    Verbally interpret relationships in data given as graphs, tables, and equations and express functions given in verbal context as a graph, table, or equation.
  \end{itemize}\\
               {\bf Pre-Requisites:} &Qualification through placement.\\
               {\bf Text:} & {\it College Algebra: Concepts and Contexts}, $1^{\text{st}}$ Edition, Stewart, Redlin, Watson, Panman, 2011.  ISBN 9781424089208.\\
               {\bf Course Website:} & \url{http://people.math.sc.edu/farmanb/courses/111/S15/index.html}\\
               & Test dates, solutions, and other various announcements made in class will also be posted here.\\
\end{tabular}

\section*{Coursework}
\noindent
\begin{tabular}{p{1.4in}p{5in}}
  {\bf Homework:} & Regular homework will be assigned on WebAssign.
  The class key is {\bf sc 5570 4038}.
  
  Late work will {\bf not} be accepted, except under extenuating circumstances (see Attendance).\\
  {\bf Exams:} & There will be three in-class exams.
  The exact dates of the exams will be announced during lecture, but are tentatively scheduled as follows:\\
  & Exam 1: Wednesday, February 11, 2015,\\
  & Exam 2: Wednesday, March 25, 2015.\\
  & Exam 3: Wednesday, April 22, 2015\\
  & There will {\bf not} be any make-up exams or quizzes.
  If you miss one exam, your final exam grade will replace the missing exam grade.
  Any further missed exams will receive a grade of zero.\\
  {\bf Final Exam:} & There will be a cumulative final exam on Monday, May 4, 2015 at 9:00 am.
  This date can {\bf not} be changed and you may {\bf not} take the exam at another time.
  If you do not attend the final exam, you will receive a grade of zero on the exam.\\
\end{tabular}
\section*{Grading}
\begin{tabular}{p{1.4in}p{5in}}
  {\bf Scale:} & Grades will be assigned on the following scale:\\
  & \begin{tabular}{ll}
      A: &90-100\%,\\
      B: & 80-89\%,\\
      C: & 70-79\%,\\
      D: & 60-69\%,\\
      F: & $<$ 60\%.\\
    \end{tabular}\\
  {\bf Weights:} & Final grades will be calculated with the following weights:\\
  & \begin{tabular}{lr}
      Homework: & 30\%,\\
      Quizzes: &10\%,\\
      Exams: & 30\%,\\
      Final Exam: & 30\%.\\
    \end{tabular}\\
\end{tabular}
\section*{Expectations}
\noindent
\begin{tabular}{p{1.4in}p{5in}}
  {\bf Academic Integrity:} & Students are expected to act in accordance with the {\it University of South Carolina Honor Code}, 
  which can be found here: \url{http://www.housing.sc.edu/academicintegrity}.
  
  {\bf Any breach of the Honor Code will result in an F for the course.}\\
  {\bf Attendance:} & Students are obligated to complete all assigned work promptly, to attend class regularly, and to participate in whatever class discussion may occur.

  The following events or circumstances are potentially excusable absences:
  \begin{itemize}
  \item
    participation in an authorized University activity (such as musical performances, academic competitions, or varsity athletic events in which the student plays a formal role in a University sanctioned event),
  \item
    required participation in military duties,
  \item
    mandatory admission interviews for professional or graduate school which cannot be rescheduled,
  \item
    participation in legal proceedings or administrative duties that require a student's presence,
  \item
    death or major illness in a student’s immediate family,
  \item
    illness of a dependent family member
  \item
    religious holy day if listed on \url{www.interfaithcalendar.org},
  \item
    illness that is too severe or contagious for the student to attend class,
  \item
    weather-related emergencies.
  \end{itemize}
  For more information, see the University Attendance Policy: \url{http://bulletin.sc.edu/content.php?catoid=36\&navoid=3738}.
\end{tabular}
\end{document}
