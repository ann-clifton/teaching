\documentclass[12pt]{amsart}
\usepackage{amsmath,amsthm,amssymb,amsfonts,enumerate,mymath,tikz-cd,fancyhdr}
\openup 5pt
\author{Blake Farman\\University of South Carolina}
\title{Math 116\\Final Exam}
\date{December 8, 2015}
\pdfpagewidth 8.5in
\pdfpageheight 11in
\usepackage[margin=1in]{geometry}

\renewcommand{\qedsymbol}{}

\begin{document}
\maketitle

\begin{center}
  \fbox{\fbox{\parbox{5.5in}{\centering
        Answer the questions in the spaces provided on the
        question sheets and turn them in at the end of the class period.
        If you require extra space, use the back of the page and indicate that you have done so.
        
        Unless otherwise stated, all supporting work is required.
        Unsupported or otherwise mysterious answers will {\bf not receive credit.}
        You may {\it not} use any calculators.}}}
\end{center}

\vspace{0.2in}
\makebox[\textwidth]{Name:\enspace\hrulefill}
\vspace{0.2in}

$$
\begin{array}{|c|c|c|}
  \hline
  \text{Problem} & \text{Points Earned} & \text{Points Possible}\\
  \hline
  1 & & 10\\
  \hline
  2 & & 10\\
  \hline
  3 & & 10\\
  \hline
  4 & & 10\\
  \hline
  5 & & 10 \\
  \hline
  6 & & 10 \\
  \hline
  7 & & 10 \\
  \hline
  8 & & 10 \\
  \hline
  9 & & 10 \\
  \hline
  10 & & 10 \\
  \hline
  \text{Total} & & 100\\
  \hline
\end{array}
$$

\newpage


\theoremstyle{plain}
\newtheorem{thm}{}
\newtheorem{lem}{Lemma}
\theoremstyle{definition}
\newtheorem{defn}{Definition}

\begin{thm}[10 Points]\label{ex1}
  Find the period, frequency, and amplitude of $y = 2\cos(2x) + 2$, then graph one period.
\end{thm}

\newpage

\begin{thm}[10 Points]
	Graph the function $f(x) = x^2 - 2x - 3$.
	Label the $x$-intercepts, $y$-intercept, and the vertex.
	\vspace{3.5in}
\end{thm}

\newpage

\begin{thm}[10 Points]\label{ex3}
  Let $f(x) = x^2 - 9$ and $g(x) = \sqrt{x}$.
  \begin{enumerate}[(a)]
  \item
    Compute $(f \circ g)(x)$.
    \vspace{2in}
  \item
    Compute $(g \circ f)(x)$.
    \vspace{2in}
  \end{enumerate}
\end{thm}

\newpage

\begin{thm}[10 Points]
	Find all the solutions (real and complex) to the equation 
	$$x^4 - 8x = 0.$$
	\vspace{3in}
\end{thm}

\newpage

\begin{thm}[10 Points]
  Show that
  $$\tan\left(\sec^{-1}(x)\right) = \sqrt{x - 1}.$$
\end{thm}

\newpage

\begin{thm}[10 Points]
	Consider the polynomial $p(x) = x^3 - x^2 - 9x + 9$.
	Observe that 
	$$p(1) = 1 - 1 - 9 + 9 = 0.$$
	Use this information to factor $p(x)$ completely.
	\vspace{3.5in}
\end{thm}

\newpage

\begin{thm}[10 Points]
	Use the identity
        $$\sin(\theta - \phi) = \sin(\theta)\cos(\phi) - \sin(\phi)\cos(\theta)$$
	to show that
	$$\sin\left(\frac{\pi}{4}\right) = \cos\left(\frac{\pi}{4}\right).$$
        Hint: $\frac{1}{4} = \frac{1}{2} - \frac{1}{4}.$
\end{thm}

\newpage

\begin{thm}[10 Points]
	Simplify 
	$$\frac{x + \sqrt{2}}{x^2 - 2}$$
	by rationalizing the numerator.
	\vspace{3in}
\end{thm}

\newpage

\begin{thm}[10 Points]\label{ex4}
  Determine whether $f(x) = 3^{2x + 1}$ is invertible.
  If it is, then compute the inverse.  
  Otherwise, explain why it does not have an inverse.
\end{thm}

\newpage

\begin{thm}[10 Points]\label{ex5}
  Solve the following equations for $x$.
  \begin{enumerate}[(a)]
  \item
    $$\log_2(x^2 - 9) - \log_2(x - 3) = 3$$
    \vspace{3.5in}
  \item
    $$2^{x^2} = \frac{16^{x}}{16}$$
  \end{enumerate}
\end{thm}
\end{document}
