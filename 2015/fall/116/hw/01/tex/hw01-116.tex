\documentclass[12pt]{book}
\usepackage{amsmath,amsthm,amssymb,amsfonts,enumerate,mymath,tikz-cd,fancyhdr}
\openup 5pt
\author{Blake Farman\\University of South Carolina}
\title{Math 116\\Homework 01}
\date{October 18, 2016}
\pdfpagewidth 8.5in
\pdfpageheight 11in
\usepackage[margin=1in]{geometry}

\theoremstyle{definition}
\newtheorem{thm}{}
\renewcommand{\qedsymbol}{}

\begin{document}
\maketitle

\section*{1.1}
In the following exercies, simplify and reduce to lowest terms.
\setcounter{thm}{12}
\begin{thm}
  $\displaystyle{\frac{\frac{xy}{x + y}}{\frac{x^2y}{(x + y)^3}}}$
\end{thm}

\begin{thm}
$\displaystyle{\frac{\frac{xy}{x - y}}{\frac{x^2}{y} \cdot \frac{y^3}{x}}}$
\end{thm}

\section*{1.2}
In the following exercises, express as a single fraction and simplify.
\setcounter{thm}{15}
\begin{thm}
  $\displaystyle{\frac{\frac{1}{x} - \frac{1}{y}}{\frac{1}{x} + \frac{1}{y}}}$
\end{thm}

\setcounter{thm}{20}
\begin{thm}
  $\displaystyle{\frac{4yz}{x^2} - \frac{2z}{xy^2} + \frac{1}{xyz}}$
\end{thm}

\section*{1.3}
In the following exercises, simplify.
\setcounter{thm}{6}
\begin{thm}
  $\displaystyle{2x(y - 3) - y(x + xy) + 2y(x + 1)}$
\end{thm}

\setcounter{thm}{7}
\begin{thm}
  $\displaystyle{x(y + z) - z(x + y) + 2y(x - z) - x(3y - 2z)}$
\end{thm}

\section*{1.4}

\setcounter{thm}{9}

\begin{thm}
  Show by example that $(x^{-2} + y^{-2})^2 \neq x^{-4} + y^{-4}$; that is, find values for $x$ and $y$ so that the two sides are unequal for those values ({\it Hint}: Just dive in and try some.  Maybe you'll be lucky).
\end{thm}

Simplify using only positive exponents:
\setcounter{thm}{13}
\begin{thm}
  $\displaystyle{\frac{x^4y^2}{x^{-3}} \div \frac{x^3y^{-2}}{y^5}}$
\end{thm}

\section*{1.5}
Simplify the expression as much as possible, using rational exponent notation where appropriate:
\setcounter{thm}{13}
\begin{thm}
  $\displaystyle{\left(\frac{25}{16}\right)^{-3/2}}$
\end{thm}

\setcounter{thm}{29}
\begin{thm}
  If $x^2 + y^2 = 25$, can we conclude that $x + y = 5$?
  Why or why not?
\end{thm}

\section*{1.8}

\setcounter{thm}{0}
\begin{thm}
  Represent the following sets of numbers using interval notation and number line representation:
  \begin{enumerate}[(a)]
    \item
      $\displaystyle{-1 \leq x \leq 3}$
    \item
      $\displaystyle{-1 < x \leq 3}$
    \item
      $\displaystyle{-3 \leq x < 1}$
    \item
      $\displaystyle{-3 \leq x \leq 4}$
  \end{enumerate}
  
\end{thm}

\setcounter{thm}{2}
\begin{thm}
  Represent the following intervals using inequalities:
  \begin{enumerate}[(a)]
  \item
    $\displaystyle{(3,7)}$
  \item
    $\displaystyle{(-4,-1]}$
  \item
    $\displaystyle{(-\infty, 19]}$
  \item
    $\displaystyle{[2,10)}$
  \item
    $\displaystyle{[-2,-1]}$
  \end{enumerate}
\end{thm}

\setcounter{thm}{4}
\begin{thm}
  Simplify if possible:
  \begin{enumerate}[(a)]
  \item
    $\displaystyle{(-\infty, 5) \cap [3,\infty)}$
  \item
    $\displaystyle{(-\infty, 5) \cup [3,\infty)}$
  \item
    $\displaystyle{(-\infty, -2) \cap [-2, \infty)}$
  \item
    $\displaystyle{(-\infty, \infty) \cap [4,7]}$
  \item
    $\displaystyle{[3,5] \cap (10, \infty)}$
  \item
    $\displaystyle{(-\infty, 5] \cap [5, \infty)}$
  \end{enumerate}
\end{thm}
\end{document}
