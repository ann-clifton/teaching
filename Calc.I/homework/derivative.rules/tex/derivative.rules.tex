\documentclass[10pt]{amsart}
\usepackage{style}

\title{Derivative Rules}
\date{November 12, 2018}
\author{Blake Farman}
\address{Lafayette College}

\begin{document}
\maketitle

\makenameslot

\noindent Use the following to compute the derivative of the given function.
\begin{thm*}
  Let \(c\) and \(n\) be constants.
  If \(f\) and \(g\) are differentiable functions, then
  \begin{enumerate}[(a)]
  \item
    \(\frac{\dif}{\dif x}\left(c\right) = 0\)
  \item
    \(\frac{\dif}{\dif x}\left(x^n\right) = nx^{n-1}\)
  \item
    \(\frac{\dif}{\dif x}\left(cf(x)\right) = cf^\prime(x)\)
  \item
    \(\frac{\dif}{\dif x}\left(f(x) + g(x)\right) = f^\prime(x) + g^\prime(x)\)
  \item
    \(\frac{\dif}{\dif x}\left(f(x) - g(x)\right) = f^\prime(x) - g^\prime(x)\)
  \end{enumerate}
\end{thm*}

\begin{thm}
  \(f(x) = \pi^{400}\)
\end{thm}

\vspace{1.5in}
\begin{thm}
  \(f(x) = 10x^4 + 3x^2 - 7x + 500\pi\)
\end{thm}

\newpage

\begin{thm}
  \(f(x) = 6\sqrt[3]{x^2} + 2\sqrt{x^3}\)
\end{thm}

\vspace{1.5in}

\begin{thm}
  \(f(x) = (x + 2)^2\)
\end{thm}

\vspace{1.5in}

\begin{thm}
  \(f(x) = (3x - 1)(x + 2)\)
\end{thm}

\vspace{1.5in}

\begin{thm}
  \(f(x) = \dfrac{1}{x^{12}} + 7x - 21\)
\end{thm}

\newpage

\noindent Find the equation of the line tangent to the given curve at the given point.
\begin{thm}
  \(f(x) = 2x^3 - x^2 + 2\), \((1,3)\).
\end{thm}

\vspace{2in}

\begin{thm}
  \(f(x) = \sqrt{x}\), \((1,1)\).
\end{thm}

\vspace{2in}

\begin{thm}
  \(f(x) = x^2\), \((1,1)\)
\end{thm}
\end{document}

