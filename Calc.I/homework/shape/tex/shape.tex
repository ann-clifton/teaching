\documentclass[10pt]{amsart}
\usepackage{style}

\title{Derivatives and Shape}
\date{September 17, 2018}
\author{Blake Farman}
\address{Lafayette College}

\begin{document}
\maketitle

\makenameslot
\section*{Critical Numbers}
\noindent
Find the critical numbers of the given function
\begin{multicols}{2}
  \begin{thm}
    \(f(x) = x^3 + 6x^2 - 15x\)
  \end{thm}
  
  \begin{thm}
    \(f(x) = 2x^3 + x^2 + 2x\)
  \end{thm}
\end{multicols}

\vspace{2in}

\begin{multicols}{2}
  \begin{thm}
    \(f(x) = \abs{3x - 4}\)
  \end{thm}

  \begin{thm}
    \(f(x) = \dfrac{x - 1}{x^2 - y + 1}\)
  \end{thm}
\end{multicols}

\newpage
\section*{First Derivatives}
\noindent
For each function use the first derivative to find
\begin{itemize}
\item
  the interval(s) where the given function is increasing
\item
  the interval(s) where the given function is decreasing
\item
  the local maximum/minimum values
\end{itemize}
\begin{multicols}{2}
  \begin{thm}
    \(f(x) = 2x^3 - 9x^2 + 12x - 3\)
  \end{thm}

  \begin{thm}
    \(f(x) = x^4 - 2x^2 + 3\)
  \end{thm}
\end{multicols}

\vspace{3in}
\begin{multicols}{2}
  \begin{thm}
    \(f(x) = \dfrac{x}{x^2+1}\)
  \end{thm}

  \begin{thm}
    \(f(x) = \dfrac{x^2}{x - 4}\)
  \end{thm}
\end{multicols}

\newpage
\section*{Second Derivatives}
\noindent
For each function use the second derivative to find
\begin{itemize}
\item
  the interval(s) where the given function is concave up
\item
  the interval(s) where the given function is concave down
\item
  the local maximum/minimum values
\item
  the inflection points
\end{itemize}
\begin{multicols}{2}
  \begin{thm}
    \(f(x) = 2x^3 - 9x^2 + 12x - 3\)
  \end{thm}

  \begin{thm}
    \(f(x) = x^4 - 2x^2 + 3\)
  \end{thm}
\end{multicols}

\vspace{3in}
\begin{multicols}{2}
  \begin{thm}
    \(f(x) = \dfrac{x}{x^2+1}\)
  \end{thm}

  \begin{thm}
    \(f(x) = \dfrac{x^2}{x - 4}\)
  \end{thm}
\end{multicols}
\end{document}

