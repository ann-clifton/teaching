\documentclass[10pt]{amsart}
\usepackage{style}

\title{Curve Sketching}
\date{September 17, 2018}
\author{Blake Farman}
\address{Lafayette College}

\begin{document}
\maketitle

\makenameslot

\begin{thm}
  Sketch the curve \[f(x) = 3x^4 - 8x^3 + 6x^2\]
  \begin{enumerate}[(a)]
  \item\label{sketching first step}
    State the domain of \(f\).
    \vspace{1in}
  \item
    Find the intercepts and express them as an \((x,y)\) pair.
    Write NONE if there are none.
    \begin{center}
      \begin{tabular}{rl}
      x-intercept(s): & \line(1,0){300} \\ \\
      y-intercept: & \line(1,0){300}
      \end{tabular}
    \end{center}
    \vspace{1in}
  \item
    Is the function even, odd, or neither? What type of symmetry does the function have?
    \newpage
  \item
    Find the asymptotes.
    Write NONE if there are none.\\ \\
    \begin{center}
      \begin{tabular}{rl}
        Horizontal: & \line(1,0){300}\\\\
        Oblique: & \line(1,0){300}\\\\
        Vertical: & \line(1,0){300}
      \end{tabular}
    \end{center}
  \item
    Find the intervals where the function is increasing and decreasing.
    Write NONE if not applicable.\\ \\
    \begin{center}
      \begin{tabular}{rl}
        Increasing: & \line(1,0){300}\\\\
        Decreasing: & \line(1,0){300}
      \end{tabular}
    \end{center}
  \item
    State the local maximum and local minimum value(s).
    Write NONE if not applicable.\\ \\
    \begin{center}
      \begin{tabular}{rl}
        Local maximum value(s): & \line(1,0){300}\\\\
        Local minimum value(s): & \line(1,0){300}
      \end{tabular}
    \end{center}

  \item\label{sketching penultimate step}
    Find the intervals on which the function is concave up and concave down. State the inflection points.
    Write NONE if not applicable.\\ \\
    \begin{center}
      \begin{tabular}{rl}
        Concave Up: & \line(1,0){300}\\\\
        Concave Down: & \line(1,0){300}\\\\
        Inflection Points: & \line(1,0){300}
      \end{tabular}
    \end{center}
    \newpage
  \item
    Use your answers to Parts~\eqref{sketching first step}-\eqref{sketching penultimate step} to sketch the curve.
    Be sure that your graph is labeled and neat. Messy/incoherent graphs will receive zero points.
  \end{enumerate}
\end{thm}

\newpage
\begin{thm}
  Sketch the curve \[f(x) = \frac{x^2 + 1}{x+1}\]
  \begin{enumerate}[(a)]
  \item\label{sketching first step}
    State the domain of \(f\).
    \vspace{1in}
  \item
    Find the intercepts and express them as an \((x,y)\) pair.
    Write NONE if there are none.
    \begin{center}
      \begin{tabular}{rl}
      x-intercept(s): & \line(1,0){300} \\ \\
      y-intercept: & \line(1,0){300}
      \end{tabular}
    \end{center}
    \vspace{1in}
  \item
    Is the function even, odd, or neither? What type of symmetry does the function have?
    \newpage
  \item
    Find the asymptotes.
    Write NONE if there are none.\\ \\
    \begin{center}
      \begin{tabular}{rl}
        Horizontal: & \line(1,0){300}\\\\
        Oblique: & \line(1,0){300}\\\\
        Vertical: & \line(1,0){300}
      \end{tabular}
    \end{center}

  \item
    Find the intervals where the function is increasing and decreasing.
    Write NONE if not applicable.\\ \\
    \begin{center}
      \begin{tabular}{rl}
        Increasing: & \line(1,0){300}\\\\
        Decreasing: & \line(1,0){300}
      \end{tabular}
    \end{center}
  \item
    State the local maximum and local minimum value(s).
    Write NONE if not applicable.\\ \\
    \begin{center}
      \begin{tabular}{rl}
        Local maximum value(s): & \line(1,0){300}\\\\
        Local minimum value(s): & \line(1,0){300}
      \end{tabular}
    \end{center}

  \item\label{sketching penultimate step}
    Find the intervals on which the function is concave up and concave down. State the inflection points.
    Write NONE if not applicable.\\ \\
    \begin{center}
      \begin{tabular}{rl}
        Concave Up: & \line(1,0){300}\\\\
        Concave Down: & \line(1,0){300}\\\\
        Inflection Points: & \line(1,0){300}
      \end{tabular}
    \end{center}
    \newpage
  \item
    Use your answers to Parts~\eqref{sketching first step}-\eqref{sketching penultimate step} to sketch the curve.
    Be sure that your graph is labeled and neat. Messy/incoherent graphs will receive zero points.
  \end{enumerate}
\end{thm}
\end{document}

