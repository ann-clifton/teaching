\documentclass[10pt]{amsart}
\usepackage{style}

\title[Chain Rule]{Chain Rule Practice}
\date{September 21, 2018}
\author{Blake Farman}
\address{Lafayette College}

\begin{document}
\maketitle

\makenameslot

\begin{thm*}[The Chain Rule]
  Assume that \(g\) is differentiable at \(x\) and \(f\) is differentiable at \(g(x)\).
  Then composition of \(f\) with \(g\), \(f \circ g(x) = f\left(g(x)\right)\), is differentiable at \(x\) and
  \[\frac{\dif}{\dif x} f \circ g(x) = f^\prime\left(g(x)\right)\cdot g^\prime(x).\]
\end{thm*}

\begin{thm}
  Let \(f(x) = (3x^2 + 1)^2\).
  
  \begin{enumerate}[(a)]
  \item
    Expand \(f(x)\), then take the derivative.
    \vspace{1in}
  \item
    Write \(f(x) = (3x^2 + 1)^2 = (3x^2 + 1)(3x^2 + 1)\) and apply the Product Rule.
    \vspace{1in}
  \item
    Apply the chain rule directly to \(f(x)\).
    \newpage
  \item
    Are your answers in parts (a), (b), and (c) the same?
    Why or why not?
    \vspace{1.5in}
  \end{enumerate}
\end{thm}

\begin{thm}
  Assume that \(f\) is a differentiable function and let \(g(x) = \left(f\left(\sqrt{x}\right)\right)^3\).
  \begin{enumerate}[(a)]
  \item
    Compute \(g^\prime(x)\).  Your answer should include both \(f\) and \(f^\prime\).
    \vspace{2in}
  \item
    If \(f(2) = 1\) and \(f^\prime(2) = -2\), calculate \(g^\prime(4)\).
    \vspace{1in}
  \end{enumerate}
\end{thm}

Find the derivative of the given function.
\begin{thm}
  \(f(x) = (1 + x + x^2)^{99}\)
\end{thm}
\vspace{1.5in}

\begin{thm}
  \(g(\theta) = (2 - \sin(\theta))^{3/2}\)
\end{thm}

\newpage

\begin{thm}
  \(f(x) = (2x - 3)^4 (x^2 + x + 1)^5\).
\end{thm}
\vspace{1.5in}

\begin{thm}
  \(g(\theta) = \cos(\theta)^2\)
\end{thm}
\vspace{1.5in}

\begin{thm}
  \(h(y) = \left(\dfrac{y^4 + 1}{y^2 + 1}\right)^5\)
\end{thm}
  \vspace{1.5in}

\begin{thm}
  \(f(\theta) = \cot^2(\sin(\theta))\)
\end{thm}
  \vspace{1.5in}

\begin{thm}
  \(f(t) = \sqrt{\dfrac{t}{t^2 + 4}}\)
\end{thm}

\end{document}

