\documentclass[10pt]{amsart}
\usepackage{amsmath,amsthm,amssymb,amsfonts,enumerate,mymath,tikz-cd,fancyhdr}
\openup 5pt
\author{Blake Farman\\University of South Carolina}
\title{Math 141: Hour Quiz 1}
\date{February 21, 2013}
\pdfpagewidth 8.5in
\pdfpageheight 11in
\usepackage[margin=1in]{geometry}

\begin{document}
\maketitle

\begin{center}
\fbox{\fbox{\parbox{5.5in}{\centering
Answer the questions in the spaces provided on the
question sheets and turn them in at the end of the lab. 
You may use all prior lab work (homework, lab activities, etc.).
However, you may not work with any other student.
Unless otherwise stated, no supporting work is required.}}}
\end{center}

\vspace{0.2in}
\makebox[\textwidth]{Name and section:\enspace\hrulefill}
\vspace{0.2in}

\theoremstyle{plain}
\newtheorem{thm}{}
\newtheorem{lem}{Lemma}
\theoremstyle{definition}
\newtheorem{defn}{Definition}

\begin{thm}[4 Points]
  Let $f(x) = x + \frac{1}{x}$ and $g(x) = \frac{x+1}{x + 2}$.
  \begin{enumerate}[(a)]
  \item
    What is the domain of $f(x)$?
    \vspace{.5in}
  \item
    What is the domain of $g(x)$?
    \vspace{.5in}
  \item
    Compute $(f \circ g)(x)$.
    Simplify your answer.
    \vspace{.5in}
  \item
    Find the domain of $(f \circ g)(x)$. 
    \vspace{.5in}
  \end{enumerate}
\end{thm}

\begin{thm}[1 Point]
  Compute $\lim_{x \rightarrow 0} \frac{\sqrt{x + 4} - 2}{x}$.
\end{thm}
\pagebreak
\begin{thm}[2 Points]
  Let $f(x) = \frac{t^2 - 9}{2t^2 +7t + 3}$.
  \begin{enumerate}[(a)]
  \item
    Compute $\lim_{t \rightarrow -3} f(t).$
    \vspace{.5in}
  \item
    Justify whether $f(t)$ is continuous at $t = -3$?
    If not, how can you make $f(t)$ continuous at $t = -3$?
    \vspace{1.0in}
  \end{enumerate}
\end{thm}

\begin{thm}[4 Points]
  Let $f(x) = \frac{2x^2 + x - 1}{x^2 + x - 2}$.
  \begin{enumerate}[(a)]
  \item
    What is the domain of $f(x)$?
    \vspace{.5in}
  \item
    Find all horizontal asymptotes for $f(x)$.
    \vspace{.5in}
  \item
    Find all vertical asymptotes for $f(x)$.
    \vspace{.5in}
  \item
    Use this information to sketch a graph of $f(x)$.
    \vspace{.5in}
  \end{enumerate}
\end{thm}
\pagebreak
\begin{thm}[2 Points]
  A ball is thrown into the air with an initial velocity of $40$ ft/s.
  Its height (in feet) after $t$ seconds is given by $f(t) = 40t - 16t^2$.
  \begin{enumerate}[(a)]
  \item
    Find the velocity when $t = 2$.
    \vspace{.5in}
  \item
    Find the tangent line at $t = 2$.
    \vspace{.5in}
  \end{enumerate}
\end{thm}

\begin{thm}[2 Points]
  Let $f(x) = 2x^3 + x^2 + 2$.
  \begin{enumerate}[(a)]
    \item
      Use the Intermediate Value Theorem to briefly justify that $f(x)$ has a root, $c$, in the interval (-2,-1).
      [Hint: You may find it helpful to plot the function in Maple.]
      \vspace{1.5in}
    \item
      Use the fsolve command to find the value of $c$.
      Approximate this value to 5 decimal places.
      \vspace{.5in}
  \end{enumerate}
\end{thm}

\begin{thm}[Bonus - 5 Points]
  Decide whether the following statement is true or false:
  \begin{quotation}
    Let $f(x)$ be a function defined on the closed interval $[a,b]$ with $f(a) \neq f(b)$.
    For every real number $t$ satisfying $f(a) < t < f(b)$, there exists a real number $c$ satisfying $a < c < b$ such that $f(c) = t$.
  \end{quotation}
  If you believe it to be true, briefly explain why.
  If you believe it to be false, provide a function $f(x)$ and a closed interval $[a,b]$ where this statement fails to be true.
\end{thm}
\end{document}
