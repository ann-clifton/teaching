\documentclass[12pt]{amsart}
\usepackage{amsmath,amsthm,amssymb,amsfonts,enumerate,mymath,tikz-cd,fancyhdr}
\openup 5pt
\author{Blake Farman\\University of South Carolina}
\title{Math 116: Exam 02}
\date{December 2, 2013}
\pdfpagewidth 8.5in
\pdfpageheight 11in
\usepackage[margin=1in]{geometry}

\renewcommand{\qedsymbol}{}

\begin{document}
\maketitle

\begin{center}
\fbox{\fbox{\parbox{5.5in}{\centering
Answer the questions in the spaces provided on the
question sheets and turn them in at the end of the class period. 
Unless otherwise stated, all supporting work is required.
You may {\it not} use any calculators.}}}
\end{center}

\vspace{0.2in}
\makebox[\textwidth]{Name:\enspace\hrulefill}
\vspace{0.2in}

\theoremstyle{plain}
\newtheorem{thm}{}
\newtheorem{lem}{Lemma}
\theoremstyle{definition}
\newtheorem{defn}{Definition}

\begin{thm}[20 Points]\label{ex1}
  Find the period, frequency, and amplitude of $y = 4\sin(3x) - 1$, then graph one period.
\end{thm}

\newpage

\begin{thm}[20 Points]\label{ex2}
  Find the period, frequency, and amplitude of $y = 3\cos(2x) + 2$, then graph one period.
\end{thm}

\newpage
\begin{thm}[20 Points]\label{ex3}
  Let $f(x) = x^2 - 2x$ and $g(x) = \sqrt{x}$.
  \begin{enumerate}[(a)]
  \item
    Compute $(f \circ g)(x)$.
    \vspace{2in}
  \item
    Compute $(g \circ f)(x)$.
    \vspace{2in}
  \end{enumerate}
\end{thm}

\begin{thm}[20 Points]\label{ex4}
  Determine whether $g(x) = \sqrt[3]{5x + 1}$ is invertible.
  If it is, then compute the inverse.  
  Otherwise, explain why it does not have an inverse.
\end{thm}

\newpage

\begin{thm}[20 Points]\label{ex5}
  Solve the following equations for $x$.
  \begin{enumerate}[(a)]
  \item
    $$ 2\log_2(\sqrt{x + 2}) - \log_2\left(\frac{1}{x - 2}\right) = 5 $$
    \vspace{3.5in}
  \item
    $$2^{-4x} = 16 \cdot 2^{x^2}$$
  \end{enumerate}
\end{thm}
\end{document}
