\documentclass[12pt]{amsart}
\usepackage{amsmath,amsthm,amssymb,amsfonts,enumerate, paralist,xcolor,mymath}
\openup 5pt
\author{Blake Farman\\University of South Carolina}
\title{Math 111:\\Homework 01 Solutions}
\date{May 20, 2013}
\pdfpagewidth 8.5in
\pdfpageheight 11in
\usepackage[margin=1in]{geometry}

\newtheorem{thm}{}

\renewcommand{\qedsymbol}{}

\begin{document}
\maketitle

\section*{A.2}

\setcounter{thm}{2}
\begin{thm}
  The set of numbers between but not including $2$ and $7$ can be written as follows:
  
  \begin{proof}[Solution]
    \begin{enumerate}[(a)]
    \item
      $\left\{x \in \R \;\middle\vert\; 2 < x < 7 \right\}$ in set builder notation.
    \item
      $(2, 7)$ in interval notation.
    \end{enumerate}    
  \end{proof}
\end{thm}

\begin{thm}
  Explain the differences between the following two sets: $A = [-2, 5]$ and $B = (-2, 5)$.
  
  \begin{proof}[Solution]
    The set $A$ contains the endpoints, $2$ and $-5$, but the set $B$ does not.
  \end{proof}
\end{thm}

27 \& 29: Find the indicated set if $A = \{1,2,3,4,5,6,7\}$, $B = \{2,4,6,8\}$, and $C = \{7,8,9,10\}$.

\setcounter{thm}{26}
\begin{thm}
  \begin{inparaenum}[(a)]
  \item
    $A \cup B$
    \hspace{50mm}
  \item
    $A \cap B$
  \end{inparaenum}
  
  \begin{proof}[Solution]
    \begin{enumerate}[(a)]
    \item
      The union of $A$ and $B$ is the set containing the elements of both sets,
      $$A \cup B = \{1, 2, 3, 4, 5, 6, 7, 8\}.$$
    \item
      The intersection of $A$ and $B$ is the set containing the elements common to both sets,
      $$A \cap B = \{2, 4, 6\}.$$
    \end{enumerate}
  \end{proof}
\end{thm}

\setcounter{thm}{28}
\begin{thm}
  \begin{inparaenum}[(a)]
  \item
    $A \cup C$
    \hspace{50mm}
  \item
    $A \cap C$
  \end{inparaenum}
  
  \begin{proof}[Solution]
    \begin{enumerate}[(a)]
    \item
      The union of $A$ and $C$ is the set containing the elements of both sets, 
      $$A \cup C = \{1, 2, 3, 4, 5, 6, 7, 8, 9, 10\}.$$
    \item
      The intersection of $A$ and $B$ is the set containing the elements common to both sets,
      $$A \cap C = \{7\}.$$
    \end{enumerate}
  \end{proof}
\end{thm}

\setcounter{thm}{48}
\begin{thm}
  Express the inequality in interval notation, and then graph the corresponding interval.
  $$-5 \leq x < -2.$$
  
  \begin{proof}[Solution]
    The inequality $-5 \leq x < -2$ represents the half-open interval that includes the endpoint $-5$, but not the endpoint $-2$ and so can be written $[-5, 2)$.
    
  \end{proof}
\end{thm}

\setcounter{thm}{62}
\begin{thm}
  Find the distance between the given numbers.\\
  \begin{enumerate}[(a)]
  \item
    $2$ and $17$ 
  \item
    $-3$ and $21$
  \item
    $\frac{11}{8}$ and $-\frac{3}{8}$.
  \end{enumerate}
  
  \begin{proof}[Solution]
    \begin{enumerate}[(a)]
    \item
      The distance between $2$ and $17$ is given by
      $$\operatorname{d}(2, 17) = \abs{2 - 17} = \abs{-15} = 15.$$
    \item
      The distance between $-3$ and $21$ is giveb by
      $$\operatorname{d}(-3, 21) = \abs{-3 - 21} = \abs{-24} = 24.$$
    \item
      The distance between $11/8$ and $-3/8$ is given by 
      $$\operatorname{d}\left(\frac{11}{8}, -\frac{3}{8}\right) = \abs{\frac{11}{8} - \left(-\frac{3}{8}\right)} = \abs{\frac{11 + 3}{8}} = 14/8 = 7/4.$$
    \end{enumerate}
  \end{proof}
\end{thm}

\section*{A.3}

\setcounter{thm}{0}
\begin{thm}
  Using exponential notation, we can write the product $5 \cdot 5 \cdot 5 \cdot 5 \cdot 5 \cdot 5$ as \line(1,0){40}.
  
  \begin{proof}[Solution]
    This is the product of $5$ with itself 6 times, we we can write it as $5^6$.
  \end{proof}
\end{thm}

\setcounter{thm}{2}
\begin{thm}
  When we multiply two powers with the same base, we \line(1,0){40}\ the exponents.
  So\\ 
  \indent $3^4\cdot 3^5 =$ \line(1,0){40}.
  
  \begin{proof}[Solution]
    When we multiply two powers with the same base, we add the exponents, so 
    $$3^4 \cdot 3^5 = 3^{4 + 5} = 3^9.$$
  \end{proof}
\end{thm}

\setcounter{thm}{5}
\begin{thm}
  Express the following without using exponents.
  \begin{enumerate}[(a)]
  \item
    $2^{-1} =$ \line(1,0){40}.
  \item
    $2^{-3} =$ \line(1,0){40}.
  \item
    $\left(\frac{1}{2}\right)^{-1} =$ \line(1,0){40}.
  \end{enumerate}
  
  \begin{proof}[Solution]
    When a number is raised to a negative exponent, this is the same as raising its reciprocal to the absolute value of the exponent.
    Hence the solutions are
    \begin{enumerate}[(a)]
    \item
      $\displaystyle{2^{-1} = \left(\frac{1}{2}\right)^{1} = \frac{1}{2}}$,
    \item
      $\displaystyle{2^{-3} = \left(\frac{1}{2}\right)^3 = \frac{1}{2^3} = \frac{1}{8}}$,
    \item
      $\displaystyle{\left(\frac{1}{2}\right)^{-1}} = 2^1 = 2$.
    \end{enumerate}
  \end{proof}
\end{thm}

29 \& 33:  Use the rules of exponents to write each expression in as simple a form as possible.
\setcounter{thm}{28}
\begin{thm}
  $\displaystyle{\frac{7^5 \cdot 7^{-3}}{7^2}}$.
  
  \begin{proof}[Solution]
    Simplifying the numerator, we add the exponents in the product to obtain $7^5 \cdot 7^{-3} = 7^{5 + (-3)} = 7^2$.
    Therefore the ratio reduces to
    $$\frac{7^5 \cdot 7^{-3}}{7^2} = \frac{7^2}{7^2} = 1.$$
  \end{proof}
\end{thm}

\setcounter{thm}{32}
\begin{thm}
  $\displaystyle{\left(\frac{1}{2}\right)^4 \left(\frac{5}{2}\right)^{-2}}$.
  
  \begin{proof}[Solution]
    Expanding each ratio in turn we have
    $$\left(\frac{1}{2}\right)^4 = \frac{1}{2^4} = \frac{1}{16}$$
    and
    $$\left(\frac{5}{2}\right)^{-2} = \left(\frac{2}{5}\right)^2 = \frac{2^2}{5^2} = \frac{4}{25}.$$
    Combining these calculations we obtain
    $$\left(\frac{1}{2}\right)^4 \left(\frac{5}{2}\right)^{-2} = \frac{1}{16}\frac{4}{25} = \frac{1}{4\cdot 25} = \frac{1}{100}.$$
  \end{proof}
\end{thm}

\setcounter{thm}{51}
\begin{thm}
  Simplify the expression, and eliminate any negative exponents:
  $$\left(\frac{-2x^2}{y^3}\right)^3.$$
  
  \begin{proof}[Solution]
    Using the rules of exponentiation we have
    $$\left(\frac{-2x^2}{y^3}\right)^3 = \frac{(-2)^3(x^2)^3}{(y^3)^3} = \frac{-8x^{2\cdot 3}}{y^{3 \cdot 3}} = \frac{-8x^6}{y^9}$$
  \end{proof}
\end{thm}

\section*{A.4}

\setcounter{thm}{0}
\begin{thm}
  Using exponential notation, we can write $\sqrt[3]{5}$ as \line(1,0){40}.
  
  \begin{proof}[Solution]
    The third root of $5$, $\sqrt[3]{5}$, can be written using rational exponents as $5^{1/3}$.
  \end{proof}
\end{thm}

\setcounter{thm}{3}
\begin{thm}
  Explain what $4^{3/2}$ means, and then calculate $4^{3/2}$ in two different ways:
  $$4^{3/2} = \left(4^{1/2}\right)^{{\color{red}\fbox{}}} =\ \line(1,0){40} \hspace{10mm} 4^{3/2} = \left(4^3\right)^{{\color{red}\fbox{}}} =\ \line(1,0){40}$$
  
  \begin{proof}[Solution]
    The symbol $4^{3/2}$ means the cube of the square root of $4$, written $(\sqrt{4})^3$, or, equivalently, the square root of the cube of $4$, written $\sqrt{4^3}$.
    Using exponential notation, these can be written as $\left(4^{1/2}\right)^3$ and $\left(4^3\right)^{1/2}$, respectively.
  \end{proof}
\end{thm}

\setcounter{thm}{16}
\begin{thm}
  Evaluate the expressions
  \begin{enumerate}[(a)]
  \item
    $\sqrt{\frac{4}{9}}$,
  \item
    $\sqrt[4]{256}$,
  \item
    $\sqrt[6]{\frac{1}{64}}$.
  \end{enumerate}
  
  \begin{proof}[Solution]
    \begin{enumerate}[(a)]
    \item
      Using the rules of exponents, we have
      $$\sqrt{\frac{4}{9}} = \frac{\sqrt{4}}{\sqrt{9}} = \frac{2}{3}.$$
    \item
      First we observe that $256 = 2^8$.
      Then using the rules of exponents, it follows that
      $$\sqrt[4]{256} = \sqrt[4]{2^8} = 2^{8/4} = 2^2 = 4.$$
    \item
      Here we note that $64 = 2^6$ so we obtain
      $$\sqrt[6]{\frac{1}{64}} = \sqrt[6]{2^6} = \left(2^6\right)^{1/6} = 2^{6/6} = 2^1 = 2.$$
    \end{enumerate}
  \end{proof}
\end{thm}

23, 25, \& 33: simplify the expression and eliminate any negative exponent(s).
Assume that all letters denote positive numbers.
\setcounter{thm}{22}
\begin{thm}
  $\displaystyle{x^{3/4}x^{5/4}}$.
  
  \begin{proof}[Solution]
    Using the rules of exponents, we have
    $$x^{3/4}\cdot x^{5/4} = x^{3/4 + 5/4} = x^{8/4} = x^2.$$
  \end{proof}
\end{thm}

\setcounter{thm}{24}
\begin{thm}
  $\displaystyle{\left(4b\right)^{1/2}}$.
  
  \begin{proof}[Solution]
    Again using the rules of exponents, we obtain
    $$\left(4b\right)^{1/2} = \sqrt{4b} = \sqrt{4}\sqrt{b} = 2\sqrt{b}.$$
  \end{proof}
\end{thm}

\setcounter{thm}{32}
\begin{thm}
  $\displaystyle{\left(\frac{2q^{3/4}}{r^{3/2}}\right)}^{-4}$.
  
  \begin{proof}[Solution]
    Using the rules of exponents
    $$\left(\frac{2q^{3/4}}{r^{3/2}}\right)^{-4} = \left(\frac{r^{3/2}}{2q^{3/4}}\right)^{4} = \frac{(r^{3/2})^4}{(2q^{3/4})^4} = \frac{r^{(3/2)\cdot 4}}{2^4q^{(3/4)\cdot 4}} = \frac{r^6}{16q^3}.$$
  \end{proof}
\end{thm}

\setcounter{thm}{40}
\begin{thm}
  Simplify the expression and express the answer using rational exponents.
  Assume that all letters denote positive numbers.
  $$\left(5\sqrt[3]{x}\right)\left(2\sqrt[4]{x}\right).$$
  
  \begin{proof}[Solution]
    Using the rules for exponents,
    $$\left(5\sqrt[3]{x}\right)\left(2\sqrt[4]{x}\right) = 10x^{1/3}x^{1/4} = 10x^{1/3 + 1/4} = 10x^{4/12 + 3/12} = 10x^{7/12}.$$
  \end{proof}
\end{thm}

\section*{B.1}

\setcounter{thm}{0}
\begin{thm}
  To add expressions, we add \line(1,0){40}\ terms.
  So\\
  \indent $(3a + 2b + 4) + (a - b + 1) =\ \line(1,0){40}$.
  
  \begin{proof}[Solution]
    To add expressions, we add like terms, so
    $$(3a + 2b + 4) + (a - b + 1) = 4a + b + 5.$$
  \end{proof}
\end{thm}

\begin{thm}
  To subtract expressions, we substract\ \line(1,0){40}\ terms.
  So\\ 
  \indent $(2xy + 9b + c + 10) - (xy + b + 6c + 8) =\ \line(1,0){40}$.
  \begin{proof}[Solution]
    To subtract expressions, we subtract like terms, so
    $$(2xy + 9b + c + 10) - (xy + b + 6c + 8) = xy + 8b -5c + 2.$$
  \end{proof}
\end{thm}

\setcounter{thm}{3}
\begin{thm}
  Explain how we multiply two binomials, then perform the following multiplication:\\
  \indent $(x + 2)(x + 3) =\ \line(1,0){40}.$
  
  \begin{proof}[Solution]
    When mutliplying binomials, we use FOIL to expand the expression.
    Hence
    $$(x + 2)(x + 3) = x^2 + 3x + 2x + 6 = x^2 + (3 + 2)x + 6 = x^2 + 5x + 6.$$
  \end{proof}
\end{thm}

\begin{thm}
  The Special Product Formula for the "square of a sum" is\\
  \indent $(A + B)^2 =\ \line(1,0){40}$.  
  So $(2x + 3)^2 =\ \line(1,0){40}.$
  
  \begin{proof}[Solution]
    The Special Product Formula for the ``square of a sum'' is $(A + B)^2 = A^2 + 2AB + B^2$, so
    $$(2x + 3)^2 = (2x)^2 + 2(2x)(3) + 3^2 = 4x^2 + 12x + 9.$$
  \end{proof}
\end{thm}

\begin{thm}
  The Special Product Formula for the "sum and difference of the same terms" is\\
  \indent $(A + B)(A - B) =\ \line(1,0){40}$.
  So $(5 + x)(5 - x) =\ \line(1,0){40}$.
  \begin{proof}[Solution]
    The Special Product Formula for the ``sum and difference of the same terms'' is $(A+B)(A-B) = A^2 - B^2$, so 
    $$(5 + x)(5 - x) = 5^2 - x^2 = 25 - x^2.$$
  \end{proof}
\end{thm}

\setcounter{thm}{10}
\begin{thm}
  Consider the expression $2ax - 10a + 1$.
  \begin{enumerate}[(a)]
  \item
    What are the terms of the expression?
  \item
    Find the value of the expression if $a$ is $-1$, $b$ is $4$, $x$ is $-2$, and $y$ is $3$.
  \end{enumerate}
  
  \begin{proof}[Solution]
    \begin{enumerate}[(a)]
      \item
        The terms of the expression are $2ax$, $10a$, and $1$.
      \item
        The value of the expression is
        $$2ax - 10a + 1 = 2(-1)(-2) - 10(-1) + 1 = 4 + 10 + 1 = 15.$$
    \end{enumerate}
  \end{proof}
\end{thm}

15 \& 17: Find the sum or difference.
\setcounter{thm}{14}
\begin{thm}
  $(12x - 7) - (5x - 12)$.
  \begin{proof}[Solution]
    The difference is
    $$(12x - 7) - (5x - 12) = 12x - 5x + 12 - 7 = (12 - 5)x + 5 = 7x + 5.$$
  \end{proof}
\end{thm}

\setcounter{thm}{16}
\begin{thm}
  $(4x^2 + 2x) + (3x^2 - 5x + 6)$.
  \begin{proof}[Solution]
    The sum is
    $$(4x^2 + 2x) + (3x^2 - 5x + 6) = 4x^2 + 3x^2 + 2x - 5x + 6 = (4 + 3)x^2 + (2 - 5)x + 6 = 7x^2 - 3x + 6.$$
  \end{proof}
\end{thm}

\setcounter{thm}{23}
\begin{thm}
  Multiply the two expressions using the Distributive Property: $(a + b)(x - y)$.
  
  \begin{proof}[Solution]
    Using the Distributive Property on the left, we have
    $$(a + b)(x - y) = (a + b)x - (a + b)y.$$
    The right hand side of this equation can be expanded again by using the distributive property on the right to obtain
    $$(a + b)x - (a + b)y = ax + bx - ay + by.$$
    
    Similarly, one can start with distribution on the right and then apply it again on the right to obtain the same answer,
    $$(a + b)(x - y) = a(x - y) + b(x - y) = ax - ay + bx - by.$$
  \end{proof}
\end{thm}

\setcounter{thm}{28}
\begin{thm}
  Multiply the algebraic expressions using the FOIL method and simplify: $(r - 3)(r + 5)$.
  
  \begin{proof}[Solution]
    Using FOIL, we have
    $$(r - 3)(r + 5) = r^2 + 5r - 3r - 15 = r^2 + (5 - 3)r - 15 = r^2 + 2r - 15.$$
  \end{proof}
\end{thm}

35, 36, \& 45: Multiply the algebraic expressions using a Special Product Formula and simplify.
\setcounter{thm}{34}
\begin{thm}
  $(x + 3)^2$.
  
  \begin{proof}[Solution]
    Using the square of a sum formula, we have 
    $$(x + 3)^2 = x^2 + 6x + 9.$$
  \end{proof}
\end{thm}

\begin{thm}
  $(x - 2)^2$.
  \begin{proof}[Solution]
    Using the square of a difference formula, we have
    $$(x - 2)^2 = x^2 - 2x + 4.$$
  \end{proof}
\end{thm}

\setcounter{thm}{44}
\begin{thm}
  $(x + 5)(x - 5)$.
  
  \begin{proof}[Solution]
    Using the difference of squares formula, we have
    $$(x + 5)(x - 5) = x^2 - 25.$$
  \end{proof}
\end{thm}

\setcounter{thm}{60}
\begin{thm}
  Find the product of the polynomials: $(x + 2)(x^2 + 2x + 3)$.
  \begin{proof}[Solution]
    By distribution on the left, we have
    $$(x + 2)(x^2 + 2x + 3) = (x + 2)x^2 + (x + 2)2x + (x + 2)3.$$
    Then applying distribution on the right, 
    \begin{eqnarray*}
      (x + 2)x^2 + (x+2)2x + (x + 2)3 &=& x^3 + 2x^2 + 2x^2 + 4x + 3x + 6\\
      &=& x^3 + (2 + 2)x^2 + (4 + 3)x + 6\\
      &=& x^3 + 4x^2 + 7x + 6.
    \end{eqnarray*}
  \end{proof}
\end{thm}

\section*{B.2}

\setcounter{thm}{0}
\begin{thm}
  Consider the polynomial $2x^5 + 6x^4 + 4x^3$.
  \begin{enumerate}[(a)]
  \item
    How many terms does this polynomial have?
  \item
    List the terms.
  \item
    What factor is common to each term?
  \item
    Factor the polynomial: $2x^5 + 6x^4 + 4x^3 = \line(1,0){40}$.
  \end{enumerate}
  
  \begin{proof}[Solution]
    \begin{enumerate}[(a)]
      \item
        This polynomial has three terms.
      \item
        The terms are $2x^5$, $6x^4$, and $4x^3$.
      \item
        The factor common to each is $2x^3$.
      \item
        The polynomial factors as 
        \begin{eqnarray*}
          2x^5 + 6x^4 + 4x^3 &=& 2x^3(x^2 + 3x + 2)\\
          &=& 2x^3(x + 2)(x + 1).
        \end{eqnarray*}
    \end{enumerate}
  \end{proof}
\end{thm}

\begin{thm}
  To factor the trinomial $x^2 + 7x + 10$, we look for two integers whose product is\ \line(1,0){40} and whose sum is\ \line(1,0){40}.
  These integers are\ \line(1,0){40} and\ \line(1,0){40}, so the trinomial factors as\ \line(1,0){40}.
  
  \begin{proof}[Solution]
    To factor the trinomial $x^2 + 7x + 10$, we look for two integers whose product is $10$ and whose sum is $7$.
    These integers are $2$ and $5$, so the trinomial factors as $(x + 2)(x + 5)$.
  \end{proof}
\end{thm}

\setcounter{thm}{4}
\begin{thm}
  Factor $5a - 20$.
  
  \begin{proof}[Solution]
    The only factor common to $5a$ and $20$ is $5$, so we have the factorization
    $$5a - 20 = 5(a - 4).$$
  \end{proof}
\end{thm}

\setcounter{thm}{6}
\begin{thm}
  Factor $30x^3 + 15x^4$.
  
  \begin{proof}[Solution]
    Since $30 = 2\cdot 15$, and both terms contain an $x^3$, we have the factorization
    $$30x^3 + 15x^4 = 15x^3(2 + x).$$
  \end{proof}
\end{thm}

\setcounter{thm}{8}
\begin{thm}
  Factor $-2x^3 + 16x$.
  
  \begin{proof}[Solution]
    Factoring a $-2x$ out of both terms, we have
    $$-2x^3 + 16x = -2x(x^2 - 8).$$
  \end{proof}
\end{thm}

\setcounter{thm}{20}
\begin{thm}
  Factor the trinomial $y^2 - 8y + 15$.
  
  \begin{proof}[Solution]
    Here we need two integers that multiply to $15$ and add to $-8$.
    These integers are $-5$ and $-3$, so we have
    $$y^2 - 8y + 15 = (y - 5)(y - 3).$$
  \end{proof}
\end{thm}

\setcounter{thm}{24}
\begin{thm}
  Factor the trinomial $5x^2 - 7x - 6$.
  
  \begin{proof}[Solution]
    Here, since we the leading coefficient is not $1$, we need to consider a factorization of the form $(ax + r)(bx + s) = abx + (as + br)x + rs$.
    We must have $ab = 5$, so our factorization must look like $(5x + r)(x + s)$.
    Hence we need only decide on integers $r, s$ such that $5s + r = -7$ and $rs = -6$.
    Choosing $s = -2$ and $r = 3$, we see that $5(-2) + 3 = -7$ and $(-2)3 = -6$, so our factorization is
    $$5x^2 - 7x - 6 = (5x + 3)(x - 2).$$
  \end{proof}
\end{thm}

\end{document}
