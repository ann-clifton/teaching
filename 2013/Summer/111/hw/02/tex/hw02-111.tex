\documentclass[12pt]{amsart}
\usepackage{amsmath,amsthm,amssymb,amsfonts,enumerate, paralist,xcolor,mymath}
\openup 5pt
\author{Blake Farman\\University of South Carolina}
\title{Math 111:\\Homework 02}% Solutions}
\date{May 22, 2013}
\pdfpagewidth 8.5in
\pdfpageheight 11in
\usepackage[margin=1in]{geometry}

\newtheorem{thm}{}

\renewcommand{\qedsymbol}{}

\begin{document}
\maketitle

\section*{B.3}

\setcounter{thm}{4} 
\begin{thm}
  Consider the expression
  $$\frac{1}{x} - \frac{2}{x + 1} - \frac{x}{(x+1)^2}.$$
  \begin{enumerate}[(a)]
  \item
    How many terms does this expression have?
  \item
    Find the least common denominator of all the terms.
  \item
    Perform the addition and simplify.
  \end{enumerate}
  \begin{proof}[Solution]
    \begin{enumerate}[(a)]
    \item
      There are three terms in this expression,
      $$\frac{1}{x},\, \frac{2}{x + 1},\, \text{and}\ \frac{x}{(x+1)^2}.$$
    \item
      The least common denominator is found by taking the least common multiple of all three of the denominators, $x$, $x+1$, and $(x+1)^2$.
      Since the last two denominators have $x+1$ is common, the least common multiple is found by combining the largest power of the $x+1$, which is $(x+1)^2$, and the first denominator $x$.
      Hence we have $x(x+1)^2$ as our least common denominator.
    \item
      First we manipulate the original expression so that each have a common denominator,
      $$\frac{(x+1)^2}{(x+1)^2} \cdot \left(\frac{1}{x}\right) - \frac{x(x+1)}{x(x+1)} \cdot \left(\frac{2}{x+1}\right) - \frac{x}{x} \cdot \left(\frac{x}{(x+1)^2}\right) = \frac{(x + 1)^2 - 2x(x+1) - x^2}{x(x+1)^2}.$$
      Expanding the numerator and collecting like terms we arrive at the final answer
      \begin{eqnarray*}
        \frac{(x^2 + 2x + 1) - (2x^2 + 2x) - (x^2)}{x(x+1)^2} &=& (x^2 - 2x^2 - x^2) + (2x - 2x) + 1\\
        &=& \frac{1 - 2x^2}{x(x+1)^2}
      \end{eqnarray*}
    \end{enumerate}
  \end{proof}
\end{thm}

\setcounter{thm}{16}
\begin{thm}
  Simplify the rational expression
  $$\frac{x^2 + 6x + 8}{x^2 + 5x + 4}.$$
  \begin{proof}[Solution]
    First we start by factoring the numerator and the denominator.
    In the numerator, we need two numbers that add to $6$ and multiply to $8$.
    These are $2$ and $4$, so we have the factorization $x^2 + 6x + 8 = (x + 2)(x + 4)$.
    In the denominator, we need two numbers that add to $5$ and multiply to $4$.
    These are $4$ and $1$, so we have the factorization $x^2 + 5x + 4 = (x + 1)(x + 4)$.
    Rewriting our rational expression and simplifying we have
    $$\frac{x^2 + 6x + 8}{x^2 + 5x + 4} = \frac{(x + 2)(x + 4)}{(x + 1)(x + 4)} = \frac{x + 2}{x + 1}.$$
  \end{proof}
\end{thm}

\setcounter{thm}{20}
\begin{thm}
  Perform the multiplication
  $$\frac{4x}{x^2 - 4} \cdot \frac{x + 2}{16x}.$$
  \begin{proof}[Solution]
    First we observe that we have the factorization $x^2 - 4 = (x + 2)(x - 2)$.
    Now, when we multiply rational expressions we multiply together the numerator and denominators, so
    $$\frac{4x}{x^2 - 4} \cdot \frac{x + 2}{16x} = \frac{4x(x+2)}{16x(x + 2)(x - 2)}.$$
    Noting that $16 = 4^2$, we cancel common factors in the numerator and denominator to obtain
    $$\frac{4x(x+2)}{16x(x + 2)(x - 2)} = \frac{4x(x+2)}{4^2x(x + 2)(x - 2)} = \frac{1}{4(x - 2)}.$$
  \end{proof}
\end{thm}

\setcounter{thm}{32}
\begin{thm}
  Perform the addition
  $$\frac{1}{x + 5} + \frac{2}{x - 3}.$$
  
  \begin{proof}[Solution]
    To add these rational expressions we need to first find a common denominator.
    This will be the product of $x + 5$ and $x - 3$, so 
    \begin{eqnarray*}
      \frac{1}{x + 5} + \frac{2}{x - 3} &=& \frac{x - 3}{x - 3} \cdot \left( \frac{1}{x + 5} \right) + \frac{x + 5}{x + 5} \cdot \left( \frac{2}{x - 3} \right)\\
      &=& \frac{x - 3}{(x -3)(x + 5)} + \frac{2(x + 5)}{(x - 3)(x + 5)}\\
      &=& \frac{(x - 3) + 2(x + 5)}{(x - 3)(x + 5)}\\
      &=& \frac{x - 3 + 2x + 10}{(x - 3)(x + 5)}\\
      &=& \frac{3x + 7}{(x - 3)( x + 5)}
    \end{eqnarray*}
    
  \end{proof}
\end{thm}

\setcounter{thm}{42}
\begin{thm}
  Rationalize the denominator
  $$\frac{2}{\sqrt{2} + \sqrt{7}}.$$
  
  \begin{proof}[Solution]
    To rationalize the denominator, we need to use the conjugate of $\sqrt{2} + \sqrt{7}$, which is $\sqrt{2} - \sqrt{7}$.
    Recall that by the sum and difference of same terms formula, 
    $$(\sqrt{2} + \sqrt{7})(\sqrt{2} - \sqrt{7}) = (\sqrt{2})^2 - \sqrt{7}^2 = 2 - 7 = -5.$$
    We now multiply the numerator and denominator by the conjugate, so as to preserve its value, to obtain
    $$\frac{2}{\sqrt{2} + \sqrt{7}} = \frac{\sqrt{2} - \sqrt{7}}{\sqrt{2} - \sqrt{7}} \cdot \left(\frac{2}{\sqrt{2} + \sqrt{7}}\right) = \frac{2(\sqrt{2} - \sqrt{7})}{-5} = -\frac{2(\sqrt{2} - \sqrt{7})}{5}$$
  \end{proof}
\end{thm}

\setcounter{thm}{49}
\begin{thm}
  Find the quotient and remainder using long division
  $$\frac{x^3 + 3x^2 + 4x + 3}{3x + 6}.$$
  \begin{proof}[Solution]
    
  \end{proof}
\end{thm}

\section*{C.1}
\setcounter{thm}{8}
\begin{thm}
  Solve the equation
  $$x - 3 = 2x + 6.$$
  %\begin{proof}[Solution]
  %\end{proof}
\end{thm}

\setcounter{thm}{12}
\begin{thm}
  Solve the equation
  $$2(1 - x) = 3(1 + 2x) + 5.$$
  %\begin{proof}[Solution]
  %\end{proof}
\end{thm}

\setcounter{thm}{22}
\begin{thm}
  Solve the equation
  $$\frac{2}{t} = \frac{3}{5}.$$
  %\begin{proof}[Solution]
  %\end{proof}
\end{thm}

\setcounter{thm}{26}
\begin{thm}
  Solve the equation
  $$\frac{2}{t + 6} = \frac{3}{t - 1}.$$
  %\begin{proof}[Solution]
  %\end{proof}
\end{thm}

\setcounter{thm}{34}
\begin{thm}
  Find all real solutions of the equation
  $$y^2 - 24 = 0.$$
  %\begin{proof}[Solution]
  %\end{proof}
\end{thm}

\setcounter{thm}{40}
\begin{thm}
  Find all real solutions of the equation
  $$(x + 2)^2 = 4.$$
  %\begin{proof}[Solution]
  %\end{proof}
\end{thm}

\setcounter{thm}{60}
\begin{thm}
  Solve the following equation for $x$
  $$xy = 3y - 2x$$
  %\begin{proof}[Solution]
  %\end{proof}
\end{thm}

\section*{C.2}

\setcounter{thm}{6}
\begin{thm}
  Solve the following equation by factoring
  $$3x^2 - 5x - 2 = 0.$$
  %\begin{proof}[Solution]
  %\end{proof}
\end{thm}

\setcounter{thm}{10}
\begin{thm}
  Complete the square for the given expression:
  $$x^2 + 7x + {\color{red} \fbox{\raisebox{5px}{\hspace{5px}}}} = (x + {\color{red} \fbox{\raisebox{5px}{\hspace{5px}}}}\,)^2.$$
  %\begin{proof}[Solution]
  %\end{proof}
\end{thm}

\setcounter{thm}{22}
\begin{thm}
  Solve the following equation by factoring or using the Quadratic Formula
  $$x^2 - 2x - 15 = 0.$$
  %\begin{proof}[Solution]
  %\end{proof}
\end{thm}

\setcounter{thm}{26}
\begin{thm}
  Solve the following equation by factoring or using the Quadratic Formula
  $$x^2 + 3x + 1 = 0.$$
  %\begin{proof}[Solution]
  %\end{proof}
\end{thm}

\section*{C.3}

\setcounter{thm}{0}
\begin{thm}
  Fill in the blank with an appropriate inequality sign.
  \begin{enumerate}[(a)]
  \item
    If $x < 5$, then $x - 3\ \line(1,0){20}\ 2$.
  \item
    If $x \leq 5$, then $3x\ \line(1,0){20}\ 15$.
  \item
    If $x \geq 2$, then $-3x\ \line(1,0){20}\ -6$.
  \item
    If $x < -2$, then $-x\ \line(1,0){20}\ 2$.
  \end{enumerate}
  %\begin{proof}[Solution]
  %\end{proof}
\end{thm}

\setcounter{thm}{9}
\begin{thm}
  Let $S = \{-2, -1, 0, \frac{1}{2}, 1, \sqrt{2}, 2, 4\}$.
  Determine which elements of $S$ satisfy the inequality
  $$x^2 + 2 < 4.$$
  %\begin{proof}[Solution]
  %\end{proof}
\end{thm}

\setcounter{thm}{14}
\begin{thm}
  Solve the linear inequality
  $$7 - x \geq 5.$$
  Express the solution using interval notation, and graph the solution set.
  %\begin{proof}[Solution]
  %\end{proof}
\end{thm}

\setcounter{thm}{16}
\begin{thm}
  Solve the linear inequality
  $$3x + 11 \leq 7x + 8.$$
  Express the solution using interval notation, and graph the solution set.
  %\begin{proof}[Solution]
  %\end{proof}
\end{thm}

\setcounter{thm}{22}
\begin{thm}
  Solve the non-linear inequality
  $$(x + 2)(x - 3) < 0.$$
  Express the solution using interval notation, and graph the solution set.
  %\begin{proof}[Solution]
  %\end{proof}
\end{thm}

\setcounter{thm}{24}
\begin{thm}
  Solve the non-linear inequality
  $$x^2 - 3x - 18 \leq 0.$$
  Express the solution using interval notation, and graph the solution set.  
  %\begin{proof}[Solution]
  %\end{proof}
\end{thm}

\end{document}
