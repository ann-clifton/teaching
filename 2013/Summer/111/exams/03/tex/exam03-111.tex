\documentclass[12pt]{amsart}
\usepackage{amsmath,amsthm,amssymb,amsfonts,enumerate,mymath,tikz-cd,fancyhdr}
\openup 5pt
\author{Blake Farman\\University of South Carolina}
\title{Math 111: Exam 03}
\date{June 17, 2013}
\pdfpagewidth 8.5in
\pdfpageheight 11in
\usepackage[margin=1in]{geometry}

\renewcommand{\qedsymbol}{}

\begin{document}
\maketitle

\begin{center}
  \fbox{\fbox{\parbox{5.5in}{\centering
        Answer the questions in the spaces provided on the
        question sheets and turn them in at the end of the class period. 
        Unless otherwise stated, all supporting work is required.}}}
\end{center}

\vspace{0.2in}
\makebox[\textwidth]{Name:\enspace\hrulefill}
\vspace{0.2in}

\theoremstyle{plain}
\newtheorem{thm}{}
\newtheorem{lem}{Lemma}
\theoremstyle{definition}
\newtheorem{defn}{Definition}

\section{Definitions}
\begin{thm}[4 Points]\label{ex1}
  Let $P$ be a principal investment, $r$ the interest rate, $n$ the number of times interest compounds per year, and $t$ the number of years since the investment.
  State the formula for computing compound interest.
  \vspace{1in}
\end{thm}

\begin{thm}[2 Points]\label{ex2}
  For an exponential function, $f(x)$, state the formula for percentage rate of change.
  \vspace{1 in}
\end{thm}

\begin{thm}[4 Points]\label{ex3}
  Let $a$ be a fixed positive number.
  The base $a$ logarithm of $x$ is defined by
  $$\log_a(x) = y\  \text{if and only if}\ \ \line(1,0){40}.$$
\end{thm}

\newpage

\begin{thm}[4 Points]\label{ex4}
  Let $a$ be a positive number.  Fill in the blanks.
  \begin{enumerate}[(a)]
  \item
    $\log_a(1) = \ \line(1,0){40}$.
  \item
    $\log_a(a) = \ \line(1,0){40}$.
  \item
    $\log_a(a^x) = \ \line(1,0){40}$.
  \item
    $a^{\log_a(x)} = \ \line(1,0){40}$.
  \end{enumerate}
  \vspace{1in}
\end{thm}

\begin{thm}[3 Points]
  Let $a$ and $C$ be fixed positive numbers.  Fill in the blanks.
  \begin{enumerate}[(a)]
  \item
    $\log_a(xy) = \ \line(1,0){80}$.
  \item
    $\log_a\left(\frac{x}{y}\right) = \ \line(1,0){80}$.
  \item
    $\log_a(x^C) = \ \line(1,0){80}$.
  \end{enumerate}
  \vspace{1in}
\end{thm}

\begin{thm}[2 Points]
  Let $a$ and $b$ be fixed positive numbers.
  Use the Change of Base formula to rewrite $\log_a(x)$ with base $b$.
  \vspace{1in}
\end{thm}

\begin{thm}[1 Point]
  State the Horizontal Line Test.
\end{thm}

\newpage
\section{Problems}

\begin{thm}[16 Points]\label{ex5}
  A \$400 investment is made in an account with an annual interest rate of 10\% that compounds semiannually.
  \begin{enumerate}[(a)]
  \item
    Give the formula for the interest accrued as a function of time, $t$. 
    [Hint: If you compute the growth factor without using decimals, this will make the next computation significantly easier.]
    \vspace{1in}
  \item
    Compute the interest accrued after 1 year.
    \vspace{1in}
  \item
    Give the Annual Percentage Yield for the investment.
    \vspace{1in}
  \end{enumerate}
\end{thm}

\begin{thm}[16 Points]\label{ex10}
  Compute the following logarithms.
  \begin{enumerate}[(a)]
  \item
    $\log_{3}(27)$.
    \vspace{.5in}
  \item
    $\log_{3}(81)$.
    \vspace{.5in}
  \item
    $\log_{16}(8)$.
    \vspace{.5in}
  \item
    $\log_{27}(81)$.
    \vspace{.5in}
  \end{enumerate}
\end{thm}

\newpage

\begin{thm}[16 Points]\label{ex9}
  \begin{enumerate}[(a)]
  \item
    Simplify the expression 
    $$2\log_2(\sqrt{x + 2}) - \log_2\left(\frac{1}{x - 2}\right).$$
    \vspace{2in}
  \item
    Solve the following equation for $x$
    $$ 2\log_2(\sqrt{x + 2}) - \log_2\left(\frac{1}{x - 2}\right) = 5 $$
    \vspace{1in}
  \end{enumerate}
\end{thm}

\begin{thm}[16 Points]\label{ex7}
  Solve the following equation for $x$
  $$2^{-4x} = 16 \cdot 2^{x^2}$$
\end{thm}

\newpage

\begin{thm}[16 Points]
  Let $f(x) = \sqrt{1 - x^2}$.  Determine the domain of this function.
  Use this information to carefully justify whether this function is invertible.
  \vspace{4in}
\end{thm}

\begin{thm}[Bonus - 10 Points]\label{bonus}
  Let $f$ be as in the last problem.
  Compute the composition $$f \circ f (x) = f(f(x)).$$
  Determine for which values of $x$ the function $f$ is invertible and, on this set, find its inverse.
\end{thm}
\end{document}
