\documentclass[12pt]{amsart}
\usepackage{amsmath,amsthm,amssymb,amsfonts,enumerate,mymath,tikz-cd,fancyhdr}
\openup 5pt
\author{Blake Farman\\University of South Carolina}
\title{Math 111: Final Exam}
\date{June 21, 2013}
\pdfpagewidth 8.5in
\pdfpageheight 11in
\usepackage[margin=1in]{geometry}

\renewcommand{\qedsymbol}{}

\begin{document}
\maketitle

\begin{center}
  \fbox{\fbox{\parbox{5.5in}{\centering
        Answer the questions in the spaces provided on the
        question sheets and turn them in at the end of the class period. 
        Unless otherwise stated, all supporting work is required.}}}
\end{center}

\vspace{0.2in}
\makebox[\textwidth]{Name:\enspace\hrulefill}
\vspace{0.2in}

\theoremstyle{plain}
\newtheorem{thm}{}
\newtheorem{lem}{Lemma}
\theoremstyle{definition}
\newtheorem{defn}{Definition}

\begin{thm}[10 Points]
  Add the following rational expressions and simplify the result,
  $$\frac{1}{x + \sqrt{3}} + \frac{1}{x - \sqrt{3}}.$$
  \vspace{1in}
\end{thm}

\begin{thm}[10 Points]
  Consider the two lines $f(x) = 5x + 16$ and $g(x) = 8x + 7$.
  Find the point (that is, the $(x,y)$ pair) where these two lines intersect.
  \vspace{2in}
\end{thm}

\newpage

\begin{thm}[10 Points]
  Let $f(x) = 2x^2 - 8x + 4 = 0.$
  \begin{enumerate}[(a)]
  \item
    Put $f(x)$ into standard form.
    \vspace{1in}
  \item
    Solve $f(x) = 0$.
    \vspace{1in}
  \item
    Use the information from parts (a) and (b) to sketch a graph of $f(x)$.
    Label the $y$-intercept, any $x$-intercept(s), and the vertex.
    \vspace{2in}
  \end{enumerate}
\end{thm}

\begin{thm}[10 Points]
  In the following problems, use the given information to find the equation of the line in slope-intercept form.
  \begin{enumerate}[(a)]
  \item
    The line passing through the points $(4,20)$ and $(1,14)$.
    \vspace{1in}
  \item
    The line passing through the point $(6, 12)$ and parallel to the line in part (a).
    \vspace{1in}
  \item
    The line passing through $(4, 8)$ and perpendicular to the line in part (a).
    \vspace{1in}
  \end{enumerate}
\end{thm}

\newpage

\begin{thm}[10 Points]
  A \$640 investment is made in an account with an annual interest rate of 50\% that compounds semiannually.
  \begin{enumerate}[(a)]
  \item
    Give the formula for the balance of the account as a function of time, $t$.
    [Hint: If you compute the growth factor without using decimals, this will make the next computation significantly easier.]
    \vspace{1in}
  \item
    What is the balance of the account after 1 year?
    \vspace{1in}
  \item
    Compute the interest accrued after 1 year.
    \vspace{1in}
  \item
    Give the Annual Percentage Yield for the investment.
    \vspace{1in}
  \end{enumerate}
\end{thm}

\newpage

\begin{thm}[10 Points]
  A biologist observes a population with initial size $81$.
  In two years, the biologist returns to observe the population again and finds that only $9$ remain.
  \begin{enumerate}[(a)]
  \item
    Find an exponential model for the size of the population as a function of $t$ years.
    \vspace{1in}
  \item
    Does the function from part (a) model growth or decay?
    \vspace{1.5in}
  \item
    Use the model form part (a) to determine how many years it will take for the size of the population to reach 1.
    \vspace{1.5in}
  \end{enumerate}
\end{thm}

\begin{thm}[10 Points]
  Compute the following logarithms.
  \begin{enumerate}[(a)]
  \item
    $\log_{64}(16)$.
    \vspace{.5in}
  \item
    $\log_{49}(7)$.
    \vspace{.5in}
  \item
    $\log_8(4)$.
    \vspace{.5in}
  \item
    $\log_{9}(81)$.
    \vspace{.5in}
  \end{enumerate}
\end{thm}

\newpage

\begin{thm}[10 Points]
  \begin{enumerate}[(a)]
  \item
    Simplify the expression 
    $$\log_4(x + 3) + \log_4(x - 3).$$
    \vspace{2in}
  \item
    Solve the following equation for $x$
    $$\log_4(x + 3) + \log_4(x - 3) = 2$$
    \vspace{1in}
  \end{enumerate}
\end{thm}

\begin{thm}[10 Points]
  Solve the following equation for $x$
  $$e^{x^2} = e^{-2x - 1}$$
\end{thm}

\newpage

\begin{thm}[10 Points]
  Let $f(x) = \sqrt{x}$ and $g(x) = x^2 - 9$.
  \begin{enumerate}[(a)]
  \item
    Compute the composition of $f$ with $g$, $(f \circ g)(x)$.
    \vspace{1in}
  \item
    What is the domain of $(f \circ g)(x)$?
    \vspace{1in}
  \item
    Is $(f \circ g)(x)$ invertible?  Explain why or why not.  If it is, give its inverse.
    \vspace{1in}
  \end{enumerate}
\end{thm}
\end{document}
