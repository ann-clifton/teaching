\documentclass[12pt]{amsart}
\usepackage{amsmath,amsthm,amssymb,amsfonts,enumerate,mymath,tikz-cd,fancyhdr}
\openup 5pt
\author{Blake Farman\\University of South Carolina}
\title{Math 111: Exam 03}
\date{June 17, 2013}
\pdfpagewidth 8.5in
\pdfpageheight 11in
\usepackage[margin=1in]{geometry}

\renewcommand{\qedsymbol}{}

\begin{document}
\maketitle

\begin{center}
  \fbox{\fbox{\parbox{5.5in}{\centering
        Answer the questions in the spaces provided on the
        question sheets and turn them in at the end of the class period. 
        Unless otherwise stated, all supporting work is required.}}}
\end{center}

\vspace{0.2in}
\makebox[\textwidth]{Name:\enspace\hrulefill}
\vspace{0.2in}

\theoremstyle{plain}
\newtheorem{thm}{}
\newtheorem{lem}{Lemma}
\theoremstyle{definition}
\newtheorem{defn}{Definition}

\section{Definitions}
\begin{thm}[4 Points]\label{ex1}
  Let $P$ be a principal investment, $r$ the interest rate, $n$ the number of times interest compounds per year, and $t$ the number of years since the investment.
  State the formula for computing compound interest.
  
  \begin{proof}[Solution]
    The formula for compound interest is 
    $$I(t) = P\left(1 + \frac{r}{n}\right)^{nt}.$$
  \end{proof}
\end{thm}

\begin{thm}[2 Points]\label{ex2}
  For an exponential function, $f(x)$, state the formula for percentage rate of change.
  
  \begin{proof}[Solution]
    The formula for percentage rate of change is
    $$r = \frac{f(x + 1) - f(x)}{f(x)}.$$
  \end{proof}
\end{thm}

\begin{thm}[4 Points]\label{ex3}
  Let $a$ be a fixed positive number.
  The base $a$ logarithm of $x$ is defined by
  $$\log_a(x) = y\  \text{if and only if}\ \ \line(1,0){40}.$$
  
  \begin{proof}[Solution]
    The base $a$ logarithm of $x$ is defined by
    $$\log_a(x) = y\  \text{if and only if}\ a^y = x.$$
  \end{proof}
\end{thm}



\begin{thm}[4 Points]\label{ex4}
  Let $a$ be a positive number.  Fill in the blanks.
  \begin{enumerate}[(a)]
  \item
    $\log_a(1) = \ \line(1,0){40}$.
  \item
    $\log_a(a) = \ \line(1,0){40}$.
  \item
    $\log_a(a^x) = \ \line(1,0){40}$.
  \item
    $a^{\log_a(x)} = \ \line(1,0){40}$.
  \end{enumerate}

  \begin{proof}[Solution]
    \begin{enumerate}[(a)]
    \item
      $\log_a(1) = 0$.
    \item
      $\log_a(a) = 1$.
    \item
      $\log_a(a^x) = x$.
    \item
      $a^{\log_a(x)} = x$.
    \end{enumerate}
  \end{proof}  
\end{thm}

\begin{thm}[3 Points]
  Let $a$ and $C$ be fixed positive numbers.  Fill in the blanks.
  \begin{enumerate}[(a)]
  \item
    $\log_a(xy) = \ \line(1,0){80}$.
  \item
    $\log_a\left(\frac{x}{y}\right) = \ \line(1,0){80}$.
  \item
    $\log_a(x^C) = \ \line(1,0){80}$.
  \end{enumerate}

  \begin{proof}[Solution]
    \begin{enumerate}[(a)]
    \item
      $\log_a(xy) = \log_a(x) + \log_a(y)$.
    \item
      $\log_a\left(\frac{x}{y}\right) = \log_a(x) - \log_a(y)$.
    \item
      $\log_a(x^C) = C \log_a(x)$.
    \end{enumerate}
  \end{proof}

\end{thm}

\begin{thm}[2 Points]
  Let $a$ and $b$ be fixed positive numbers.
  Use the Change of Base formula to rewrite $\log_a(x)$ with base $b$.

  \begin{proof}[Solution]
    The Change of Base Formula states
    $$\log_a(x) = \frac{\log_b(x)}{\log_b(a)}.$$
  \end{proof}
\end{thm}

\begin{thm}[1 Point]
  State the Horizontal Line Test.
  
  \begin{proof}[Solution]
    The Horizontal Line Test states that the graph of a function is injective (one-to-one) if and only if any horizontal line passes through the graph in at most one place.
  \end{proof}
\end{thm}


\section{Problems}

\begin{thm}[16 Points]\label{ex5}
  A \$400 investment is made in an account with an annual interest rate of 10\% that compounds semiannually.
  \begin{enumerate}[(a)]
  \item
    Give the formula for the interest accrued as a function of time, $t$. 
    [Hint: If you compute the growth factor without using decimals, this will make the next computation significantly easier.]
  \item
    Compute the interest accrued after 1 year.
  \item
    Give the Annual Percentage Yield for the investment.
  \end{enumerate}
  
  \begin{proof}[Solution]
    \begin{enumerate}[(a)]
    \item
      First compute the growth factor,
      $$a = \left(1 + \frac{10}{200}\right)^2 = \left(\frac{210}{200}\right)^2 = \frac{21^2}{20^2} = \frac{441}{400}.$$
      The formula for the interest accrued as a function of time, $t$, is then
      $$I = 400 \left(\frac{441}{400}\right)^t.$$
    \item
      The interest accrued after 1 year is
      $$I(1) - I(0) = 400 \left(\frac{441}{400}\right)^1 - 400= 400 - 441 = 41.$$
    \item
      Using the formula for the Annual Percentage Yield for the investment we have
      $$\text{APY} = \frac{I(1) - I(0)}{I(0)} = \frac{41}{400}.$$
    \end{enumerate}
  \end{proof}
\end{thm}

\begin{thm}[16 Points]\label{ex10}
  Compute the following logarithms.
  \begin{enumerate}[(a)]
  \item
    $\log_{3}(27)$.
  \item
    $\log_{3}(81)$.
  \item
    $\log_{16}(8)$.
  \item
    $\log_{27}(81)$.
  \end{enumerate}

  \begin{proof}[Solution]
    \begin{enumerate}[(a)]
    \item
      $\log_{3}(27) = \log_3(3^3) = 3$.
    \item
      $\log_{3}(81) = \log_3(3^4) = 4$.
    \item
      Using the Change of Base formula we have
      $$\log_{16}(8) = \frac{\log_2(8)}{\log_2(16)} = \frac{3}{4}.$$
    \item
      Using the Change of Base formula we have
      $$\log_{27}(81) = \frac{\log_3(81)}{\log_3(27)} = \frac{4}{3}.$$
    \end{enumerate}
  \end{proof}
\end{thm}

\begin{thm}[16 Points]\label{ex9}
  \begin{enumerate}[(a)]
  \item
    Simplify the expression 
    $$2\log_2(\sqrt{x + 2}) - \log_2\left(\frac{1}{x - 2}\right).$$
  \item
    Solve the following equation for $x$
    $$ 2\log_2(\sqrt{x + 2}) - \log_2\left(\frac{1}{x - 2}\right) = 5 $$
  \end{enumerate}
  
  \begin{proof}[Solution]
    \begin{enumerate}[(a)]
    \item
      The expression simplifies to 
      \begin{eqnarray*}
        2\log_2(\sqrt{x + 2}) - \log_2\left(\frac{1}{x - 2}\right) &=& \log_2((\sqrt{x + 2})^2) - (\log_2(1) -\log_2(x - 2))\\
        &=& \log_2(x + 2) - (0 -  \log_2(x - 2))\\
        &=& \log_2(x + 2) + \log_2(x - 2)\\
        &=& \log_2((x + 2)(x - 2))\\
        &=& \log_2(x^2 - 4).
      \end{eqnarray*}
    \item
      Using the result of part (a) we have
      $$2\log_2(\sqrt{x + 2}) - \log_2\left(\frac{1}{x - 2}\right) = \log_2(x^2 - 4) = 5.$$
      Then it follows that
      $$2^{\log_2(x^2 - 4)} = x^2 - 4 = 2^5 = 32.$$
      Adding $4$ to both sides we have $x^2 = 36$ and so $x = \pm 6$.
      Since $-6 + 2 < 0$ and $-6 - 2 < 0$, it follows $-6$ is not in the domain of either $\log_2(\sqrt{x + 2})$ or $\log_2\left(\frac{1}{x -2}\right)$ and so the only solution is $x = 6$.
    \end{enumerate}
  \end{proof}
\end{thm}

\begin{thm}[16 Points]\label{ex7}
  Solve the following equation for $x$
  $$2^{-4x} = 16 \cdot 2^{x^2}$$
  
  \begin{proof}[Solution]
    Multiplying both sides of the equation by $2^{4x}$ and dividing by $16$ we have
    $$2^{x^2} \cdot 2^{4x} = 2^{x^2 + 4x} = \frac{1}{16}.$$
    Taking the base $2$ logarithm of both sides we have
    $$x^2 + 4x = -4$$
    which gives $x^2 + 4x + 4 = (x + 2)^2 = 0$.
    Therefore the only solution is $x = -2$.
  \end{proof}
\end{thm}

\begin{thm}[16 Points]
  Let $f(x) = \sqrt{1 - x^2}$.  Determine the domain of this function.
  Use this information to carefully justify whether this function is invertible.

  \begin{proof}[Solution]
    The domain of this function are all the values of $x$ satisfying $1 - x^2 = (1 + x)(1 - x) \geq 0$.
    Checking the points $x = -2$, $x = 0$, and $x = 2$ we have
    $(1 - 2)(1 + 2) = -3 < 0$, $(1 + 0)(1 - 0) = 1 > 0$, and $(1 + 2)(1 - 2) = -1 < 0$.
    Hence the domain is the set $[-1, 1]$.
    Since $f(1) = f(-1) = 0$, this function fails the Horizontal Line Test for the line $y = 0$ and so it is not injective.
    Therefore $f$ is not invertible.
  \end{proof}
\end{thm}

\begin{thm}[Bonus - 10 Points]\label{bonus}
  Let $f(x)$ be as in the last problem.
  Compute the composition $$f \circ f (x) = f(f(x)).$$
  Determine for which values of $x$ the function $f$ is invertible and, on this set, find its inverse.
  
  \begin{proof}[Solution]
    By composing the function $f$ with itself we get
    $$f(f(x)) = \sqrt{1 - (\sqrt{1 - x^2})^2} = \sqrt{1 - (1 - x^2)} = \sqrt{x^2}.$$
    Note that by squaring $x$ and then taking the square root, we are always getting the absolute value of $x$.
    That is if $0 \leq x \leq 1$, 
    $$f(f(-x)) = \sqrt{(-x)^2} = \sqrt{x^2} = x\ \text{and}\ f(f(x)) = \sqrt{(x)^2} = \sqrt{x^2} = x.$$
    This tells us that the function $f(x)$ with domain $[0,1]$ is its own inverse.
    %The second is that we observe for all numbers $-1 \leq x \leq 0$, $f(f(x)) \neq x$ and so $f$ is not the inverse of the function $f$ defined on $[-1,0]$. 
    %However, not all is lost; the problem is rectified by using $-f(t)$.  
    %Namely, if $0 \leq x \leq 1$, then 
    %$$-f(f(t)) = $
  \end{proof}
\end{thm}
\end{document}
