\documentclass[12pt]{amsart}
\usepackage{amsmath,amsthm,amssymb,amsfonts,enumerate,mymath,tikz-cd,fancyhdr}
\openup 5pt
\author{Blake Farman\\University of South Carolina}
\title{Math 111: Exam 01}
\date{May 24, 2013}
\pdfpagewidth 8.5in
\pdfpageheight 11in
\usepackage[margin=1in]{geometry}

\renewcommand{\qedsymbol}{}

\begin{document}
\maketitle

\begin{center}
\fbox{\fbox{\parbox{5.5in}{\centering
Answer the questions in the spaces provided on the
question sheets and turn them in at the end of the class period. 
Unless otherwise stated, all supporting work is required.}}}
\end{center}

\vspace{0.2in}
\makebox[\textwidth]{Name:\enspace\hrulefill}
\vspace{0.2in}

\theoremstyle{plain}
\newtheorem{thm}{}
\newtheorem{lem}{Lemma}
\theoremstyle{definition}
\newtheorem{defn}{Definition}

\section{Definitions}
\begin{thm}[3 Points]\label{ex1}
  State the three Special Product Formulas for algebraic expressions $A$ and $B$:
  \begin{enumerate}[(a)]
  \item
    Sum and difference of same terms:
  \item
    Square of a sum:
  \item
    Square of a difference:
  \end{enumerate}
  
  \begin{proof}[Solution]
    \begin{enumerate}[(a)]
    \item
      The sum and difference of same terms is given by
      $$(A + B)(A - B) = A^2 - B^2.$$
    \item
      The square of a sum is given by
      $$(A + B)^2 = A^2 + 2AB + B^2.$$
    \item
      The square of a difference is given by
      $$(A - B)^2 = A^2 - 2AB + B^2.$$
    \end{enumerate}
  \end{proof}
\end{thm}

\begin{thm}[6 Points]\label{ex2}
  Let $a, b$ be non-zero real numbers and $m, n$ integers.
  Fill in the blanks
  \begin{enumerate}[(i)]
  \item
    $\displaystyle{a^0 =\ \line(1,0){40}}$,
  \item
    $\displaystyle{a^{-n} =\ \line(1,0){40}}.$
  \item
    $\displaystyle{a^m \cdot a^n =\ \line(1,0){40}}$
  \item
    $\displaystyle{\frac{a^m}{a^n}\ \line(1,0){40}}$
  \item
    $\displaystyle{\left(a \cdot b\right)^n\ \line(1,0){40}}$
  \item
    $\displaystyle{\left(\frac{a}{b}\right)^n\ \line(1,0){40}}$
  \end{enumerate}
  
  \begin{proof}[Solution]
    The identities are as follows
    \begin{enumerate}[(i)]
    \item
      $\displaystyle{a^0 = 1}$,
    \item
      $\displaystyle{a^{-n} = \frac{1}{a^n}}.$
    \item
      $\displaystyle{a^m \cdot a^n = a^{m + n}}.$
    \item
      $\displaystyle{\frac{a^m}{a^n} = a^{m - n}}.$
    \item
      $\displaystyle{\left( a \cdot b \right)^n = a^nb^n}.$
    \item
      $\displaystyle{\left( \frac{a}{b} \right)^n = \frac{a^n}{b^n}}.$
    \end{enumerate}
  \end{proof}
\end{thm}



\begin{thm}[2 Points]\label{ex3}
  State the general form of a quadratic equation and the Quadratic Formula.
  
  \begin{proof}[Solution]
    The general form of a quadratic equation is $ax^2 + bx + c = 0$.
    The Quadratic Formula, which gives the roots of these equations, is
    $$x = \frac{-b \pm \sqrt{b^2 - 4ac}}{2a}.$$
  \end{proof}
\end{thm}

\begin{thm}[3 Points]\label{ex4}
  Fill in the blanks:\\
  \begin{center}
    To make $x^2 + bx$ a perfect square, add $\fbox{\raisebox{10px}{\hspace{10px}}}$\,.
    This gives the perfect square
    $$x^2 + bx + \fbox{\raisebox{10px}{\hspace{10px}}} = \left(x + \fbox{\raisebox{10px}{\hspace{10px}}}\,\right)^2.$$
  \end{center}

  \begin{proof}[Solution]
    To make $x^2 + bx$ a perfect square, add $\displaystyle{\left(\frac{b}{2}\right)^2}$.
    This gives the perfect square
    $$x^2 + bx + \left(\frac{b}{2}\right)^2 = \left(x + \frac{b}{2}\right)^2.$$
  \end{proof}
\end{thm}

\begin{thm}[3 Points]\label{ex5}
  Fill in the blanks:\\
  \begin{center}
    A variable $y$ is a function of a variable $x$ if each value of\ \line(1,0){40}\ corresponds to exactly\ \line(1,0){40}\ value of\ \line(1,0){40}.
  \end{center}

  \begin{proof}[Solution]
    A variable $y$ is a function of a variable $x$ if each value of $x$ corresponds to exactly one value of $y$.
  \end{proof}
\end{thm}

\begin{thm}[3 Points]\label{ex6}
  Expand the following product $(a + b)(c + d)$
  \begin{enumerate}[(a)]
    \item
      using the distributive method.
    \item
      using FOIL.
  \end{enumerate}

  \begin{proof}[Solution]
    \begin{enumerate}[(a)]
    \item
      Using the distributive method from left to right, we have 
      $$(a+ b)(c + d) = (a + b)c + (a + b)d = ac + bc + ad + bc.$$
      Similarly, from right to left,
      $$(a + b)(c + d) = a(c + d) + b(c + d) = ac + ad + bc + bd.$$
    \item
      Using the FOIL method we have
      $$(a + b)(c + d) = ac + ad + bc + bd.$$
    \end{enumerate}
  \end{proof}
\end{thm}


\section{Exercises}
\begin{thm}[20 Points]\label{ex7}
  Consider the equation
  $$x^2 + 3y = 9.$$
  \begin{enumerate}[(a)]
    \item
      Does this equation define $y$ as function of $x$?  
      {\it Briefly} justify why or why not.
      If it does, give the value of $y$ when $x = 3$.
    \item
      Does this equation define $x$ as function of $y$?  {\it Briefly} justify why or why not.
      If it does, give the value of $x$ when $y = 5/3$.
  \end{enumerate}
  
  \begin{proof}[Solution]
    \begin{enumerate}[(a)]
    \item
      Yes, this equation does define $y$ as a function of $x$.
      To see this, solve for $y$ in terms of $x$ to get
      $$y = \frac{9 - x^2}{3},$$
      and so we see that for every input $x$, there is exactly one output value $y$.

      When $x = 3$, we have 
      $$y = \frac{9 - (3)^2}{3} = \frac{9 - 9}{3} = 0.$$
    \item
      This equation does {\it not} define $x$ as a function of $y$.
      Subtracting $3y$ from both sides of the equation gives us
      $$x^2 = 9 - 3y.$$
      To solve for $x$, we take the square root of both sides to obtain
      $$x = \pm\sqrt{9 - 3y}.$$
      This means that for each value of $y$, there are {\it two} values of $x$ and hence this is not a function.
    \end{enumerate}
  \end{proof}
\end{thm}

\begin{thm}[20 Points]\label{ex8}
  Add the following rational expressions and simplify the result,
  $$\frac{2}{x^2 - 1} + \frac{1}{x+1}.$$
  
  \begin{proof}[Solution]
    We first observe that as the difference of two squares, we have the factorization $x^2 - 1 = (x + 1)(x -1)$.
    Hence we find a common denominator by multiplying the numerator and denominator of the second term by $x - 1$,
    \begin{eqnarray*}
      \frac{2}{x^2 - 1} + \frac{1}{x+1} &=& \frac{2}{(x + 1)(x - 1)} + \frac{x - 1}{x - 1} \cdot \left(\frac{1}{x+1}\right)\\
      &=& \frac{2 + (x - 1)}{(x+1)(x-1)}\\
      &=& \frac{x + 1}{(x + 1)(x - 1)}\\
      &=& \frac{1}{x - 1}.      
    \end{eqnarray*}
  \end{proof}
\end{thm}

\begin{thm}[20 Points]\label{ex9}
  Solve the equation
  $$2x^2 - 8x + 4 = 0.$$

  \begin{proof}[Solution]
    We can solve this equation using the Quadratic Formula.
    Namely,
    \begin{eqnarray*}
      x &=& \frac{8 \pm \sqrt{(-8)^2 - 4(2)(4)}}{2(2)}\\
      &=& \frac{8 \pm \sqrt{64 - 32}}{4}\\
      &=& \frac{8 \pm \sqrt{32}}{4}\\
      &=& \frac{8 \pm \sqrt{2^5}}{4}\\
      &=& \frac{8 \pm 4\sqrt{2}}{4}\\
      &=& \frac{4(2 \pm \sqrt{2})}{4}\\
      &=& 2 \pm \sqrt{2}.
    \end{eqnarray*}

    Alternatively, this can be done by completing the square.
    First we divide both sides of the equation by $2$ to obtain $x^2 - 4x + 2 = 0.$
    Since the signs alternate through the expression, the best match is to turn this equation into a square of a difference, which has the form $(A - B)^2 = A^2 - 2AB + B^2$.
    Subtracting $2$ from both sides, we have 
    $$x^2 - 4x = -2,$$
    we see that if we add $\displaystyle{\left(\frac{4}{2}\right)^2} = (2)^2 = 4$ to both sides, we have
    $$x^2 - 4x + 4 = (x - 2)^2 = 2.$$
    Taking the square root of both sides we have 
    $$x - 2 = \pm \sqrt{2}$$
    and then adding $2$ to boths sides gives us our final answer,
    $$x = 2 \pm \sqrt{2}$$.
  \end{proof}
\end{thm}

\begin{thm}[20 Points]\label{ex10}
  Solve the inequality $$x^2 \geq 9.$$
  Express the solution using interval notation and graph the solution.
  
  \begin{proof}[Solution]
    By first subtracting $9$ from both sides of the inequality, we have the inequality
    $$x^2 - 9 = (x - 3)(x + 3) \geq 0.$$
    Hence when $x = 3$ or $x = -3$, the inequality is satisifed.
    It remains only to check points between $-3$ and $3$, to the left of $-3$, and to the right of $3$.

    For the point between $-3$ and $3$, we choose $x = 0$ and see that
    $$0^2 - 9 = -9 < 0,$$
    so the inequality is {\it not} satisified in the region $(-3,3)$.
    For the point to the left of $-3$, we choose $x = -4$ and get
    $$(-4)^2 - 9 = 16 - 9 = 7 \geq 0,$$
    which shows us that the inequality is satisified to the left of $-3$.
    Similarly, for $x = 4$ we have
    $$(4)^2 - 9 = 16 - 9 = 7 \geq 0,$$
    which shows us that the inequality is satisfied to the right of $3$.
    Therefore our solutions lie in the set
    $$(-\infty, -3] \cup [3, \infty).$$
  \end{proof}
\end{thm}

\begin{thm}[Bonus - 10 Points]\label{bonus}
  Derive the Quadratic Formula, as stated in Problem~\ref{ex3}.
  [Hint: Use Problem~\ref{ex4}].
  
  \begin{proof}[Solution]
    To derive the Quadratic Formula, we use the equation $ax^2 + bx + c = 0$ and completing the square.
    First, divide both sides of the equation by $a$ to obtain
    $$x^2 + \frac{b}{a}x + \frac{c}{a} = 0$$
    and then subtract $\frac{c}{a}$ from both sides, which yields
    $$x^2 + \frac{b}{a}x = -\frac{c}{a}.$$
    We now complete the square by adding to both sides the quantity $\displaystyle{\left(\frac{b}{2a}\right)^2},$
    $$x^2 + \frac{b}{a}x + \left(\frac{b}{2a}\right)^2 = \left(x + \left(\frac{b}{2a}\right) \right)^2 = \left(\frac{b}{2a}\right)^2 -\frac{c}{a} = \frac{b^2}{4a^2} - \frac{c}{a}.$$
    Using the common denominator $4a^2$ on the right-hand side, we have
    $$\left(x + \left(\frac{b}{2a}\right) \right)^2 = \frac{b^2}{4a^2} - \frac{4a}{4a} \cdot \left(\frac{c}{a}\right) = \frac{b^2 - 4ac}{4a^2}.$$
    Next we take the square root of both sides to obtain
    $$x + \frac{b}{2a} = \pm \sqrt{\frac{b^2 - 4ac}{4a^2}} = \pm \frac{\sqrt{b^2 - 4ac}}{\sqrt{4a^2}} = \pm\frac{\sqrt{b^2 - 4ac}}{2a}.$$
    Finally, subtracting $\frac{b}{2a}$ from both sides gives the familiar equation
    $$x = \frac{-b \pm \sqrt{b^2 - 4ac}}{2a}.$$
  \end{proof}
\end{thm}
\end{document}
