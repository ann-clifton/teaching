\documentclass[12pt]{amsart}
\usepackage{amsmath,amsthm,amssymb,amsfonts,enumerate,mymath,tikz-cd,fancyhdr}
\openup 5pt
\author{Blake Farman\\University of South Carolina}
\title{Math 111: Exam 01}
\date{May 24, 2013}
\pdfpagewidth 8.5in
\pdfpageheight 11in
\usepackage[margin=1in]{geometry}

\renewcommand{\qedsymbol}{}

\begin{document}
\maketitle

\begin{center}
\fbox{\fbox{\parbox{5.5in}{\centering
Answer the questions in the spaces provided on the
question sheets and turn them in at the end of the class period. 
Unless otherwise stated, all supporting work is required.}}}
\end{center}

\vspace{0.2in}
\makebox[\textwidth]{Name:\enspace\hrulefill}
\vspace{0.2in}

\theoremstyle{plain}
\newtheorem{thm}{}
\newtheorem{lem}{Lemma}
\theoremstyle{definition}
\newtheorem{defn}{Definition}

\section{Definitions}
\begin{thm}[3 Points]\label{ex1}
  State the three Special Product Formulas for algebraic expressions $A$ and $B$:
  \begin{enumerate}[(a)]
  \item
    Sum and difference of same terms:
    \vspace{.3in}
  \item
    Square of a sum:
    \vspace{.3in}
  \item
    Square of a difference:
    \vspace{.3in}
  \end{enumerate}
\end{thm}

\begin{thm}[6 Points]\label{ex2}
  Let $a, b$ be non-zero real numbers and $m, n$ integers.
  Fill in the blanks
  \begin{enumerate}[(i)]
  \item
    $\displaystyle{a^0 =\ \line(1,0){40}}$,
  \item
    $\displaystyle{a^{-n} =\ \line(1,0){40}}.$
  \item
    $\displaystyle{a^m \cdot a^n =\ \line(1,0){40}}$
  \item
    $\displaystyle{\frac{a^m}{a^n}\ \line(1,0){40}}$
  \item
    $\displaystyle{\left(a \cdot b\right)^n\ \line(1,0){40}}$
  \item
    $\displaystyle{\left(\frac{a}{b}\right)^n\ \line(1,0){40}}$
  \end{enumerate}
\end{thm}

\newpage

\begin{thm}[2 Points]\label{ex3}
  State the general form of a quadratic equation and the Quadratic Formula.
  \vspace{1in}
\end{thm}

\begin{thm}[3 Points]\label{ex4}
  Fill in the blanks:\\
  \begin{center}
    To make $x^2 + bx$ a perfect square, add $\fbox{\raisebox{10px}{\hspace{10px}}}$\,.
    This gives the perfect square
    $$x^2 + bx + \fbox{\raisebox{10px}{\hspace{10px}}} = (x + \fbox{\raisebox{10px}{\hspace{10px}}}\,)^2.$$
  \end{center}
\end{thm}

\begin{thm}[3 Points]\label{ex5}
  Fill in the blanks:\\
  \begin{center}
    A variable $y$ is a function of a variable $x$ if each value of\ \line(1,0){40}\ corresponds to exactly\ \line(1,0){40}\ value of\ \line(1,0){40}.
  \end{center}
  \vspace{1in}
\end{thm}

\begin{thm}[3 Points]\label{ex6}
  Expand the following product $(a + b)(c + d)$
  \begin{enumerate}[(a)]
    \item
      using the distributive method.
      \vspace{1in}
    \item
      using FOIL.
      \vspace{1in}
  \end{enumerate}
\end{thm}

\newpage
\section{Exercises}
\begin{thm}[20 Points]\label{ex7}
  Consider the equation
  $$x^2 + 3y = 9.$$
  \begin{enumerate}[(a)]
    \item
      Does this equation define $y$ as function of $x$?  
      {\it Briefly} justify why or why not.
      If it does, give the value of $y$ when $x = 3$.
      \vspace{1in}
    \item
      Does this equation define $x$ as function of $y$?  {\it Briefly} justify why or why not.
      If it does, give the value of $x$ when $y = 5/3$.
      \vspace{1in}
  \end{enumerate}
\end{thm}

\begin{thm}[20 Points]\label{ex8}
  Add the following rational expressions and simplify the result,
  $$\frac{2}{x^2 - 1} + \frac{1}{x+1}.$$
\end{thm}

\newpage

\begin{thm}[20 Points]\label{ex9}
  Solve the equation
  $$2x^2 - 8x + 4 = 0.$$
  \vspace{2in}
\end{thm}

\begin{thm}[20 Points]\label{ex10}
  Solve the inequality $$x^2 \geq 9.$$
  Express the solution using interval notation and graph the solution.
  \vspace{2in}
\end{thm}

\newpage
\begin{thm}[Bonus - 10 Points]\label{bonus}
  Derive the Quadratic Formula, as stated in Exercise~\ref{ex3}.
  [Hint: Use Exercise~\ref{ex4}].
\end{thm}
\end{document}
