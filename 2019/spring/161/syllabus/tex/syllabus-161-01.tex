\documentclass[12pt]{amsart}
\usepackage{amsmath,amsthm,amssymb,amsfonts,enumerate,hyperref,amsaddr,longtable}
\openup 5pt

\def\semester{Spring}
\def\course{161}
\def\courseSection{01}
\def\school{Lafayette College}
\def\year{2019}
\def\email{farmanb@lafayette.edu}
\def\building{Pardee Hall}
\def\officeNumber{225A}
\def\courseWebsite{https://sites.lafayette.edu/farmanb/teaching/math-161-s19/}

\author[Blake Farman]{Blake Farman}
\title[Syllabus]{Syllabus\\Math \course-\courseSection \\ \semester\ \year}
\address{Lafayette College}
\date{January 28, \year}
\pdfpagewidth 8.5in
\pdfpageheight 11in
\usepackage[margin=1in]{geometry}

\begin{document}
\maketitle

\section*{Contact Information}
\noindent
\textbf{E-mail:} \href{mailto:farmanb@lafayette.edu}{farmanb@lafayette.edu}\\
\textbf{Office:} \building\ room \officeNumber\\
\textbf{Phone:} (610) 330-5906\\
\textbf{Office Hours:} Monday/Wednesday, 12:30 pm - 2:00 pm.\\
Additionally, I am available by appointment if these times are not suitable.

\section*{Course Information}
\noindent
\textbf{Lectures:}
Monday/Wednesday/Friday,  9:15 am - 10:45 am in Pardee Hall, room 217.\\

\noindent\textbf{Pre-Requisites:} High school trigonometry.\\

\noindent\textbf{Course Objectives:}
Math 161 (Calculus I) provides an introduction to calculus for students of mathematics, engineering, and the sciences. Topics include limits, derivatives, techniques of differentiation, definite integrals, the fundamental theorem of calculus, and applications of derivatives and integrals.\\

\noindent\textbf{Learnings Outcomes:}
The calculus sequence will help students
\begin{itemize}
\item
  perform fundamental computational techniques of calculus,
\item
  understand the basic concepts and vocabulary of calculus,
\item
  learn to use symbolic, graphical, and numerical methods in an integrated way to investigate and solve problems in various contexts,
\item
  learn to formulate problems in mathematical terms, and
\item
  develop their ability to learn mathematics.
\end{itemize}
\noindent\textbf{Course Website:} The URL for the course website is
\begin{center}
  \url{\courseWebsite}
\end{center}
Here you can find a digital copy of the syllabus and other important information.\\
%, lecture notes, test dates, and other various announcements.\\

\noindent\textbf{Text:}
The required text for this course is
\begin{center}
  {\it Calculus}, $8^{\text{th}}$ Edition, James Stewart, 2016. ISBN 978-1-285-74062-1.
\end{center}

\noindent
However you choose to obtain a copy, be aware that it is {\bf expected} that you will read the text outside of lecture.
In particular, it is highly suggested that you take some time to read the section to be covered ahead of lecture.
In making your choice, be sure that you choose an option that you will read.

\section*{Assessments}
\subsection*{Homework}

\noindent
Homework will be assigned and collected regularly.
The problems are chosen to highlight the core concepts from each section.
Mastery of these homework sets serves as a good indicator for exam performance.
As such, you should ensure that you fully understand the material on these homework sets; that is, upon completion of the homework set, you should be capable of completing similar problems without the aid of the text, a computer, a calculator, or any other tools not available during an exam.

Late work will {\bf not} be accepted, and you are solely responsible for ensuring that these assignments are completed on time.
Do {\bf not} leave these assignments until the last minute.

\subsection*{Quizzes:}

\noindent
There will be regular quizzes given during class.
Problems appearing on the quizzes will be selected from the homework problems, and will be graded as though they were on an exam.
As such, these quizzes should be considered as practice for the in-class examinations.

\subsection*{Exams:}
\noindent There will be three in-class exams and a cumulative final exam.
The in-class exams are tentatively scheduled as follows:
\begin{center}
  \begin{tabular}{ll}
    Exam 1: & Wednesday, February 20, 2019,\\
    Exam 2: & Monday, April 1, 2019, and\\
    Exam 3: & Monday, April 22, 2019.\\
  \end{tabular}\\
\end{center}
Though unexpected, any deviations from this schedule will be announced during lecture and reflected on the course website.

\subsection*{Missed Assessments:}
There will {\bf not} be any make-up exams or quizzes.
If you miss one exam, your final exam grade will replace the missing exam grade.
\textbf{This policy is intended only for exams missed due to illness, injury, etc.  
  It does NOT mean that your lowest exam grade will be dropped.}
Any further missed exams will receive a grade of zero.

\section*{Grading}
\noindent\textbf{Scale:}
Grades will be assigned on the following scale:
\begin{center}
  \begin{tabular}{rlrlrl}
    A-: & 90-92\% & A: &93-100\%\\
    B-: & 80-82\% & B: & 83-86\% & B+: &87-89\%\\
    C-: & 70-72\% & C: & 73-76\% & C+: &77-79\%\\
    D-: & 60-62\% & D: & 63-66\% & D+: &67-69\%\\
    F: & $<$ 60\%.\\
  \end{tabular}\\
\end{center}
\textbf{Weights:}
Final grades will be calculated with the following weights:
\begin{center}
  \begin{tabular}{rl}
    Homework: & 10\%,\\
    Quizzes: &15\%,\\
    Exams: & 45\%,\\
    Final Exam: & 30\%.\\
  \end{tabular}\\
\end{center}

\section*{Academic Support}
\subsection*{Calculus Cavalry}
\noindent
Open peer tutoring is available in Pardee Hall room 218 beginning on September 3 with the following hours:
\begin{center}
  \begin{tabular}{rl}
    Mondays: &7-9 pm,\\
    Tuesdays: &4-6 pm,\\
    Wednesdays: &7-9 pm,\\
    Thursdays: &4-6 pm and 7-9 pm,\\
    Sundays: &4-6 pm.
  \end{tabular}
\end{center}

\subsection*{Academic Resource Hub/ATTIC}
\noindent
The Academic Resource Hub (formerly ATTIC) provides academic services to enhance student success and is located on the third floor of Scott Hall.

Resources available to students include:
\begin{itemize}
\item 
  Tutoring and Supplemental Instruction
\item
  Academic Enrichment Resources
\item
  Accessibility Services
\item
  Services for Varsity Student Athletes
\end{itemize}
For more information, see the website at \url{http://attic.lafayette.edu}.

\subsection*{Disability statement:}
In compliance with Lafayette College policy and equal access laws, I am available to discuss appropriate academic accommodations that you may require as a student with a disability.  Requests for academic accommodations need to be made during the first two weeks of the semester, except for unusual circumstances, so arrangements can be made.  Students must register with the Office of the Dean of Advising and Co-Curricular Programs for disability verification and for determination of reasonable academic accommodations.


\section*{Expectations}
\noindent
\subsection*{Academic Integrity:}
To maintain the scholarly standards of the College and, equally important, the personal ethical standards of our students, it is essential that written assignments be a student’s own work, just as is expected in examinations and class participation. A student who commits academic dishonesty is subject to a range of penalties, including suspension or expulsion. Finally, the underlying principle is one of intellectual honesty. If a person is to have self-respect and the respect of others, all work must be his/her own.

Any student found responsible of academic dishonesty \textbf{will receive a grade of F in the course} and disciplinary action according to the procedure outlined in \href{https://conduct.lafayette.edu/wp-content/uploads/sites/93/2018/08/StudentHandboook-2018-19.pdf}{Student Handbook}.

\subsection*{Attendance:}
Lecture is the longest stretch of time each week in which you have access to an interactive learning resource (i.e. me).
As such, lecture is arguably the most valuable aspect of the course and you are expected to not only attend class, but to also actively engage with the material (e.g. ask questions, contribute answers, etc.).
Cell phones and other distractions should either be left at home or be silenced and remain stored your bag.
If you find yourself unable to attend the lecture, please contact me in advance, if possible, to see what you will miss.

\section*{Federal Credit Hour Requirement}
The student work in this course is in full compliance with the federal definition of a four credit hour course.
Please see the Registrar's Office web site

\begin{center}
  \url{http://registrar.lafayette.edu/additional-resources/cep-course-proposal}
\end{center}
for the full policy and practice statement.

\section*{Schedule}
Below is a tentative schedule for the course.
\begin{center}
  \begin{longtable}{lll}
    Date & Section(s) & Material\\
    \hline
    1/28 & 6.6 & Inverse trigonometric functions\\
    1/30 & 6.6, 6.8 & Indeterminate forms and L'H\^opital's Rule\\
    2/1 & 6.8\\
    2/4 & 5.1, 5.2 & Areas and Volumes\\
    2/6 & 5.1, 5.2\\
    2/8 & 7.1 & Integration by Parts\\
    2/11 & 7.4 & Partial fractions\\
    2/13 & 7.5 & Strategy for Integrations\\
    2/15 & 7.7 & Approximate integration\\
    2/18 & 7.7 & Using error bounds\\
    2/20 & 7.8 & Improper integrals\\
    2/22 & & Review for Exam 1\\
    2/25 & & Exam 1\\
    \hline\\
    2/27 & 9.1 & Introduction to differential equations\\ 
    3/1 & 9.2 & Direction fields and Euler's method\\
    3/4 & 9.3 & Separable differential equations\\
    3/6 & 6.5 & Exponential growth and decay\\
    3/8 & 10.1 & Parametric equations\\
    3/11 & 10.2 & Tangents and arclength\\
    3/13 & 10.3 & Polar coordinates\\
    3/15 & 10.4 & Areas in polar coordinates\\
    3/18 - 3/22 & & Spring Break\\
    3/25 & & Review for Exam 2\\
    3/27 & & Exam 2\\
    \hline\\
    3/29 & 11.1 & Sequences\\
    4/1 & 11.2 & Series\\ 
    4/3 & 11.3 & The integral test\\
    4/5 & 11.4 & The comparison test\\
    4/8 & 11.5 & Alternating series\\ 
    4/10 & 11.6 & Ratio and root tests\\ 
    4/12 & 11.7 & Strategy for testing series\\
    4/15 & 11.8 & Power series\\
    4/17 & 11.9 & Representing functions\\
    4/19 & 11.10 & Taylor and Maclaurin series\\
    4/22 & 11.10\\
    4/24 & 11.11 & Applications of Taylor polynomials\\
    4/27 & 11.11 \\
    4/29 & & Review for Exam 3\\
    5/1  & & Exam 3\\
    \hline
    5/3 & & TBA\\
    5/6 & & TBA\\
    5/8 & & Final Exam Review\\
    5/10 & & Final Exam Review\\
  \end{longtable}
\end{center}

\end{document}
