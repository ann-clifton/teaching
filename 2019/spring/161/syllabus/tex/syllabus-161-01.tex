\documentclass[12pt]{amsart}
\usepackage{amsmath,amsthm,amssymb,amsfonts,enumerate,hyperref,amsaddr,longtable,multicol}
\usepackage[all]{nowidow}
\openup 5pt

\def\semester{Spring}
\def\course{161}
\def\courseSection{01}
\def\school{Lafayette College}
\def\year{2019}
\def\email{farmanb@lafayette.edu}
\def\building{Pardee Hall}
\def\officeNumber{225A}
\def\courseWebsite{https://sites.lafayette.edu/farmanb/teaching/math-161-s19/}

\author[Blake Farman]{Blake Farman}
\title[Syllabus]{Syllabus\\Math \course-\courseSection \\ \semester\ \year}
\address{Lafayette College}
\date{January 28, \year}
\pdfpagewidth 8.5in
\pdfpageheight 11in
\usepackage[margin=1in]{geometry}

\begin{document}
\maketitle

\section*{Contact Information}
\noindent
\textbf{E-mail:} \href{mailto:farmanb@lafayette.edu}{farmanb@lafayette.edu}\\
\textbf{Office:} \building\ room \officeNumber\\
\textbf{Phone:} (610) 330-5906\\
\textbf{Office Hours:} Monday/Wednesday, 12:30 pm - 2:00 pm.\\
Additionally, I am available by appointment if these times are not suitable.

\section*{Course Information}
\subsection*{Lectures}
Monday/Wednesday/Friday,  9:15 am - 10:45 am in Pardee Hall, room 217.\\

\subsection*{Pre-Requisites} High school trigonometry.\\

\subsection*{Course Objectives}
Math 161 (Calculus I) provides an introduction to calculus for students of mathematics, engineering, and the sciences. Topics include limits, derivatives, techniques of differentiation, definite integrals, the fundamental theorem of calculus, and applications of derivatives and integrals.\\

\subsection*{Learning Outcomes}
The calculus sequence will help students
\begin{itemize}
\item
  perform fundamental computational techniques of calculus,
\item
  understand the basic concepts and vocabulary of calculus,
\item
  learn to use symbolic, graphical, and numerical methods in an integrated way to investigate and solve problems in various contexts,
\item
  learn to formulate problems in mathematical terms, and
\item
  develop their ability to learn mathematics.
\end{itemize}

\subsection*{Text}
The required text for this course is
\begin{center}
  {\it Calculus}, $8^{\text{th}}$ Edition, James Stewart, 2016. ISBN 978-1-285-74062-1.
\end{center}

\noindent
However you choose to obtain a copy, be aware that it is {\bf expected} that you will read the text outside of lecture.
In particular, it is highly suggested that you take some time to read the section to be covered ahead of lecture.
In making your choice, be sure that you choose an option that you will read.

\subsection*{Course Website} The URL for the course website is
\begin{center}
  \url{\courseWebsite}
\end{center}
Here you can find a digital copy of the syllabus and other important information.\\

\section*{Grading}
This course will use \textbf{Standards-Based Grading}.
This grading style emphasizes demonstration of subject mastery by students over accumulation of points.
The course content is broken into 15 \textit{standards} listed below.

\subsection*{Standards}
The following is a list of standards that you are expected to master by the end of this semester.
\begin{multicols}{2}
  \begin{enumerate}[1.]
\item
  Functions
  \begin{itemize}
  \item
    Limits
  \item
    Continuity of a function
  \item
    Intermediate Value Theorem
  \end{itemize}
\item
  Derivatives
  \begin{itemize}
  \item
    Limit Definition
  \item
    Tangent line to a function at a point
  \item
    Instantaneous and average rates of change
  \end{itemize}
\item
  Derivative Rules
  \begin{itemize}
  \item
    Power rule
  \item
    Sum/Difference rule
  \item
    Constant multiple rule
  \end{itemize}
\item
  Product and Quotient Rules
\item
  Chain Rule
\item
  Implicit Differentiation and Related Rates
\item
  How derivatives affect the shape of a graph
  \begin{itemize}
  \item
    Critical numbers
  \item
    First derivative test
  \item
    Increasing/Decreasing test
  \item
    Second derivative test
  \item
    Concavity and Inflection points
  \end{itemize}
\item
  Asymptotes
  \begin{itemize}
  \item
    Horizontal
  \item
    Vertical
  \item
    Slant/Oblique
  \end{itemize}
\item
  Curve sketching
\item
  Closed interval method and Optimization
\item
  Integrals
  \begin{itemize}
  \item
    Areas and distances
  \item
    Riemann Sums
  \item
    Definite integrals
  \end{itemize}
\item
  Fundamental Theorem of Calculus
  \begin{itemize}
  \item
    Anti-derivatives
  \item
    Indefinite integrals
  \end{itemize}
\item
  Substitution
\item
  Inverse Functions
  \begin{itemize}
  \item
    Exponentials and their properties
  \item
    Logarithms and their properties
  \end{itemize}
\item
  Calculus of Inverse Functions
  \begin{itemize}
  \item
    Derivatives and integrals involving exponentials
  \item
    Derivatives and integrals involving logarithms
  \end{itemize}
  \end{enumerate}
\end{multicols}

\subsection*{Problem Scoring}
Each problem that you encounter during this semester will be scored on the following scale:
\begin{center}
  \begin{tabular}{|r|l|}
    \hline
    \textbf{M}astery & The given solution is correct with no content related errors.\\
    (4 Points) & Appropriate justification is provided in a clear, easy to follow manner.\\
    \hline
    \textbf{P}roficiency & The given solution is mostly correct, with only minor content errors.\\
    (3 Points) & Appropriate justification is provided.\\
    \hline
    \textbf{I}mproving & The given solution is only partially correct or lacks justification.\\
    (2 Points) & \\
    \hline
    \textbf{R}ookie & The given solution is incorrect, but correct techniques were identified.\\
    (1 Point) & \\
    \hline
    \textbf{N}ot assessable & The given solution was blank, illegible, or used inappropriate techniques. \\
    (0 points) & \\
    \hline
  \end{tabular}
\end{center}

\subsection*{Mastery}
You can achieve mastery of a standard by receiving a score of \textbf{M} on in-class assessments and the final exam.
Once you have achieved mastery, problems explicitly from that standard will no longer appear on your assessments.
However, calculus builds upon itself, so the concepts in any given standard will certainly reappear in later standards.

It is important to note that, unlike a traditional grading scheme, you will be afforded multiple opportunities to display mastery and your past performance does not affect mastery.


\subsection*{Scale}
Letter grades will be assigned based on the following table.
\begin{center}
  \begin{tabular}{|l|c|c|}
    \hline
    & \# Standards mastered & \# Homework sets mastered\\
    \hline
    A & 14-15 & 90\%-100\%\\
    \hline
    B+ & 13 & 86\%-89\%\\
    \hline
    B & 12 & 80\%-85\%\\
    \hline
    C+ & 11 & 76\%-79\%\\
    \hline
    C & 10 & 70\%-75\%\\
    \hline
    D+ & 9 & 66\%-69\%\\
    \hline
    D & 8& 60\% - 65\%\\
    \hline
    F & \(\leq 7\) & \(< 60\%\)\\
    \hline
  \end{tabular}
\end{center}
In order to attain the letter grade in a given row, you must satisfy both criteria.
If you have mastered the number of standards in a row, but you have not mastered the appropriate number of homework sets, then you will be bumped down one row.

\section*{Assessments}

\subsection*{Homework}
Regular homework will be assigned, collected, and scored.
The problems are chosen to highlight the core concepts from the standards.
Mastery of these homework sets serves as a good indicator for quiz and exam performance.
As such, you should ensure that you fully understand the material on these homework sets; that is, upon completion of the homework set, you should be capable of completing similar problems without the aid of the text, a computer, a calculator, or any other tools not available during an exam.

%Late work will {\bf not} be accepted, and you are solely responsible for ensuring that these assignments are completed on time.
%Do {\bf not} leave these assignments until the last minute.

\subsection*{Quizzes}
Quizzes will be given during class time, at least once per week.
You should expect to see a quiz shortly after completion of a homework set.
Each quiz will contain problems from only one standard, and will generally be your first opportunity to demonstrate mastery.
If you receive a score of \textbf{M} on each problem on a quiz, then you will have mastered that standard.

\subsection*{Exams}
There will be three in-class exams and a final exam.
Each exam will be comprised of questions corresponding to the standards that you have not yet mastered and labeled accordingly.
Mastery of a standard depends \textbf{only} on attaining a score of \textbf{M} on the questions corresponding to that standard.

The in-class exams are tentatively scheduled as follows:
\begin{center}
  \begin{tabular}{rl}
    Exam 1: & Wednesday, February 20, 2019,\\
    Exam 2: & Monday, April 1, 2019, and\\
    Exam 3: & Monday, April 22, 2019.
  \end{tabular}
\end{center}

\subsection*{Re-Assessment}
You will have the opportunity to have a single homework set that you have not mastered re-assessed once each week in my office, either during office hours or at another arranged time.
I will provide you with at least one problem from the relevant standard which you will work out on the board.
Your goal in this re-assessment is to convince me that you have mastered the homework set.
If you are successful, then the score for that assignment will be changed to \textbf{M}.

\section*{Academic Support}
\subsection*{Calculus Cavalry}
\noindent
Open peer tutoring is available in Pardee Hall room 218 with the following hours:
\begin{center}
  \begin{tabular}{rl}
    Mondays: & 7-9 pm,\\
    Tuesdays: & 4-6 pm,\\
    Wednesdays: & 7-9 pm,\\
    Thursdays: & 4-6 pm and 7-9 pm,\\
    Sundays: &4-6 pm.
  \end{tabular}
\end{center}

\subsection*{Academic Resource Hub}
\noindent
The Academic Resource Hub (formerly ATTIC) provides academic services to enhance student success and is located on the third floor of Scott Hall.

Resources available to students include:
\begin{itemize}
\item 
  Tutoring and Supplemental Instruction
\item
  Academic Enrichment Resources
\item
  Accessibility Services
\item
  Services for Varsity Student Athletes
\end{itemize}
For more information, see the website at \url{http://hub.lafayette.edu}.

\subsection*{Disability statement}
In compliance with Lafayette College policy and equal access laws, I am available to discuss appropriate academic accommodations that you may require as a student with a disability.  Requests for academic accommodations need to be made during the first two weeks of the semester, except for unusual circumstances, so arrangements can be made.  Students must register with the Office of the Dean of Advising and Co-Curricular Programs for disability verification and for determination of reasonable academic accommodations.


\section*{Expectations}
\noindent
\subsection*{Academic Integrity}
To maintain the scholarly standards of the College and, equally important, the personal ethical standards of our students, it is essential that written assignments be a student’s own work, just as is expected in examinations and class participation. A student who commits academic dishonesty is subject to a range of penalties, including suspension or expulsion. Finally, the underlying principle is one of intellectual honesty. If a person is to have self-respect and the respect of others, all work must be his/her own.

Any student found responsible of academic dishonesty \textbf{will receive a grade of F in the course} and disciplinary action according to the procedure outlined in \href{https://conduct.lafayette.edu/wp-content/uploads/sites/93/2018/08/StudentHandboook-2018-19.pdf}{Student Handbook}.

\subsection*{Attendance}
Lecture is the longest stretch of time each week in which you have access to an interactive learning resource (i.e. me).
As such, lecture is arguably the most valuable aspect of the course and you are expected to not only attend class, but to also actively engage with the material (e.g. ask questions, contribute answers, etc.).
Cell phones and other distractions should either be left at home or be silenced and remain stored your bag.
If you find yourself unable to attend the lecture, please contact me in advance, if possible, to see what you will miss.

\section*{Federal Credit Hour Requirement}
The student work in this course is in full compliance with the federal definition of a four credit hour course.
Please see the Registrar's Office web site

\begin{center}
  \url{http://registrar.lafayette.edu/additional-resources/cep-course-proposal}
\end{center}
for the full policy and practice statement.

\section*{Schedule}
Below is a tentative schedule for the course.
\begin{center}
  \begin{longtable}{lll}
    Date & Section(s) & Material\\
    \hline
    1/28 & 6.6 & Inverse trigonometric functions\\
    1/30 & 6.6, 6.8 & Indeterminate forms and L'H\^opital's Rule\\
    2/1 & 6.8\\
    2/4 & 5.1, 5.2 & Areas and Volumes\\
    2/6 & 5.1, 5.2\\
    2/8 & 7.1 & Integration by Parts\\
    2/11 & 7.4 & Partial fractions\\
    2/13 & 7.5 & Strategy for Integrations\\
    2/15 & 7.7 & Approximate integration\\
    2/18 & 7.7 & Using error bounds\\
    2/20 & 7.8 & Improper integrals\\
    2/22 & & Review for Exam 1\\
    2/25 & & Exam 1\\
    \hline\\
    2/27 & 9.1 & Introduction to differential equations\\ 
    3/1 & 9.2 & Direction fields and Euler's method\\
    3/4 & 9.3 & Separable differential equations\\
    3/6 & 6.5 & Exponential growth and decay\\
    3/8 & 10.1 & Parametric equations\\
    3/11 & 10.2 & Tangents and arclength\\
    3/13 & 10.3 & Polar coordinates\\
    3/15 & 10.4 & Areas in polar coordinates\\
    3/18 - 3/22 & & Spring Break\\
    3/25 & & Review for Exam 2\\
    3/27 & & Exam 2\\
    \hline\\
    3/29 & 11.1 & Sequences\\
    4/1 & 11.2 & Series\\ 
    4/3 & 11.3 & The integral test\\
    4/5 & 11.4 & The comparison test\\
    4/8 & 11.5 & Alternating series\\ 
    4/10 & 11.6 & Ratio and root tests\\ 
    4/12 & 11.7 & Strategy for testing series\\
    4/15 & 11.8 & Power series\\
    4/17 & 11.9 & Representing functions\\
    4/19 & 11.10 & Taylor and Maclaurin series\\
    4/22 & 11.10\\
    4/24 & 11.11 & Applications of Taylor polynomials\\
    4/27 & 11.11 \\
    4/29 & & Review for Exam 3\\
    5/1  & & Exam 3\\
    \hline
    5/3 & & TBA\\
    5/6 & & TBA\\
    5/8 & & Final Exam Review\\
    5/10 & & Final Exam Review\\
  \end{longtable}
\end{center}

\end{document}
