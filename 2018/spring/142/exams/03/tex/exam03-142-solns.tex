\documentclass[12pt]{amsart}
\usepackage{amsmath,enumerate,commath,tikz}
\usetikzlibrary{angles,quotes}
\openup 5pt
\author[Blake Farman]{Blake Farman\\University of South Carolina}
\title[Exam 03]{Math 142: Exam 03}
\date{April 17, 2018}
\pdfpagewidth 8.5in
\pdfpageheight 11in
\usepackage[margin=1in]{geometry}

\renewcommand{\qedsymbol}{}

\begin{document}
\maketitle

\begin{center}
  \fbox{\fbox{\parbox{5.5in}{\centering
        Answer the questions in the spaces provided on the
        question sheets and turn them in at the end of the class period.

        Unless otherwise stated, all supporting work is required.
        Unsupported or otherwise mysterious answers will \textbf{not receive credit.}
        
        You may \textbf{not} use a calculator or any other electronic device, including cell phones, smart watches, etc.
        By writing your name on the line below, you indicate that you have read and understand these directions.


        It is advised, although not required, that you check your answers.}}}
\end{center}

\vspace{0.2in}
\makebox[\textwidth]{Name:\enspace\hrulefill}
\vspace{0.2in}

\theoremstyle{definition}
\newtheorem{thm}{}
\renewcommand{\qedsymbol}{}

\[
\begin{array}{|c|c|c|}
  \hline
  \text{Problem} & \text{Points Earned} & \text{Points Possible}\\
  \hline
  1 & & 25\\
  \hline
  2 & & 25\\
  \hline
  3 & & 25\\
  \hline
  4 & & 25\\
  \hline
  \text{Exam 1 Bonus} & & -\\
  \hline
  \text{Exam 2 Bonus} & & -\\
  \hline
  \text{Total} & & 100\\
  \hline
\end{array}
\]

\newpage

\section{Problems}

\begin{thm}[25 Points]
  Find the radius and interval of convergence for the power series
  \[\sum_{n = 0}^\infty \frac{3^nx^n}{n!}.\]
\end{thm}

\begin{proof}[Solution]
  We use the Ratio Test.
  First we compute the limit
  \[\rho 
  = \lim_{n \to \infty} \abs{\frac{3^{n+1}x^{n+1}}{(n+1)!} \cdot \frac{n!}{3^nx^n}}
  = \lim_{n \to \infty} \frac{3\abs{x}}{n + 1}
  = \lim_{n \to \infty} 3\abs{x} \cdot \lim_{n \to \infty} \frac{1}{n+1} = 3\abs{x} \cdot 0 = 0.\]
  Since \(\rho < 1\) holds for all \(x\), the radius of convergence is \(R = \infty\) and thus this series converges absolutely on \(\mathbb{R} = (-\infty,\infty)\).
\end{proof}

\begin{thm}[25 Points]
  Find the Maclaurin series for the function
  \[f(x) = \frac{1}{(1-x)^3}.\]
\end{thm}

\begin{proof}[Solution]
  First we recognize that on \((-1,1)\) we have the power series
  \[F(x) = \frac{1}{1-x} = \sum_{n = 0}^\infty x^n\]
  so taking the derivative of the left-hand side twice yields
  \[F^\prime(x) = \frac{\dif }{\dif x} (1 - x)^{-1}
  = (-1)(1 - x)^{-2}(-1)
  = (1 - x)^{-2}\]
  and
  \[
  F^{\prime\prime}(x) = (-2)(1 - x)^{-3}(-1)
  = 2\frac{1}{(1 - x)^3}.
  \]
  Dividing both sides by 2 yields
  \[\frac{1}{2}F^{\prime\prime}(x) = \frac{1}{(1 - x)^3} = f(x).\]
  Applying Term-by-Term Differentiation twice to the Maclaurin series for \(F(x)\) we have
  \[F^\prime(x) = \sum_{n = 0}^\infty \frac{\dif}{\dif x}x^n = \sum_{n = 0}^\infty nx^{n-1} = \sum_{n = 1}^\infty nx^{n-1}\]
  and
  \[F^{\prime\prime}(x) = \sum_{n = 1}^\infty \frac{\dif}{\dif x} nx^{n-1}
  = \sum_{n = 1}^\infty n(n-1)x^{n-2}
  = \sum_{n = 2}^\infty n(n-1)x^{n-2}\]
  Therefore
  \[f(x) = \frac{1}{(1 - x)^3} =\frac{1}{2}F^{\prime\prime}(x)
  = \frac{1}{2}\sum_{n = 2}^\infty n(n-1)x^{n-2}
  = \sum_{n = 2}^\infty \frac{n(n-1)}{2}x^{n-2}\]
  holds for \(-1 < x < 1\).
\end{proof}

\begin{proof}[Alternative Solution]
  We could also compute this series directly.
  We have
  \[f(x) = \frac{1}{(1 - x)^{3}} = (1 - x)^{-3}\]
  so using
  \[\frac{\dif}{\dif x}(1 - x) = -1\]
  the Chain Rule, the derivatives are
  \begin{eqnarray*}
    f^\prime(x) &=& (-3)(1 - x)^{-4}(-1)
    = \frac{3}{(1 - x)^{4}}
    = \frac{(1 + 2)!}{2}\frac{1}{(1-x)^{3 + 1}}\\
    f^{\prime\prime}(x) &=& (-4)(3)(1 - x)^{-5}(-1)
    = (4)(3)(1 - x)^{-5}
    = \frac{4!}{2!(1 - x)^{5}}
    = \frac{(2 + 2)!}{2(1 - x)^{3 + 2}} \\
    f^{\prime\prime\prime}(x) &=& (-5)(4)(3)(1-x)^{-6}(-1)
    = (5)(4)(3)(1 - x)^{-6}
    = \frac{5!}{2(1 - x)^{3+3}}
    = \frac{(3 + 2)!}{2(1 - x)^{3 + 3}}\\
    &\cdots&\\
    f^{k}(x) &=& \frac{(k + 2)!}{2(1 - x)^{3 + k}}\\
    &\cdots&
  \end{eqnarray*}
  This gives us
  \[f^k(0) = \frac{(k+2)!}{2(1 - 0)^{3 + k}} = \frac{(k+2)!}{2}\]
  and so the Maclaurin series is
  \[\sum_{k = 0}^\infty \frac{f^k(0)}{k!} x^k = \sum_{k = 0}^\infty \frac{(k+2)}{2(k!)} x^k = \sum_{k = 0}^\infty \frac{(k+2)(k+1)}{2} x^k.\]
  Using the Ratio Test we see that
  \[\lim_{k \to \infty} \abs{\frac{(k+3)(k+2)x^{k+1}}{2} \cdot \frac{2}{(k+2)(k+3)x^k}} = \lim_{k \to \infty} \frac{k^2 + 5k + 6}{k^2 + 3k + 1} \abs{x} = \abs{x}\]
  implies this series converges on \((-1,1)\).
  Note, however, that in contrast with the previous solution, this does not give us any information about whether this series converges to \(f(x)\)!
\end{proof}

\begin{thm}[25 Points]
  Find the Maclaurin series for \(x\ln(1 + 2x)\).
\end{thm}

\begin{proof}[Solution]
  Either recall that on \((-1,1]\)
  \[\ln(1 + x) = \sum_{n = 1}^\infty \frac{(-1)^{n-1}}{n} x^n\]
  or, if you haven't memorized this formula, use substitution to get
  \[\frac{1}{1+x} = \frac{1}{1 - (-x)} = \sum_{n = 0}^\infty (-x)^n = \sum_{n = 0}^\infty (-1)^nx^n,\]
  integrate the function using the change of variables \(u = 1 + x\), \(\dif u = \dif x\) to get
  \[\int \frac{\dif x}{1 + x} = \ln(1 + x) + c,\]
  then apply Term-by-Term Integration to get
  \[\ln(1 + x) + c = \int \frac{\dif x}{1 + x} = \sum_{n = 0}^\infty \int (-1)^nx^n\dif x = \sum_{n = 0}^\infty \frac{(-1)^n}{n + 1} x^{n+1} = \sum_{n = 1}^\infty \frac{(-1)^{n-1}}{n} x^n\]
  and, finally, observe that evaluating the left-hand side at \(x = 0\) yields
  \[\ln(1 + 0) + c = 0 + c = c\]
  and evaluating the right-hand side at \(x = 0\) yields
  \[\sum_{n = 1}^\infty \frac{(-1)^{n-1}}{n} 0^n = \sum_{n = 0}^\infty 0 = 0\]
  to see that \(c = 0\).

  Using substitution and the formula for the product of two convergent power series, we have
  \[x\ln(1 + 2x) = x \sum_{n = 1}^\infty \frac{(-1)^{n-1}}{n} (2x)^n
  = x \sum_{n = 1}^\infty \frac{(-1)^{n-1}2^n}{n} x^n
  = \sum_{n = 1}^\infty \frac{(-1)^{n-1}2^n}{n} x^{n+1}.\]
  Having made the substitution, we note that this holds for \(-1 < 2x \leq 1\) or, equivalently, \(-1/2 < x \leq -1/2\).
\end{proof}
\begin{proof}[Bare-hand Solution]
  First we recall that
  \[\frac{1}{1-x} = \sum_{n=0}^\infty x^n,\, -1 \leq x \leq 1\]
  so
  \[\frac{1}{1 + 2x} = \frac{1}{1 - (-2x)} = \sum_{n = 0}^\infty (-2x)^n = \sum_{n = 0}^\infty (-1)^n2^nx^n\]
  holds so long as \(\abs{-2x} = 2\abs{x} < 1\) or, equivalently, \(\abs{x} < 1/2\).
  Now we observe that using the change of variables \(u = 1 + 2x\), \(\dif u/2 = \dif x\), the integral of the left-hand side is
  \[\int \frac{\dif x}{1 + 2x} = \int \frac{\dif u}{2u} = \frac{1}{2} \int \frac{\dif u}{u} = \frac{1}{2}\ln(u) + c = \frac{1}{2}\ln(1 + 2x) + c\]
  while applying Term-by-Term Integration to the series yields
  \[\frac{1}{2}\ln(1 + 2x) + c = \int \frac{\dif x}{1 + 2x}
  = \sum_{n = 0}^\infty \int\left[(-1)^n2^nx^n\right]\dif x
  = \sum_{n = 0}^\infty \frac{(-1)^n2^n}{n+1}x^{n+1}\]
  on \((-1/2,1/2)\).
  Evaluating the left-hand side of this equation at \(x = 0\) we have
  \[\frac{1}{2}\ln(1 + 2(0)) + c = \frac{1}{2}\ln(1) + c = 0 + c = c\]
  while evaluating the right-hand side of this equation at \(x = 0\) we have
  \[\sum_{n = 0}^\infty \frac{(-1)^n2^n}{n+1}0^{n+1} = \sum_{n=0}^\infty 0 = 0\]
  implies that \(c = 0\).
  Multiplying both sides by \(2x\) we obtain
  \[x\ln(1 + 2x) = 2x\sum_{n = 0}^\infty \frac{(-1)^n2^n}{n+1}x^{n+1}
  = \sum_{n = 0}^\infty \frac{(-1)^n2^{n+1}}{n+1}x^{n+2}
  = \sum_{n = 1}^\infty \frac{(-1)^{n-1}2^{n}}{n}x^{n+1}\]
  for \(-1/2 < x < 1/2\).
  This equality also holds for \(x = 1/2\), but requires some care.
\end{proof}

\begin{thm}[25 Points]
  Find the Maclaurin series for
  \[\ln\left(\frac{1 + x}{1 - x}\right)\]
\end{thm}

\begin{proof}[Solution]
  First, rewrite
  \[\ln\left(\frac{1+x}{1-x}\right) = \ln(1 + x) - \ln(1 - x).\]
  Next, we recall that
  \[\ln(1 + x) = \sum_{n = 1}^\infty \frac{(-1)^{n-1}}{n} x^n\]
  holds for \(-1 < x \leq 1\).
  Using substitution we obtain
  \[\ln(1 - x) = \ln(1 + (-x)) =
  \sum_{n = 1}^\infty \frac{(-1)^{n-1}(-1)^n}{n} (-x)^n
  = \sum_{n = 1}^\infty \frac{(-1)^{2n - 1}}{n} x^n
  = -\sum_{n = 1}^\infty \frac{1}{n} x^n\]
  because
  \[(-1)^{2n - 1} = (-1)^{2n}(-1)^{-1} = \frac{((-1)^2)^n}{-1} = \frac{1^n}{-1} = \frac{1}{-1} = -1,\]
  which holds for \(-1 \leq x < 1\).

  Since both of these series converge on \((-1,1)\), we have
  \begin{eqnarray*}
    \ln\left(\frac{1 + x}{1-x}\right) &=& \sum_{n = 1}^\infty \frac{(-1)^{n-1}}{n} x^n - \left(-\sum_{n = 1}^\infty \frac{1}{n} x^n\right)\\
    &=& \sum_{n = 1}^\infty \frac{(-1)^{n-1}}{n} x^n + \sum_{n = 1}^\infty \frac{1}{n} x^n\\
    &=& \sum_{n = 1}^\infty \left[\frac{(-1)^{n-1}}{n} x^n + \frac{1}{n} x^n\right]\\
    &=& \sum_{n = 1}^\infty \frac{(-1)^{n-1} +1}{n} x^n\\
  \end{eqnarray*}
  When \(n\) is an even number, \(n - 1\) is odd, so 
  \[(-1)^{n-1} + 1 = -1 + 1 = 0\]
  and when \(n\) is an odd number, \(n - 1\) is even, so
  \[(-1)^{n - 1} + 1 = 1 + 1 = 2.\]
  Therefore
  \[\ln\left(\frac{1 + x}{1 - x}\right) = \sum_{k=0}^\infty \frac{2}{2k+1} x^{2k+1}\]
  holds for \(-1 < x < 1\).
  We note that because the function
  \[\ln\left(\frac{1 + x}{1 - x}\right)\]
  is not defined at \(x = 1\) and when \(x = -1\) the series
  \[\sum_{k = 0}^\infty \frac{2}{2k + 1} (-1)^{2k + 1} = \sum_{k = 0}^\infty \frac{-2}{2k + 1} = -2 - \frac{2}{3} - \frac{2}{5} - \ldots\]
  does not converge to zero, this equality does not hold at either of the endpoints.
  
\end{proof}

\begin{proof}[Bare-hand Solution]
  First, rewrite
  \[\ln\left(\frac{1+x}{1-x}\right) = \ln(1 + x) - \ln(1 - x).\]
  We see that by making the change of variables \(u = 1 + x\), \(\dif u = \dif x\) we have
  \[\int \frac{\dif x}{1 + x} = \int \frac{\dif u}{u} = \ln(u) + c_1 = \ln(1 + x) + c_2\]
  and making the change of variables \(u = 1-x\), \(-\dif u = -\dif x\) we have
  \[\int \frac{\dif x}{1 - x} = - \int \frac{\dif u}{u} = -(\ln(u) + c_2) = -\ln(1 - x) - c_2.\]
  Letting \(c = c_1 - c_2\) we have
  \begin{eqnarray*}
    \ln(1 + x) - \ln(1 - x) + c 
    &=&\int \frac{\dif x}{1 + x} + \int \frac{\dif x}{1 - x}\\
    &=&\int\left[\frac{1}{1 + x} + \frac{1}{1 - x}\right]\dif x
  \end{eqnarray*}
  For \(-1 < x < 1\) we have
  \[
  \frac{1}{1 + x} + \frac{1}{1 - x} = \sum_{n = 0}^\infty (-x)^n + \sum_{n = 0}^\infty x^n = \sum_{n = 0}^\infty \left[(-1)^n + 1\right]x^n
  \]
  We observe that when \(n\) is even \((-1)^n + 1 = 1 + 1 = 2\) and when \(n\) is odd \((-1)^n + 1 = -1 + 1 = 0\) so
  \[\frac{1}{1 + x} + \frac{1}{1 - x} = \sum_{k = 0}^\infty 2x^{2k}\]
  Now, putting this all together and applying Term-by-Term Integration we have
  \begin{eqnarray*}
    \ln(1 + x) - \ln(1 - x) + c 
    &=&\int\left[\frac{1}{1 + x} + \frac{1}{1 - x}\right]\dif x\\
    &=& \sum_{k=0}^\infty \int 2x^{2k}\dif x\\
    &=& \sum_{k=0}^\infty \frac{2}{2k+1} x^{2k+1}\\
  \end{eqnarray*}
  Evaluating the left-hand side of this equation at \(x = 0\) we get
  \(\ln(1 + 0) - \ln(1 - 0) + c = \ln(1) - \ln(1) + c = c\)
  and evaluating the right-hand side of this equation at \(x = 0\) we get
  \[\sum_{k = 0}^\infty \frac{2}{2k+1} 0^{2k+1} = \sum_{k = 0}^\infty 0 = 0.\]
  Therefore
  \[\ln\left(\frac{1 + x}{1 - x}\right) = \ln(1 + x) - \ln(1 - x) = \sum_{k=0}^\infty \frac{2}{2k+1} x^{2k+1}\]
  for \(-1 < x < 1\).
\end{proof}

\begin{thm}[Bonus - Exam 1]
  Decide whether 
  \[\int_2^\infty \frac{\dif x}{x^2 - 1}\]
  converges or diverges.
  If it converges, find the value of the integral.
\end{thm}

\begin{proof}[Solution]
  By definition we have
  \[\int_2^\infty \frac{\dif x}{x^2 - 1} = \lim_{t \to \infty} \int_2^t \frac{\dif x}{x^2 - 1}.\]
  Factoring the denominator as \(x^2 - 1 = (x + 1)(x - 1)\) we can do the definite integral by partial fraction decomposition as follows.
  Set
  \[\frac{1}{(x - 1)(x + 1)} = \frac{A}{x - 1} + \frac{B}{x + 1}\]
  then clear denominators to get
  \[1 = A(x + 1) + B(x - 1) = (A + B)x + (A - B)\]
  and equate coefficients to obtain the system
  \begin{eqnarray*}
    0 &=& A + B\\
    1 &=& A - B.
  \end{eqnarray*}
  Adding the two equations together gives \(1 = 2A\), while subtracting the second equation from the first gives \(-1 = 2B\).
  Thus \(A = 1/2\), \(B = -1/2\), and
  \begin{eqnarray*}
    \lim_{t \to \infty} \int_2^t \frac{\dif x}{x^2 - 1} &=& \lim_{t \to \infty}\left(\frac{1}{2}\int_2^t \frac{\dif x}{x - 1} - \frac{1}{2} \int_2^t \frac{\dif x}{x + 1}\right)\\
    &=& \lim_{t \to \infty} \left(\frac{1}{2}\left[\ln \abs{x - 1} - \ln\abs{x+1}\right]_2^t\right)\\
    &=& \lim_{t \to \infty} \left(\frac{\ln \abs{t - 1} - \ln\abs{t + 1} - \ln(2 - 1) + \ln(2 + 1)}{2}\right)\\
    &=& \lim_{t \to \infty} \frac{1}{2}\left(\ln\abs{\frac{t - 1}{t + 1}} + \ln(3)\right).
  \end{eqnarray*}
  Since both the natural logarithm and the absolute value functions are continuous we have 
  \[\lim_{t \to \infty} \ln\abs{\frac{t - 1}{t + 1}} = \ln\left(\lim_{t \to \infty} \abs{\frac{t -  1}{t + 1}}\right) = \ln\abs{\lim_{t\to\infty} \frac{t -1}{t + 1}} = \ln(1) = 0.\]
  Therefore
  \[\int_2^\infty \frac{\dif x}{x^2 - 1} =  \lim_{t \to \infty} \frac{1}{2}\left(\ln\abs{\frac{t - 1}{t + 1}} + \ln(3)\right) = \frac{0 + \ln(3)}{2} = \frac{\ln(3)}{2}.\]
\end{proof}

\begin{thm}[Bonus - Exam 2]
  Determine whether the series
  \[\sum_{n = 1}^\infty (-1)^n\frac{2n}{4n^2 + 1}\]
  converges conditionally, converges absolutely, or diverges.
\end{thm}

\begin{proof}[Solution]
  First we observe that this series does not converge absolutely.
  By computing the limit
  \[\lim_{n \to \infty} \frac{\frac{2n}{4n^2 + 1}}{\frac{1}{n}} = \lim_{n \to \infty} \frac{2n^2}{4n^2 + 1} = \frac{1}{2} > 0\]
  we see that the series
  \[\sum_{n = 1}^\infty \frac{2n}{4n^2 + 1}\]
  diverges by Part~(1) of the Limit Comparison Test because the Harmonic Series diverges.

  Next we try the Alternating Series Test.
  The first and third conditions are easy to verify: when \(1 \leq n\) it's clear that
  \[0 < u_n = \frac{2n}{4n^2 + 1}\]
  holds and also
  \[\lim_{n \to \infty} u_n = \lim_{n \to \infty} \frac{2n}{4n^2 + 1} = \lim_{n \to \infty} \frac{n^2}{n^2} \cdot \frac{2/n}{4 + 1/n^2} = \lim_{n \to \infty} \frac{2/n}{4 + 1/n^2} = \frac{0}{4 + 0} = 0.\]
  To conclude that this series converges by the Alternating Series Test, we need only verify that for some integer \(N\), \(u_{n + 1} \leq u_n\) holds whenever \(N \leq n\).
  Towards that end let \(f(x) = \frac{2x}{4x^2 + 1}\) and observe that
  \[f^\prime(x) = \frac{2(4x^2 + 1) - (2x)(8x)}{(4x^2 + 1)^2} = \frac{8x^2 + 2 - 16x}{(4x^2 + 1)^2} = \frac{-8x^2 + 2}{(4x^2 + 1)^2} < 0\]
  holds if and only if
  \[-8x^2 +2 < 0\]
  which holds if and only if
  \[\sqrt{\frac{2}{8}} = \sqrt{\frac{1}{4}} = \frac{1}{\sqrt{4}} = \frac{1}{2} < x.\]
  Since \(f\) is a decreasing function if and only if \(f^\prime(x) < 0\), we see that
  \[u_{n + 1} = f(n + 1) \leq f(n) = u_n\]
  holds whenever \(1 \leq n\).
  Therefore the series
  \[\sum_{n = 1}^\infty (-1)^n\frac{2n}{4n^2 + 1}\]
  converges conditionally.
\end{proof}

\end{document}
