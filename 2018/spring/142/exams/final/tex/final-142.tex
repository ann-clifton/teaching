\documentclass[12pt]{amsart}
\usepackage{amsmath,enumerate,commath,tikz}
\usetikzlibrary{angles,quotes}
\openup 5pt
\author[Blake Farman]{Blake Farman\\University of South Carolina}
\title[Final Exam]{Math 142: Final Exam}
\date{May 8, 2018}
\pdfpagewidth 8.5in
\pdfpageheight 11in
\usepackage[margin=1in]{geometry}

\renewcommand{\qedsymbol}{}

\begin{document}
\maketitle

\begin{center}
  \fbox{\fbox{\parbox{5.5in}{\centering
        Answer the questions in the spaces provided on the
        question sheets and turn them in at the end of the class period.

        Unless otherwise stated, all supporting work is required.
        Unsupported or otherwise mysterious answers will \textbf{not receive credit.}
        
        You may \textbf{not} use a calculator or any other electronic device, including cell phones, smart watches, etc.
        By writing your name on the line below, you indicate that you have read and understand these directions.


        It is advised, although not required, that you check your answers.}}}
\end{center}

\vspace{0.2in}
\makebox[\textwidth]{Name:\enspace\hrulefill}
\vspace{0.2in}

\theoremstyle{definition}
\newtheorem{thm}{}
\renewcommand{\qedsymbol}{}

\[
\begin{array}{|c|c|c|}
  \hline
  \text{Problem} & \text{Points Earned} & \text{Points Possible}\\
  \hline
  1 & & 10\\
  \hline
  2 & & 10\\
  \hline
  3 & & 10\\
  \hline
  4 & & 10\\
  \hline
  5 & & 10\\
  \hline
  6 & & 10\\
  \hline
  7 & & 10\\
  \hline
  8 & & 10\\
  \hline
  9 & & 10\\
  \hline
  10 & & 10\\
  \hline
  \text{Total} & & 100\\
  \hline
\end{array}
\]

\newpage

\section{Integrals}

In this section, determine the appropriate method and compute the given integral.
\begin{thm}[10 Points]
  Compute
  \[\int x\ln(x)\dif x.\]
\end{thm}

\vspace{3in}

\begin{thm}[10 Points]
  Compute
  \[\int \frac{\dif x}{\sqrt{9 + x^2}}\]
\end{thm}

\newpage
\begin{thm}[10 Points]
Compute
\[\int \frac{\dif x}{x^2 + 2x}\]
\end{thm}

\newpage

\section{Series}

\begin{thm}[10 Points]
  Determine whether the series
  \[\sum_{n=0}^\infty \left(\frac{5}{2^n} - \frac{2}{3^{n+1}}\right)\]
  converges or diverges.
  If it converges, find the sum of the series.
\end{thm}

\newpage

\begin{thm}[10 Points]
  Use the \textbf{integral test} to determine whether the series
  \[\sum_{n=1}^\infty \frac{n}{n^2 + 4}\]
  converges or diverges.
\end{thm}

\vspace{3in}
\begin{thm}[10 Points]
  Determine whether the series
  \[\sum_{n=1}^\infty (-1)^n\frac{n}{n^2 + 4}\]
  converges conditionally, converges absolutely, or diverges.
\end{thm}

\newpage
\section{Power Series}
\begin{thm}[10 Points]
  Find the radius and interval of convergence for the power series
  \[\sum_{n=1}^\infty \frac{4^nx^{2n}}{n}.\]
\end{thm}

\newpage

\begin{thm}[10 Points]
  Given the Maclaurin series
  \[\frac{1}{1+x} = \sum_{n = 0}^\infty (-1)^n x^n,\, -1 < x < 1\]
  find the Taylor series centered at \(a = 1\) for 
  \[f(x) = \frac{1}{x}.\]
  Where is the Taylor series equal to the function?\\
  {[Hint: \(x = 1 + (x - 1)\).]}
\end{thm}

\vspace{2in}

\begin{thm}[10 Points]
  Use your answer to Problem 8 to compute the Taylor series for
  \[f(x) = \ln(x)\]
  centered at \(a = 1\).
  Where is the Taylor series equal to the function?
\end{thm}

\newpage

\begin{thm}[10 Points]
  Use your solution to Problem 8 to find the Taylor series centered at \(a = 1\) for the function
  \[f(x) = \frac{1}{x^2}.\]
  Where is the Taylor series equal to the function?
\end{thm}

\end{document}
