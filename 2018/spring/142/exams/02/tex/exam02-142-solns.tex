\documentclass[12pt]{amsart}
\usepackage{amsmath,enumerate,commath,tikz}
\usetikzlibrary{angles,quotes}
\openup 5pt
\author[Blake Farman]{Blake Farman\\University of South Carolina}
\title[Exam 02]{Math 142: Exam 02\\Solutions}
\date{March 20, 2018}
\pdfpagewidth 8.5in
\pdfpageheight 11in
\usepackage[margin=1in]{geometry}

\renewcommand{\qedsymbol}{}

\begin{document}
\maketitle

\begin{center}
  \fbox{\fbox{\parbox{5.5in}{\centering
        Answer the questions in the spaces provided on the
        question sheets and turn them in at the end of the class period.

        Unless otherwise stated, all supporting work is required.
        Unsupported or otherwise mysterious answers will \textbf{not receive credit.}
        
        You may \textbf{not} use a calculator or any other electronic device, including cell phones, smart watches, etc.
        
        It is advised, although not required, that you check your answers.

        By writing your name on the line below, you indicate that you have read and understand these directions.}}}
\end{center}

\vspace{0.2in}
\makebox[\textwidth]{Name:\enspace\hrulefill}
\vspace{0.2in}

\theoremstyle{definition}
\newtheorem{thm}{}
\newtheorem*{caution}{\textbf{Caution}}
\renewcommand{\qedsymbol}{}

\[
\begin{array}{|c|c|c||c|c|c|}
  \hline
  \text{Tests} & \text{Points Earned} & \text{Points Possible} & \text{Problems} & \text{Points Earned} & \text{Points Possible}\\
  \hline
  1 & & 3 & 1 & & 5\\
  \hline
  2 & & 3 & 2 & & 5\\
  \hline
  3 & & 2 & 3 & & 5\\
  \hline
  4 & & 4 & 4 & & 10\\
  \hline
  5 & & 4 & 5 & & 10\\
  \hline
  6 & & 7 & 6 & & 15\\
  \hline
  7 & & 2 & 7 & & 15\\
  \hline
  8 & & 4 & \text{Subtotal} & & 65\\
  \hline
  9 & & 3 &\text{Bonus} & & 5\\
  \hline
  10 & & 3 & & &\\
  \hline
  \text{Subtotal} & & 35 & \text{Total} & & 100\\
  \hline
\end{array}
\]

\newpage

\section{Tests}

Fill in the blanks.

\begin{thm}[3 Points - \(n^\text{th}\) Term Test for Divergence]
  The series \(\sum_{n = 1}^\infty a_n\) diverges if \(\lim_{n \to \infty} a_n \neq 0\) or if \(\lim_{n \to \infty} a_n\) does not exist.
  The test is inconclusive if \(\lim_{n \to \infty} a_n = 0\).
\end{thm}

\begin{thm}[3 Points - Geometric Series]
  The series \(\sum_{n = 0}^\infty a r^n\) converges if \(\abs{r} < 1\) and diverges if \(1 \leq \abs{r}\).
  If the series converges, then its sum is
  \[\sum_{n = 0}^\infty a r^n = \frac{a}{1 - r}.\]
\end{thm}

\begin{thm}[2 Points - \(p\)-series]
  The series \(\sum_{n = 1}^\infty \frac{1}{n^p}\) converges if \(1 < p\) and
  diverges if \(p \leq 1\).
\end{thm}

\begin{thm}[4 Points - Integral Test]
  Let \(\{a_n\}\) be a sequence of positive terms.
  Suppose that \(f\) is a continuous, positive, decreasing function of the variable \(x\) for all \(N \leq x\), for some positive integer \(N\).
  Then the series \(\sum_{n = N}^\infty a_n\) and the integral \(\int_N^\infty f(x)\dif x\) either both converge or both diverge.
\end{thm}

\begin{thm}[4 Points - Comparison Test]
  Let \(\sum a_n\), \(\sum c_n\), and \(\sum d_n\) be series with non-negative terms.
  Suppose that for some integer \(N\)
  \[d_n \leq a_n \leq c_n\ \text{for all}\ N < n.\]
  \begin{enumerate}[(a)]
  \item
    If \(\sum c_n\) converges, then \(\sum a_n\) converges.
  \item
    If \(\sum d_n\) diverges, then \(\sum a_n\) diverges.
  \end{enumerate}
\end{thm}

\begin{thm}[7 Points - Limit Comparison Test]
  Suppose that \(0 < a_n\) and \(0 < b_n\) for all \(N \leq n\), for some positive integer \(N\).
  \begin{enumerate}
  \item
    If
    \[\lim_{n \to \infty} \frac{a_n}{b_n} = c > 0,\]
    then \(\sum a_n\) and \(\sum b_n\) both converge or both diverge.
  \item
    If
    \[\lim_{n \to \infty} \frac{a_n}{b_n} = 0\]
    and \(\sum b_n\) converges, then \(\sum a_n\) converges.
  \item
    If
    \[\lim_{n \to \infty} \frac{a_n}{b_n} = \infty\]
    and \(\sum b_n\) diverges, then \(\sum a_n\) diverges.
  \end{enumerate}
\end{thm}

\begin{thm}[2 Points - Absolute Convergence Test]
  If the series \(\sum \abs{a_n}\) converges, then the series \(\sum a_n\) converges.
\end{thm}

\begin{thm}[4 Points - Ratio Test]
  Let \(\sum a_n\) be any series and suppose that
  \[\lim_{n \to \infty} \abs{\frac{a_{n+1}}{a_n}} = \rho.\]
  \begin{enumerate}[(a)]
  \item
    The series converges absolutely if \(\rho < 1\).
  \item
    The series diverges if \(1 < \rho\) or \(\rho = \infty\).
  \item
    The test is inconclusive if \(\rho = 1\).
  \end{enumerate}
\end{thm}

\begin{thm}[3 Points - Root Test]
  The \(\sum a_n\) be any series and suppose that
  \[\lim_{n \to \infty} \sqrt[n]{\abs{a_n}} = \rho.\]
  \begin{enumerate}[(a)]
  \item
    The series converges absolutely if \(\rho < 1\).
  \item
    The series diverges if \(1 < \rho\) or \(\rho = \infty\).
  \item
    The test is inconclusive if \(\rho = 1\).
  \end{enumerate}
\end{thm}

\begin{thm}[3 Points - Alternating Series Test]
  The series
  \[\sum_{n = 1}^\infty (-1)^{n+1} u_n = u_1 - u_2 + u_3 - u_4 + \ldots\]
  converges if all three of the following conditions are satisfied:
  \begin{enumerate}
  \item
    \(0 < u_n\) for all \(n\),
  \item
    \(u_{n + 1} \leq u_n\) for all \(N \leq n\), for some integer \(N\), and
  \item
    \(u_n \to 0\).
  \end{enumerate}
\end{thm}

\section{Problems}
\setcounter{thm}{0}
For each of the following problems, decide which of the tests listed above is appropriate and use it to show that the given series converges or diverges.
\textbf{
  Clearly indicate the test you used and why the series fits the hypotheses of the test.
  Failure to do so will result in a zero on the problem.
}

\begin{thm}[5 Points]
  Determine whether the series
  \[\sum_{n = 0}^\infty \frac{5(3^n) + 2^{n+1}}{6^n}\]
  converges or diverges.
  If the series converges, find the sum of the series.
\end{thm}

\begin{proof}[Solution 1]
  We observe that
  \[\frac{5(3^n) + 2^{n+1}}{6^n} = \frac{5}{2^n} + \frac{2}{3^n}\]
  which is the sum of terms of two convergent geometric series, so
  \begin{eqnarray*}
    \sum_{n = 0}^\infty \frac{5(3^n) + 2^{n+1}}{6^n} &=& \sum_{n = 0}^\infty \left[\frac{5}{2^n} + \frac{2}{3^n}\right]\\
    &=& \sum_{n = 0}^\infty \frac{5}{2^n} + \sum_{n = 0}^\infty \frac{2}{3^n}\\
    &=& \frac{5}{1 - \frac{1}{2}} + \frac{2}{1 - \frac{1}{3}}\\
    &=& 5(2) + 2\left(\frac{3}{2}\right) = 10 + 3 = 13.
  \end{eqnarray*}
\end{proof}

\begin{caution}I highly discourage using the Ratio Test when addition of exponents is involved.
However, since several people did attempt to use the Ratio Test, I have included this solution.
As you will see, it is quite a mess.

It is also important to note that, while this will tell us the series converges, it does \textbf{not} provide us with any method to compute the sum of the series.
  In order to do so, we must do as in the first solution and break the series up as a sum of two convergent geometric series.
\end{caution}

\begin{proof}[Solution 2]

  We use the Ratio Test.
  We have
  \begin{eqnarray*}
  \lim_{n \to \infty} \frac{5(3^{n+1}) + 2^{n+2}}{6^{n+1}} \cdot \frac{6^n}{5(3^{n}) + 2^{n+1}}
  &=& \lim_{n \to \infty} \frac{3^{n + 1}(5 + \frac{2^{n + 2}}{3^{n+1}})}{6(3^n)(5 + \frac{2^{n+1}}{3^n})}\\
  &=& \lim_{n \to \infty} \frac{3^{n + 1}(5 + 2 \cdot \frac{2^{n + 1}}{3^{n+1}})}{2(3^{n+1})(5 + 2 \cdot \frac{2^{n}}{3^n})}\\
  &=& \lim_{n \to \infty} \frac{(5 + 2 \cdot \left(\frac{2}{3}\right)^{n+1})}{2(5 + 2 \cdot \left(\frac{2}{3}\right)^n)}\\
  &=& \frac{5 + 2(0)}{2(5 + 2(0))}\\
  &=& \frac{5}{10}\\
  &=& \frac{1}{2}\\
  &<& 1.
  \end{eqnarray*}
  Therefore the series
  \[\sum_{n = 0}^\infty \frac{5(3^n) + 2^{n+1}}{6^n}\]
  converges by the Ratio Test.
\end{proof}
\begin{thm}[5 Points]
  Determine whether the series
  \[\sum_{n=0}^\infty \frac{e^n}{e^n + n}\]
  converges or diverges.
\end{thm}

\begin{proof}[Solution]
  We observe that
  \[\lim_{n \to \infty} \frac{e^n}{e^n + n} = \lim_{n \to \infty} \frac{e^n}{e^n} \cdot \frac{1}{1 + \frac{n}{e^n}} = \lim_{n \to \infty} \frac{1}{1 + \frac{n}{e^n}}.\]
  By L'H\^opital's rule we have
  \[\lim_{n \to \infty} \frac{n}{e^n} = \lim_{n \to \infty} \frac{1}{e^n} = 0\]
  and thus
  \[\lim_{n \to \infty} \frac{e^n}{e^n + n} = \lim_{n \to \infty} \frac{1}{1 + \frac{n}{e^n}} = \frac{1}{1 + 0} = 1.\]
  Therefore this series diverges by the \(n^\text{th}\) Term Test for Divergence.
\end{proof}

\begin{thm}[5 Points]
  Determine whether the series
  \[\sum_{n=2}^\infty \frac{\ln(n)}{n}\]
  converges or diverges.
\end{thm}

\begin{proof}[Solution]
  We use the Integral Test.
  The function \(f(x) = \ln(x)/x\) is continuous, positive, and decreasing for all \(e \leq x\).
  To prove the latter claim, observe that
  \[f^\prime(x) = \frac{\frac{1}{x} \cdot x - \ln(x)}{x^2} = \frac{1 - \ln(x)}{x^2} < 0\]
  holds whenever \(e < x\) because \(1 = \ln(e)\) and \(\ln(x)\) is an increasing function.
  
  We observe that by using the substitution \(u = \ln(x)\), \(\dif u = \dif x / x\) then
  \[\int_3^\infty \frac{\ln(x)}{x}\dif x = \lim_{t \to \infty} \int_3^t \frac{\ln(x)}{x}\dif x = \lim_{t \to \infty} \int_{\ln(3)}^{\ln(t)} u\dif u = \lim_{t \to \infty} \frac{u^2}{2}\Big|_{\ln(3)}^{\ln(t)} = \lim_{t \to \infty} \frac{\ln(t)^2 - \ln(3)^2}{2} = \infty.\]
  Therefore the series
  \[\sum_{n=2}^\infty \frac{\ln(n)}{n} = \frac{\ln(2)}{2} + \sum_{n=3}^\infty \frac{\ln(n)}{n}\]
  diverges by the Integral Test.
\end{proof}

\begin{thm}[10 Points]
  Determine whether the series
  \[\sum_{n=1}^\infty \frac{n+1}{n^2\sqrt{n}}\]
  converges or diverges.
\end{thm}

\begin{proof}[Solution]
  We use the Limit Comparison Test with the convergent \(p\)-series
  \[\sum_{n = 1}^\infty \frac{1}{n^{3/2}}.\]
  Because we are only concerned with \(1 \leq n\), we have
  \[\lim_{n \to \infty} \frac{\frac{n + 1}{n^2\sqrt{n}}}{\frac{1}{n^{3/2}}}
  = \lim_{n \to \infty} \frac{n^{3/2}(n + 1)}{n^2\sqrt{n}}
  = \lim_{n \to \infty} \frac{n + 1}{\sqrt{n}\sqrt{n}}
  = \lim_{n \to \infty} \frac{n + 1}{n} = 1.\]
  Therefore this series converges by Part~(1) of the Limit Comparison Test.
\end{proof}

\begin{thm}[10 Points]
  Determine whether the series
  \[\sum_{n = 1}^\infty (-1)^{n} \frac{n^2(n+2)!}{n! 3^{2n}}\]
  converges conditionally, converges absolutely, or diverges.
\end{thm}

\begin{proof}[Solution]
  We use the Ratio Test.
  We have
  \begin{eqnarray*}
    \lim_{n \to \infty} \frac{(n + 1)^2 (n + 3)!}{(n+1)!3^{2n+2}} \frac{n! 3^{2n}}{n^2(n + 2)!} &=& \lim_{n \to \infty} \frac{(n+1)^2}{n^2} \cdot \frac{(n+3)!}{(n + 2)!} \cdot \frac{3^{2n}}{3^{2n + 2}} \cdot \frac{n!}{(n+1)!}\\
    &=& \lim_{n \to \infty} \frac{(n+1)^2}{n^2} \cdot (n + 3) \cdot \frac{1}{3^2} \cdot \frac{1}{(n + 1)}\\
    &=& \lim_{n \to \infty} \frac{(n + 1)^2(n + 3)}{9(n + 1)n^2}\\
    &=& \lim_{n \to \infty} \frac{(n + 1)(n + 3)}{9n^2}\\
    &=& \lim_{n \to \infty} \frac{n^2 + 4n + 3}{9n^2}\\
    &=& \frac{1}{9} < 1
  \end{eqnarray*}
  Therefore this series converges absolutely by the Ratio Test.
\end{proof}

\begin{thm}[15 Points]
  Determine whether the series
  \[\sum_{n = 1}^\infty \left(\frac{4n+3}{3n-5}\right)^n\]
  converges or diverges.
\end{thm}

\begin{proof}[Solution]
  We use the Ratio Test.
  We have
  \[\lim_{n \to \infty} \sqrt[n]{\left(\frac{4n+3}{3n-5}\right)^n} = \lim_{n \to \infty} \frac{4n + 3}{3n - 5} = \frac{4}{3} > 1.\]
  Therefore this series diverges by the Root Test.
\end{proof}

\begin{thm}[15 Points]
  Determine whether the series
  \[\sum_{n = 1}^\infty (-1)^n\frac{2n}{4n^2 + 1}\]
  converges conditionally, converges absolutely, or diverges.
\end{thm}

\begin{proof}[Solution]
  First we observe that this series does not converge absolutely.
  By computing the limit
  \[\lim_{n \to \infty} \frac{\frac{2n}{4n^2 + 1}}{\frac{1}{n}} = \lim_{n \to \infty} {2n^2}{4n^2 + 1} = \frac{1}{2} > 1\]
  we see that the series
  \[\sum_{n = 1}^\infty \frac{2n}{4n^2 + 1}\]
  diverges by Part~(1) of the Limit Comparison Test because the Harmonic Series diverges.

  Next we try the Alternating Series Test.
  The first and third conditions are easy to verify: when \(1 \leq n\) it's clear that
  \[0 < u_n = \frac{2n}{4n^2 + 1}\]
  holds and also
  \[\lim_{n \to \infty} u_n = \lim_{n \to \infty} \frac{2n}{4n^2 + 1} = \lim_{n \to \infty} \frac{n^2}{n^2} \cdot \frac{2/n}{4 + 1/n^2} = \lim_{n \to \infty} \frac{2/n}{4 + 1/n^2} = \frac{0}{4 + 0} = 0.\]
  To conclude that this series converges by the Alternating Series Test, we need only verify that for some integer \(N\), \(u_{n + 1} \leq u_n\) holds whenever \(N \leq n\).
  Towards that end let \(f(x) = \frac{2x}{4x^2 + 1}\) and observe that
  \[f^\prime(x) = \frac{2(4x^2 + 1) - (2x)(8x)}{(4x^2 + 1)^2} = \frac{8x^2 + 2 - 16x}{(4x^2 + 1)^2} = \frac{-8x^2 + 2}{(4x^2 + 1)^2} < 0\]
  holds if and only if
  \[-8x^2 +2 < 0\]
  which holds if and only if
  \[\sqrt{\frac{2}{8}} = \sqrt{\frac{1}{4}} = \frac{1}{\sqrt{4}} = \frac{1}{2} < x.\]
  Since \(f\) is a decreasing function if and only if \(f^\prime(x) < 0\), we see that
  \[u_{n + 1} = f(n + 1) \leq f(n) = u_n\]
  holds whenever \(1 \leq n\).
  Therefore the series
  \[\sum_{n = 1}^\infty (-1)^n\frac{2n}{4n^2 + 1}\]
  converges conditionally.
\end{proof}

\begin{thm}[Bonus - 5 Points]
  Determine whether the series
  \[\sum_{n = 1}^\infty \sin\left(\frac{1}{n}\right)\]
  converges or diverges.
\end{thm}

\begin{proof}[Solution]
  We use the Limit Comparison Test with the Harmonic Series.
  To evaluate the limit
  \[\lim_{n \to \infty} \frac{\sin\left(\frac{1}{n}\right)}{\frac{1}{n}}\]
  we observe that if we let \(h = \frac{1}{n}\), then
  \[\lim_{n \to \infty} \frac{\sin\left(\frac{1}{n}\right)}{\frac{1}{n}} = \lim_{h \to 0} \frac{\sin(h)}{h} = 1.\]
  Therefore the series
  \[\sum_{n = 1}^\infty \sin\left(\frac{1}{n}\right)\]
  diverges by Part~(1) of the Limit Comparison Test.
\end{proof}
\end{document}
