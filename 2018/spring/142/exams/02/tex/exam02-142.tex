\documentclass[12pt]{amsart}
\usepackage{amsmath,enumerate,commath,tikz}
\usetikzlibrary{angles,quotes}
\openup 5pt
\author[Blake Farman]{Blake Farman\\University of South Carolina}
\title[Exam 02]{Math 142: Exam 02}
\date{March 20, 2018}
\pdfpagewidth 8.5in
\pdfpageheight 11in
\usepackage[margin=.5in]{geometry}

\renewcommand{\qedsymbol}{}

\begin{document}
\maketitle

\begin{center}
  \fbox{\fbox{\parbox{5.5in}{\centering
        Answer the questions in the spaces provided on the
        question sheets and turn them in at the end of the class period.

        Unless otherwise stated, all supporting work is required.
        Unsupported or otherwise mysterious answers will \textbf{not receive credit.}
        
        You may \textbf{not} use a calculator or any other electronic device, including cell phones, smart watches, etc.
        
        It is advised, although not required, that you check your answers.

        By writing your name on the line below, you indicate that you have read and understand these directions.}}}
\end{center}

\vspace{0.2in}
\makebox[\textwidth]{Name:\enspace\hrulefill}
\vspace{0.2in}

\theoremstyle{definition}
\newtheorem{thm}{}
\renewcommand{\qedsymbol}{}

\[
\begin{array}{|c|c|c||c|c|c|}
  \hline
  \text{Tests} & \text{Points Earned} & \text{Points Possible} & \text{Problems} & \text{Points Earned} & \text{Points Possible}\\
  \hline
  1 & & 3 & 1 & & 5\\
  \hline
  2 & & 3 & 2 & & 5\\
  \hline
  3 & & 2 & 3 & & 5\\
  \hline
  4 & & 4 & 4 & & 10\\
  \hline
  5 & & 4 & 5 & & 10\\
  \hline
  6 & & 7 & 6 & & 15\\
  \hline
  7 & & 2 & 7 & & 15\\
  \hline
  8 & & 4 & \text{Subtotal} & & 65\\
  \hline
  9 & & 3 &\text{Bonus} & & 5\\
  \hline
  10 & & 3 & & &\\
  \hline
  \text{Subtotal} & & 35 & \text{Total} & & 100\\
  \hline
\end{array}
\]

\newpage

\section{Tests}

Fill in the blanks.

\begin{thm}[3 Points - \(n^\text{th}\) Term Test for Divergence]
  The series \(\sum_{n = 1}^\infty a_n\) diverges if \line(1,0){150}\,,\\
  \vspace{0.10in}\\
  or \line(1,0){200}\ .
  \vspace{0.10in}\\
  The test is inconclusive if \line(1,0){200}\ .
\end{thm}

\begin{thm}[3 Points - Geometric Series]
  The series \(\sum_{n = 0}^\infty a r^n\) converges if \line(1,0){200}\\
  \vspace{0.10in}\\
  and diverges if \line(1,0){200}.
  
  \noindent If the series converges, then its sum is\\
  \[\sum_{n = 0}^\infty a r^n =\ \line(1,0){200}\]
\end{thm}

\begin{thm}[2 Points - \(p\)-series]
  The series \(\sum_{n = 1}^\infty \frac{1}{n^p}\) converges if \line(1,0){200}\\
  \vspace{.10in}\\
  and diverges if \line(1,0){200}
\end{thm}

\begin{thm}[4 Points - Integral Test]
  Let \(\{a_n\}\) be a sequence of positive terms.
  Suppose that \(f\) is a\\
  \vspace{.10in}\\
  \line(1,0){200},
  \line(1,0){200},\, and\\
  \vspace{.10in}\\
  \line(1,0){200}\ function of the variable \(x\) for all \(N \leq x\), for some positive integer \(N\).
  Suppose further that \(f(n) = a_n\) for all \(N \leq n\).
  Then the series \(\sum_{n = N}^\infty a_n\) and the integral \(\int_N^\infty f(x)\dif x\)\\
  \vspace{.10in}\\
  \line(1,0){400}
\end{thm}

\begin{thm}[4 Points - Comparison Test]
  Let \(\sum a_n\), \(\sum c_n\), and \(\sum d_n\) be series with non-negative terms.
  Suppose that for some integer \(N\)
  \[d_n \leq a_n \leq c_n\ \text{for all}\ N < n.\]
  \vspace{.10in}
  \begin{enumerate}[(a)]
  \item
    If \line(1,0){200}\ converges, then \line(1,0){200}\\
    converges.
    \vspace{.10in}
  \item
    If \line(1,0){200}\ diverges, then \line(1,0){200}\ diverges.
  \end{enumerate}
\end{thm}

\begin{thm}[7 Points - Limit Comparison Test]
  Suppose that \(0 < a_n\) and \(0 < b_n\) for all \(N \leq n\), for some positive integer \(N\).
  \vspace{.10in}
  \begin{enumerate}
  \item
    If
    \[\lim_{n \to \infty} \frac{a_n}{b_n} =\ \line(1,0){200},\,\]
    then \(\sum a_n\) and \(\sum b_n\) both converge or both diverge.
    \vspace{.10in}
  \item
    If
    \[\lim_{n \to \infty} \frac{a_n}{b_n} =\ \line(1,0){200},\,\]
    \vspace{.10in}\\
    and \line(1,0){200}\ converges, then \line(1,0){200}\ converges.\\
    %\vspace{.10in}\\
    
  \item
    If
    \[\lim_{n \to \infty} \frac{a_n}{b_n} =\ \line(1,0){200},\,\]
    \vspace{.10in}\\
    and \line(1,0){200}\ diverges, then \line(1,0){200}\\
    diverges.
  \end{enumerate}
\end{thm}

\begin{thm}[2 Points - Absolute Convergence Test]
  If the series \line(1,0){200}\ converges,\\
  \vspace{.10in}\\
  then the series \line(1,0){200}\ converges.
\end{thm}

\begin{thm}[4 Points - Ratio Test]
  Let \(\sum a_n\) be any series and suppose that
  \[\lim_{n \to \infty} \abs{\frac{a_{n+1}}{a_n}} = \rho.\]
  \vspace{.10in}
  \begin{enumerate}[(a)]
  \item
    The series converges absolutely if \line(1,0){200}
    \vspace{.10in}
  \item
    The series diverges if \line(1,0){100}\ or \line(1,0){100}
    \vspace{.10in}
  \item
    The test is inconclusive if \line(1,0){200}
  \end{enumerate}
\end{thm}

\begin{thm}[3 Points - Root Test]
  The \(\sum a_n\) be any series and suppose that
  \[\lim_{n \to \infty} \sqrt[n]{\abs{a_n}} = \rho.\]
  \vspace{.10in}
  \begin{enumerate}[(a)]
  \item
    The series converges absolutely if \line(1,0){200}
    \vspace{.10in}
  \item
    The series diverges if \line(1,0){200}
    \vspace{.10in}
  \item
    The test is inconclusive if \line(1,0){200}
  \end{enumerate}
\end{thm}

\begin{thm}[3 Points - Alternating Series Test]
  The series
  \[\sum_{n = 1}^\infty (-1)^{n+1} u_n = u_1 - u_2 + u_3 - u_4 + \ldots\]
  converges if all three of the following conditions are satisfied:
  \vspace{.10in}
  \begin{enumerate}
  \item
    \line(1,0){200}
    \vspace{.10in}
  \item
    \line(1,0){200}\ for all \(N \leq n\), for some integer \(N\).
    \vspace{.10in}
  \item
    \line(1,0){200}
  \end{enumerate}
\end{thm}

\section{Problems}
\setcounter{thm}{0}
For each of the following problems, decide which of the tests listed above is appropriate and use it to show that the given series converges or diverges.
\textbf{
  Clearly indicate the test you used.
  If the test has hypotheses, clearly indicate why the series fits the hypotheses of the test.
  Failure to do so will result in a zero on the problem.
}

\begin{thm}[5 Points]
  Determine whether the series
  \(\displaystyle{\sum_{n = 0}^\infty \frac{5(3^n) + 2^{n+1}}{6^n}}\)
  converges or diverges.
  If the series converges, find the sum of the series.
\end{thm}

\vspace{2in}

\begin{thm}[5 Points]
  Determine whether the series
  \(\displaystyle{\sum_{n=0}^\infty \frac{e^n}{e^n + n}}\)
  converges or diverges.
\end{thm}

\vspace{2in}

\begin{thm}[5 Points]
  Determine whether the series
  \(\displaystyle{\sum_{n=1}^\infty \frac{\ln(n)}{n}}\)
  converges or diverges.
\end{thm}

\newpage

\begin{thm}[10 Points]
    Determine whether the series
    \(\displaystyle{\sum_{n=1}^\infty \frac{n+1}{n^2\sqrt{n}}}\)
  converges or diverges.
\end{thm}

\vspace{2in}

\begin{thm}[10 Points]
    Determine whether the series
    \(\displaystyle{\sum_{n = 1}^\infty (-1)^{n} \frac{n^2(n+2)!}{n! 3^{2n}}}\)
    converges conditionally, converges absolutely, or diverges.
\end{thm}

\vspace{2in}

\begin{thm}[15 Points]
    Determine whether the series
    \(\displaystyle{\sum_{n = 1}^\infty \left(\frac{4n+3}{3n-5}\right)^n}\)
    converges or diverges.
\end{thm}

\newpage

\begin{thm}[15 Points]
    Determine whether the series
    \(\displaystyle{\sum_{n = 1}^\infty (-1)^n\frac{2n}{4n^2 + 1}}\)
    converges conditionally, converges absolutely, or diverges.
\end{thm}

\vspace{2in}

\begin{thm}[Bonus - 5 Points]
    Determine whether the series
    \(\displaystyle{\sum_{n = 1}^\infty \sin\left(\frac{1}{n}\right)}\)
  converges or diverges.
\end{thm}
\end{document}
