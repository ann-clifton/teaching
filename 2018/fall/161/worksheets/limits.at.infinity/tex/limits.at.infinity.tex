\documentclass[10pt]{amsart}
\usepackage{style}

\title[Limits at Infinity]{Limits at Infinity Worksheet}
\date{October 15, 2018}
\author{Blake Farman}
\address{Lafayette College}

\begin{document}
\maketitle

%\begin{abstract}
%\end{abstract}
\makenameslot

The most common type of limit at infinity one encounters in calculus is that of a rational function.
The computation of such a limit rests on the following
\begin{thm*}
  If \(0 < r\) is a rational number, then
  \[\lim_{x \to \infty} \frac{1}{x^r} = 0.\]
  If \(0 < r\) is a rational number such that \(x^r\) is defined for all \(x\), then
  \[\lim_{x \to -\infty} \frac{1}{x^r} = 0.\]
\end{thm*}

%In the second statement, the caveat ``\(x^r\) is defined for all \(x\)'' is there to exclude even roots, such as \(r = 1/2\), which all have domain \((0,\infty)\).

For example, suppose we want to evaluate the limit
\[\lim_{x \to \infty} \frac{3x^2 - x - 2}{5x^2 + 4x + 1}.\]
Both the numerator and the denominator tend to \(\infty\) as \(x\) tends to \(\infty\), so we cannot yet apply any of the Limit Laws.
However, we observe that if we factor the largest power of \(x\) out of numerator and the denominator, then we obtain
\[\lim_{x \to \infty} \frac{3x^2 - x - 2}{5x^2 + 4x + 1} = \lim_{x \to \infty} \frac{x^2(3 - \frac{1}{x} - \frac{2}{x^2})}{x^2(5 + \frac{4}{x} + \frac{1}{x^2})} = \lim_{x \to \infty} \frac{3 - \frac{1}{x} - \frac{2}{x^2}}{5 + \frac{4}{x} + \frac{1}{x^2}}.\]
The Theorem above and the Limit Laws allow us to evaluate the limit of the numerator
\[\lim_{x \to \infty} \left(3 - \frac{1}{x} - \frac{2}{x^2}\right) =
\lim_{x \to \infty}3 - \lim_{x \to \infty}\frac{1}{x} - \lim_{x \to \infty}\frac{2}{x^2} = 3 - 0 - 0 = 3\]
and of the denominator
\[\lim_{x \to \infty} \left(5 + \frac{4}{x} + \frac{1}{x^2}\right) =
\lim_{x \to \infty}5 + \lim_{x \to \infty}\frac{4}{x} + \lim_{x \to \infty}\frac{1}{x^2} = 5 + 0 + 0 = 5.\]
Since both limits exists we have
\[\lim_{x \to \infty} \frac{3x^2 - x - 2}{5x^2 + 4x + 1}
= \lim_{x \to \infty} \frac{3 - \frac{1}{x} - \frac{2}{x^2}}{5 + \frac{4}{x} + \frac{1}{x^2}}
= \frac{\lim_{x \to \infty} \left(3 - \frac{1}{x} - \frac{2}{x^2}\right)}{\lim_{x \to \infty} \left(5 + \frac{4}{x} + \frac{1}{x^2}\right)} = \frac{3}{5}.\]

\newpage
Evaluate the following limits at infinity.
\begin{multicols}{3}
  \begin{thm}
    \(\displaystyle{\lim_{x \to \infty} \frac{3x - 2}{2x + 1}}\)
  \end{thm}

  \begin{thm}
    \(\displaystyle{\lim_{x \to \infty} \frac{4x^3 + 6x^2 - 2}{2x^3 - 4x + 5}}\)
  \end{thm}

  \begin{thm}
    \(\displaystyle{\lim_{x \to \infty} \frac{\sqrt{2x^2 + 1}}{3x - 5}}\)
  \end{thm}
\end{multicols}
\newpage

When the largest power of \(x\) in the numerator is larger than that of the denominator, the method is still the same, but requires the following useful fact

\begin{thm*}
  Let \(f\) and \(g\) be numbers and let  \(0 < L\) be a constant.
  \begin{enumerate}[(a)]
  \item
    If \(\lim_{x \to \infty} f(x) = \infty\) and \(\lim_{x \to \infty} g(x) = L\), then \(\lim_{x \to \infty} f(x)g(x) = \infty\).
  \item
    If \(\lim_{x \to -\infty} f(x) = \infty\) and \(\lim_{x \to -\infty} g(x) = L\), then \(\lim_{x \to -\infty} f(x)g(x) = \infty\).
  \item
    If \(\lim_{x \to \infty} f(x) = \infty\) and \(\lim_{x \to \infty} g(x) = -L\), then \(\lim_{x \to \infty} f(x)g(x) = -\infty\).
  \item
    If \(\lim_{x \to -\infty} f(x) = \infty\) and \(\lim_{x \to -\infty} g(x) = -L\), then \(\lim_{x \to -\infty} f(x)g(x) = -\infty\).
  \end{enumerate}
\end{thm*}

For example, say we want to compute the limit
\[\lim_{x \to \infty} \frac{x^2 + x}{3 - x}.\]
We factor an \(x^2\) out of the numerator and an \(x\) out of the denominator to obtain
\[\lim_{x \to \infty} \frac{x^2 + x}{3 - x} = \lim_{x \to \infty} \frac{x^2(1 + \frac{1}{x})}{x(\frac{3}{x} - 1)} \lim_{x \to \infty} x\frac{(1 + \frac{1}{x})}{\frac{3}{x} - 1}.\]
Since \(\lim_{x \to \infty} x = \infty\) and 
\[\lim_{x \to \infty} \frac{\left(1 + \frac{1}{x}\right)}{\frac{3}{x} - 1} = \frac{\lim_{x \to \infty}\left(1 + \frac{1}{x}\right)}{\lim_{x \to \infty} \left(\frac{3}{x} - 1\right)} = \frac{\lim_{x \to \infty}1 + \lim_{x \to \infty}\frac{1}{x}}{\lim_{x \to \infty} \frac{3}{x} - \lim_{x \to \infty}1} = \frac{1 + 0}{0 - 1} = -1\]
it follows from part (c) of the Theorem that
\[\lim_{x \to \infty} \frac{x^2 + x}{3 - x} = -\infty.\]

Evaluate the following limits
\begin{multicols}{3}
  \begin{thm}
    \(\displaystyle{\lim_{x \to \infty} \frac{x + 3x^2}{4x - 1}}\)
  \end{thm}

  \begin{thm}
    \(\displaystyle{\lim_{x \to \infty} \frac{x^3 - x}{x^2 - 6x + 5}}\)
  \end{thm}


  \begin{thm}
    \(\displaystyle{\lim_{x \to \infty} \frac{x^4 - 3x^2 + x}{x^3 - x + 2}}\)
  \end{thm}

\end{multicols}
\newpage

When the highest power of \(x\) in the denominator is larger than that of the numerator, the computation is simpler.
Say we want to compute
\[\lim_{x \to \infty} \frac{3x + 2}{x^2 + 2x + 1}.\]
We factor an \(x\) out of the numerator and an \(x^2\) out of the denominator to obtain
\[\lim_{x \to \infty} \frac{3x + 2}{x^2 + 2x + 1}
= \lim_{x \to \infty} \frac{x\left(3 + \frac{2}{x}\right)}{x^2\left(1 + \frac{2}{x} + \frac{1}{x^2}\right)}
= \lim_{x \to \infty} \frac{1}{x} \frac{\left(3 + \frac{2}{x}\right)}{1 + \frac{2}{x} + \frac{1}{x^2}}
= 0 \left(\frac{3 + 0}{1 + 0 + 0}\right)
= 0 \cdot 3
= 0.\]

Evaluate the following limits.
\begin{multicols}{3}
  \begin{thm}
    \(\displaystyle{\lim_{x \to \infty} \frac{1 - x^2}{x^3 - x + 1}}\)
  \end{thm}

  \begin{thm}
    \(\displaystyle{\lim_{x \to \infty} \frac{1 + x^4}{x^6 + 1}}\)
  \end{thm}

  \begin{thm}
    \(\displaystyle{\lim_{x \to \infty} \frac{x - 2}{x^2 + 1}}\)
  \end{thm}

\end{multicols}

\newpage
Sometimes, limits at infinity of rational functions can be disguised.  For example, the limit
\[\lim_{x \to \infty} \left(\sqrt{x^2 + 1} - x\right)\]
does not appear to be the limit of a rational function.
We cannot use the Limit Laws to evaluate the limit directly because
\[\lim_{x \to \infty} \sqrt{x^2 + 1} = \infty\ \text{and}\ \lim_{x \to \infty} x = \infty,\]
which is to say \underline{\textbf{it is incorrect to write}}
\[\lim_{x \to \infty} \left(\sqrt{x^2 + 1} - x\right) = \lim_{x \to \infty} \sqrt{x^2 + 1} - \lim_{x \to \infty} x = \infty - \infty = 0!\]
To evaluate this limit, we think of this as a rational function with denominator 1 and rationalize the numerator:
\begin{gather*}
  \lim_{x \to \infty} \left(\sqrt{x^2 + 1} - x\right)
  = \lim_{x \to \infty} \frac{\sqrt{x^2 + 1} - x}{1} \left(\frac{\sqrt{x^2 + 1} + x}{\sqrt{x^2 + 1} + x}\right)
  = \lim_{x \to \infty} \frac{x^2 + 1 - x^2}{\sqrt{x^2\left(1 + \frac{1}{x^2}\right)} + x}\\
  = \lim_{x \to \infty} \frac{1}{x\sqrt{1 + \frac{1}{x^2}} + x}
  = \lim_{x \to \infty} \frac{1}{x\left(\sqrt{1 + \frac{1}{x^2}} + 1\right)}
  =\lim_{x \to \infty} \frac{1}{x}\frac{1}{\sqrt{1 + \frac{1}{x^2}} + 1}\\
  = 0 \left(\frac{1}{\sqrt{1 + 0} + 1}\right)
  = 0.
\end{gather*}

Evaluate the following limits.
\begin{multicols}{3}
  \begin{thm}
  \(\displaystyle{\lim_{x \to \infty} \left(\sqrt{9x^2 + x} - 3x\right)}\)
\end{thm}

  \begin{thm}
  \(\displaystyle{\lim_{x \to \infty} \left(\sqrt{4x^2 + 3x} + 2x\right)}\)
\end{thm}

  \begin{thm}
  \(\displaystyle{\lim_{x \to \infty} \left(x - \sqrt{x}\right)}\)
\end{thm}

\end{multicols}
\end{document}
