\documentclass[10pt]{amsart}
\usepackage{style}

\title{Antiderivatives}
\date{October 31, 2018}
\author{Blake Farman}
\address{Lafayette College}

\begin{document}
\maketitle

\makenameslot

For Problems~\eqref{first antideriv}-\eqref{last antideriv}, find the most general antiderivative of the function.
\begin{multicols}{3}
\begin{thm}\label{first antideriv}
  \(f(x) = 4x + 7\)
\end{thm}

\begin{thm}
  \(f(x) = x^2 - 3x + 2\)
\end{thm}

\begin{thm}
  \(f(x) = 6x^5 - 8x^4 - 9x^2\)
\end{thm}
\end{multicols}

\vspace{2in}

\begin{multicols}{3}
  \begin{thm}
    \(f(x) = (x - 5)^2\)
  \end{thm}

  \begin{thm}
    \(f(x) = \pi^2\)
  \end{thm}

  \begin{thm}\label{last antideriv}
    \(f(x) = \sqrt[3]{x^2} + x\sqrt{x}\)
  \end{thm}
\end{multicols}

\newpage

\begin{thm}
  Find the antiderivative \(F\) of \(f(x) = x + 2\sin(x)\) that satisfies \(F(0) = -6\).
\end{thm}

\vspace{3in}

In Problems~\eqref{first multiple antideriv}-\eqref{last multiple antideriv}, find \(f\).
\begin{multicols}{3}
  \begin{thm}\label{first multiple antideriv}
    \(f^{\prime\prime}(x) = x^6 - 4x^4 + x + 1\)
  \end{thm}

  \begin{thm}
    \(f^{\prime\prime}(x) = x^{2/3} + x^{-2/3}\)
  \end{thm}

  \begin{thm}\label{last multiple antideriv}
    \(f^{\prime\prime\prime}(t) = \sqrt{t} - 2\cos(t)\)
  \end{thm}
\end{multicols}

\newpage

\begin{thm}
  A particle is moving with velocity
  \[v(t) = t^2 - 3\sqrt{t}\]
  and initial position \(s(4) = 8\).
  Find the function, \(s\), that models the position of the particle.
\end{thm}

\vspace{3in}

\begin{thm}
  A particle is moving with the acceleration
  \[a(t) = 3\cos(t) - 2\sin(t),\]
  initial position \(s(0) = 0\), and initial velocity \(v(0) = 4\).
  Find the function, \(s\), that models the position of the particle.
\end{thm}
\end{document}

