\documentclass[10pt]{amsart}
\usepackage{style}

\title[Mean Value Theorem]{Mean Value Theorem Practice}
\date{October 12, 2018}
\author{Blake Farman}
\address{Lafayette College}

\begin{document}
\maketitle

\makenameslot
%\begin{thm*}[Rolle]
%  Let \(f\) be a function that satisfies the following hypotheses:
%  \begin{enumerate}
%  \item
%    \(f\) is continuous on the closed interval \([a,b]\).
%  \item
%    \(f\) is differentiable on the open interval \((a,b)\).
%  \item
%    \(f(a) = f(b)\).
%  \end{enumerate}
%  Then there is a number \(c\) in \((a,b)\) such that \(f^\prime(c) = 0\).
%\end{thm*}

\begin{thm*}[Mean Value]
  Let \(f\) be a function that satisfies the following hypotheses:
  \begin{enumerate}
  \item
    \(f\) is continuous on the closed interval \([a,b]\).
  \item
    \(f\) is differentiable on the open interval \((a,b)\).
  \end{enumerate}
  Then there is a number \(c\) in \((a,b)\) such that
  \[f^\prime(c) = \frac{f(b) - f(a)}{b - a}\]
  or, equivalently,
  \[f(b) - f(a) = f^\prime(c)(b - a).\]
\end{thm*}

In Problems~\ref{first MVT} through \ref{last MVT}, verify that the function satisfies the hypotheses of the Mean Value Theorem on the given interval and find all numbers \(c\) that satisfy its conclusion.
\begin{thm}\label{first MVT}
  \(f(x) = x^3 - x^2 - 6x + 2,\, [0,3]\)
\end{thm}

\vspace{1.5in}

\begin{thm}
  \(f(x) = \cos(2x),\, [\pi/8, 7\pi/8]\)
\end{thm}

\newpage

\begin{thm}
    \(f(x) = 3x^2 + 2x + 5,\, [-1,1]\)
\end{thm}
\vspace{1.5in}
\begin{thm}\label{last MVT}
  \(f(x) = \dfrac{x}{x + 2},\, [1,4]\)
\end{thm}

\vspace{1.5in}

\begin{thm}
  Let \(f(x) = 1 - x^{2/3}\).
  Show that \(f(-1) = f(1)\), but there is no number \(c\) in \((-1,1)\) such that
  \[f^\prime(c) = \frac{f(1) - f(-1)}{1 - (-1)} = \frac{0}{2} = 0.\]
  Why does this not contradict the Mean Value Theorem?
\end{thm}

\newpage

\begin{thm}
  Show that the equation \(1 + 2x + x^3 + 4x^5 = 0\) has \textit{exactly} one real solution.
\end{thm}

\vspace{4in}

\begin{thm}
  Let \(c\) be a constant.
  Show that the equation \(x^4 + 4x + c = 0\) has \textit{at most} two real roots.
\end{thm}

\newpage

\section*{Challenge}



Recall that the polynomial
\[p(x) = a_n x^n + a_{n-1} x^{n-1} + \ldots + a_1 x + a_0,\, a_n \neq 0\]
is said to have \textit{degree} \(n\), the highest power of \(x\) that appears.
Recall also that we say that a number \(r\) is a \textit{root} of \(p\) if
\[p(r) = a_nr^n + a_{n-1}r^{n-1} + \ldots a_1 r + a_0 = 0,\]
and we say that \(r\) is a \textit{real root} if \(r\) is a real number.
We want to answer the following question:
\begin{displayquote}
  ``How many real roots can a polynomial of degree \(n\) have?''
\end{displayquote}
For \(n = 1\) and \(n = 2\) we have very explicit answers to this question from a high school algebra class.
Similar explicit results are known for \(n = 3\) (Cardano's formula) and \(n = 4\), though are quite nasty and likely unfamiliar.
For \(5 \leq n\), no such formulas exist.

For a polynomial of degree one, \(p(x) = m x + b\), the answer to this question is simple: \(p\) has \textit{exactly} one real root, \(-b/m\):
\[p\left(-\frac{b}{m}\right) = m\left(-\frac{b}{m}\right) + b = -b + b = 0\]
For a polynomial of degree two, \(p(x) = a x^2 + b x + c\), the answer to this question also has a relatively simple answer: \(p\) has \textit{at most} two real roots
\[\frac{-b \pm \sqrt{b^2 - 4ac}}{2a}.\]
These two answers lead us to believe that a degree \(n\) polynomial should have \textit{at most} \(n\) real roots.
%In this case, the number of real roots are determined by the discriminant, \(D = b^2 - 4ac\), as follows.\\

%\noindent
%\(\underline{\textbf{D = 0:}}\) \(p\) has exactly one real root
%\[r = \frac{-b \pm \sqrt{D}}{2a} = \frac{-b \pm \sqrt{0}}{2a} = -\frac{b}{2a}.\]

%\noindent
%\(\underline{\textbf{0 \textless\ D:}}\) there are exactly two real roots
%\[r_1 = \frac{-b + \sqrt{D}}{2a}\ \text{and}\ r_2 = \frac{-b - \sqrt{D}}{2a}.\]\\

%\noindent
%\(\underline{\textbf{D \textless\ 0:}}\) there are no real roots because \(\sqrt{D}\) is not a real number (there are no real numbers that square to a negative number).

First we investigate for a differentiable function, \(f\), how the number of solutions to \(f(x) = 0\) relates to the number of solutions to \(f^\prime(x) = 0\).
We observe that if there exist numbers \(r_1 < r_2\) such that \(f(r_1) = f(r_2) = 0\), then by the Mean Value Theorem applied on the interval \([r_1, r_2]\) there exists a number \(s\) in \((r_1, r_2)\) such that
\[f^\prime(s) = \frac{f(r_2) - f(r_1)}{r_2 - r_1} = \frac{0}{r_2 - r_1} = 0.\]

\begin{thm}\label{roots of deriv}
  Assume that there exist numbers \(r_1 < r_2 < r_3 < r_4\) such that \(f(r_1) = f(r_2) = f(r_3) = f(r_4) = 0\).
  Show that there are three numbers \(s_1 < s_2 < s_3\) such that \(f^\prime(s_1) = f^\prime(s_2) = f^\prime(s_3) = 0\).
\end{thm}
\newpage

\noindent We can now use this result to answer our original question in the case that \(n = 3\):
\begin{thm}\label{FTA}
  Show that a polynomial of degree three
  \[p(x) = a_3x^3 + a_2x^2 + a_1x +a_0\]
  has \textit{at most} three real roots.
  That is to say, there are at most three numbers \(r_1 < r_2 < r_3\) such that
  \[p(r_1) = p(r_2) = p(r_3) = 0.\]
     {[Hint: According to Problem~\ref{roots of deriv}, what would happen if \(p\) had four real roots?]}
\end{thm}

\vspace{2in}

\begin{thm}
  Generalize your result from Problem~\ref{roots of deriv}:\\

  \noindent Assume that for some integer \(m\) there exist numbers
  \(r_1 < r_2 < \ldots < r_m\)
    such that
    \[f(r_1) = f(r_2) = \ldots = f(r_m) = 0.\]
    Show that there are \(m-1\) numbers
    \(s_1 < s_2 < \ldots < s_{m-1}\)
    such that
    \[f^\prime(s_1) = f^\prime(s_2) = \ldots = f^\prime(s_{m-1}).\]
\end{thm}
\newpage
\begin{thm}
  Generalize your solution to Problem~\ref{FTA}:\\
  
  \noindent Show that a polynomial of degree \(n\) has \textit{at most} \(n\) real roots.
\end{thm}
\end{document}

