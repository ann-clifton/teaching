\documentclass[10pt]{amsart}
\usepackage{style}

\title[Mean Value Theorem]{Mean Value Theorem Practice}
\date{October 12, 2018}
\author{Blake Farman}
\address{Lafayette College}

\begin{document}
\maketitle

\makenameslot
%\begin{thm*}[Rolle]
%  Let \(f\) be a function that satisfies the following hypotheses:
%  \begin{enumerate}
%  \item
%    \(f\) is continuous on the closed interval \([a,b]\).
%  \item
%    \(f\) is differentiable on the open interval \((a,b)\).
%  \item
%    \(f(a) = f(b)\).
%  \end{enumerate}
%  Then there is a number \(c\) in \((a,b)\) such that \(f^\prime(c) = 0\).
%\end{thm*}

\begin{thm*}[Mean Value]
  Let \(f\) be a function that satisfies the following hypotheses:
  \begin{enumerate}
  \item
    \(f\) is continuous on the closed interval \([a,b]\).
  \item
    \(f\) is differentiable on the open interval \((a,b)\).
  \end{enumerate}
  Then there is a number \(c\) in \((a,b)\) such that
  \[f^\prime(c) = \frac{f(b) - f(a)}{b - a}\]
  or, equivalently,
  \[f(b) - f(a) = f^\prime(c)(b - a).\]
\end{thm*}

In Problems~\ref{first MVT} through \ref{last MVT}, verify that the function satisfies the hypotheses of the Mean Value Theorem on the given interval and find all numbers \(c\) that satisfy its conclusion.
\begin{thm}\label{first MVT}
  \(f(x) = x^3 - x^2 - 6x + 2,\, [0,3]\)
\end{thm}

\begin{thm}
  \(f(x) = \cos(2x),\, [\pi/8, 7\pi/8]\)
\end{thm}

\begin{thm}
    \(f(x) = 3x^2 + 2x + 5,\, [-1,1]\)
\end{thm}

\begin{thm}\label{last MVT}
  \(f(x) = \dfrac{x}{x + 2},\, [1,4]\)
\end{thm}

\begin{thm}
  Let \(f(x) = 1 - x^{2/3}\).
  Show that \(f(-1) = f(1)\), but there is no number \(c\) in \((-1,1)\) such that
  \[f^\prime(c) = \frac{f(1) - f(-1)}{1 - (-1)} = \frac{0}{2} = 0.\]
  Why does this not contradict the Mean Value Theorem?
\end{thm}

\begin{thm}
  Show that the equation \(1 + 2x + x^3 + 4x^5 = 0\) has \textit{exactly} one real solution.
\end{thm}

\section*{Challenge}

Recall that a polynomial
\[p(x) = a_n x^n + a_{n-1} x^{n-1} + \ldots + a_1 x + a_0,\, a_n \neq 0\]
is said to have \textit{degree} \(n\), the highest power of \(x\) that appears.
We say that a number \(r\) is a \textit{root} of \(p\) if
\[p(r) = a_nr^n + a_{n-1}r^{n-1} + \ldots a_1 r + a_0 = 0\].

A degree one polynomial, \(p(x) = m x + b\), always has \textit{exactly} one real root,
\[p\left(-\frac{b}{m}\right) = m\left(-\frac{b}{m}\right) + b = -b + b = 0\]
while a degree two polynomial, \(p(x) = a x^2 + b x + c\), has \textit{at most} two real roots
\[\frac{-b \pm \sqrt{b^2 - 4ac}}{2a}.\]
In the latter case, the number of roots are determined by the discriminant, \(D = b^2 - 4ac\).
If \(D = 0\), then \(p\) has exactly one real root
\[r = \frac{-b \pm \sqrt{D}}{2a} = \frac{-b \pm \sqrt{0}}{2a} = -\frac{b}{2a}.\]
If \(0 < D\) then there are exactly two real roots
\[r_1 = \frac{-b + \sqrt{D}}{2a}\ \text{and}\ r_2 = \frac{-b - \sqrt{D}}{2a}.\]
If \(D < 0\), then \(\sqrt{D}\) is not a real number (there are no real numbers that square to a negative), so there are no real roots.


\begin{thm}\label{FTA}
  Show that a polynomial of degree three
  \[p(x) = a_3x^3 + a_2x^2 + a_1x +a_0\]
  has \textit{at most} three real roots.
  That is to say, there are at most three numbers \(r_1 < r_2 < r_3\) such that
  \[p(r_1) = p(r_2) = p(r_3) = 0.\]
     {[Hint: How does the number of roots of \(p(x)\) affect the number of roots of \(p^\prime(x)\)?]}
\end{thm}

\begin{thm}
  Generalize your solution to Problem~\ref{FTA}.  That is:\vspace{\baselineskip}\\
  
  \noindent Show that a polynomial of degree \(n\) has \textit{at most} \(n\) real roots.
\end{thm}
\end{document}

