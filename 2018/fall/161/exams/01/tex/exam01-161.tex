\documentclass[12pt]{amsart}
\usepackage{style}

\author{Blake Farman}
\address{Lafayette College}
\title[Exam 1]{Exam 1\\Math 161}
\date{September 26, 2018}

\begin{document}
\maketitle

\begin{center}
  \fbox{\fbox{\parbox{5.5in}{\centering
        \noindent Answer the questions in the spaces provided on the
        question sheets and turn them in at the end of the exam period.\\


        \noindent It is advised, although not required, that you check your answers.
        All supporting work is required.
        Unsupported or otherwise mysterious answers will \textbf{not receive credit.}\\
        
        \noindent You may \textbf{not} use a calculator or any other electronic device, including cell phones, smart watches, etc.\\
        
        \noindent By writing your name on the line below, you indicate that you have read and understand these directions.}}}
\end{center}

\vspace{0.2in}
\makebox[\textwidth]{Name:\enspace\hrulefill}
\vspace{0.2in}

\theoremstyle{definition}
\newtheorem{thm}{}
\renewcommand{\qedsymbol}{}

\[
\begin{array}{|c|c|c|}
  \hline
  \text{Problem} & \text{Points Earned} & \text{Points Possible}\\
  \hline
  1 & & 6\\
  \hline
  2 & & 2\\
  \hline
  3 & & 1\\
  \hline
  4 & & 8\\
  \hline
  5 & & 5\\
  \hline
  6 & & 5\\
  \hline
  7 & & 3\\
  \hline
  8 & & 10\\
  \hline
  9 & & 10\\
  \hline
  10 & & 5\\
  \hline
  11 & & 15\\
  \hline
  12 & & 15\\
  \hline
  13 & & 15\\
  \hline
  \text{Total} & & 100\\
  \hline
\end{array}
\]


\newpage
\section*{Fill In the Blank}

\begin{thm}[6 Points - Limit Laws]
  Suppose that \(c\) is a constant and the limits
  \[\lim_{x \to a} f(x) = L\ \text{and}\ \lim_{x \to a} g(x) = M\]
  exist.
  Then
    \begin{enumerate}
    \item
      \(\lim_{x \to a} \left[f(x) + g(x)\right] =\ \line(1,0){150}\)
      \vspace{.25in}
    \item
      \(\lim_{x \to a} \left[f(x) - g(x)\right] =\ \line(1,0){150}\)
      \vspace{.25in}
    \item
      \(\lim_{x \to a} [c(fx)] =\ \line(1,0){150}\)
      \vspace{.25in}
    \item
      \(\lim_{x \to a} \left[f(x)g(x)\right] =\ \line(1,0){150}\)
      \vspace{.25in}
    \item
      \(\lim_{x \to a} \frac{f(x)}{g(x)} =\ \line(1,0){150},\, \text{if}\ \line(1,0){150}\)
      \vspace{.25in}
    \end{enumerate}
\end{thm}

\begin{thm}[2 Points]
  A function \(f\) is continuous at a number \(a\) if
  \vspace{.15in}
  \[\line(1,0){150}\ =\ \line(1,0){150}.\]
\end{thm}

\begin{thm}[1 Point]
  The derivative of the function \(f(x)\) is the function
  \vspace{.25in}
  \[f^\prime(x) = \lim_{h \to 0}\ \line(1,0){150}\]
\end{thm}

\begin{thm}[8 Points - Derivative Rules]
  Let \(c\) be a constant.
  If \(f\) and \(g\) are differentiable functions, then
  \begin{enumerate}
  \item
    \(\frac{\dif}{\dif x}\left(c\right) =\ \line(1,0){150}\)
    \vspace{.25in}
  \item
    \(\frac{\dif}{\dif x}\left(x^n\right) =\ \line(1,0){150}\)
    \vspace{.25in}
  \item
    \(\frac{\dif}{\dif x}\left(cf(x)\right) =\ \line(1,0){150}\)
    \vspace{.25in}
  \item
    \(\frac{\dif}{\dif x}\left(f(x) + g(x)\right) =\ \line(1,0){150}\)
    \vspace{.25in}
  \item
    \(\frac{\dif}{\dif x}\left(f(x) - g(x)\right) =\ \line(1,0){150}\)
    \vspace{.25in}
  \item
    \(\frac{\dif}{\dif x}\left(f(x)g(x)\right) =\ \line(1,0){200}\)
    \vspace{.25in}
  \item
    \(\frac{\dif}{\dif x}\left(\frac{f(x)}{g(x)}\right) =\ \line(1,0){200}\)
    \vspace{.25in}
  \item
    \(\frac{\dif}{\dif x}\left(f \circ g (x)\right) = \frac{\dif}{\dif x} f(g(x)) =\ \line(1,0){200}\)
  \end{enumerate}
\end{thm}

\section*{Problems}
\noindent
Use the function \(\displaystyle{f(x) = \frac{x^2 - 9}{x^2 + 2x - 3}}\) to answer Problems~\ref{removable discont}-\ref{fix discont}.

\begin{thm}[5 Points]\label{removable discont}
  Use the Limit Laws to compute \(\lim_{x \to -3} f(x)\).
\end{thm}

\vspace{2in}

\begin{thm}[5 Points]\label{infinite limit}
  Use the Limit Laws to compute \(\lim_{x \to 1} f(x)\).
\end{thm}

\vspace{2in}

\begin{thm}[3 Points]\label{fix discont}
    Find the value \(L\) that makes the function
    \[g(x) = \left\{\begin{matrix}
    f(x) & \text{if}\ x \neq -3,\\
    L & \text{if}\ x = -3
    \end{matrix}
    \right.\]
    continuous at \(x = -3\).
\end{thm}

\newpage

\begin{thm}[10 Points]
  Use the Limit Laws to compute
  \[\lim_{x \to 16} \frac{4 - \sqrt{x}}{x - 16}.\]
\end{thm}

\vspace{3in}

\noindent Use \(f(x) = 3x^2 + 1\) to answer Problems~\ref{Diff by def} and \ref{check diff by def}.

\begin{thm}[10 Points]\label{Diff by def}
  Compute \(f^\prime(x)\) by the \textbf{definition}.
\end{thm}

\vspace{3in}

\begin{thm}[5 Points]\label{check diff by def}
  Use the derivative rules to check that your answer to part (a) is correct.
\end{thm}

\newpage

\noindent In the following problems, find the equation of the line tangent to the given curve at the given point.
\textbf{Do not compute the derivative from the definition!}
\begin{thm}[15 Points]
  \(\displaystyle{f(x) = (x^4 - 3x^2 + 5)^3}\); \((0,125)\).
\end{thm}

\vspace{2in}
\begin{thm}[15 Points]
  \(\displaystyle{g(x) = \frac{x}{1 - x^2}}\); \((2,1)\).
\end{thm}

\vspace{2in}

\begin{thm}[15 Points]
  \(\displaystyle{h(x) = x^3\sin(2x)}\); \((\pi,0)\).
\end{thm}

\end{document}
