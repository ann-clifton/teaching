\documentclass[12pt]{amsart}
\usepackage{style}

\author{Blake Farman}
\address{Lafayette College}
\title[Exam 1]{Math 161: Exam 1}
\date{September 26, 2018}

\begin{document}
\maketitle

\begin{center}
  \fbox{\fbox{\parbox{5.5in}{\centering
        Answer the questions in the spaces provided on the
        question sheets and turn them in at the end of the class period.

        Unless otherwise stated, all supporting work is required.
        Unsupported or otherwise mysterious answers will \textbf{not receive credit.}
        
        You may \textbf{not} use a calculator or any other electronic device, including cell phones, smart watches, etc.
        By writing your name on the line below, you indicate that you have read and understand these directions.


        It is advised, although not required, that you check your answers.}}}
\end{center}

\vspace{0.2in}
\makebox[\textwidth]{Name:\enspace\hrulefill}
\vspace{0.2in}

\theoremstyle{definition}
\newtheorem{thm}{}
\renewcommand{\qedsymbol}{}

\[
\begin{array}{|c|c|c|}
  \hline
  \text{Problem} & \text{Points Earned} & \text{Points Possible}\\
  \hline
  1 & & 10\\
  \hline
  2 & & 10\\
  \hline
  3 & & 10\\
  \hline
  4 & & 10\\
  \hline
  5 & & 10\\
  \hline
  6 & & 10\\
  \hline
  7 & & 10\\
  \hline
  8 & & 10\\
  \hline
  9 & & 10\\
  \hline
  10 & & 10\\
  \hline
  \text{Total} & & 100\\
  \hline
\end{array}
\]


\newpage
\section*{Fill In the Blank}

\begin{thm}[3 Points]
  Suppose \(f(x)\) is defined when \(x\) is near the number \(a\).
  We write
  \[\lim_{x \to a} f(x) = L\]
  if we can make the values of \(f(x)\)\ \line(1,0){150}\ by restricting \(x\) to be \vspace{.5in}\\
  \line(1,0){150}\,, but not equal to \line(1,0){150}.
\end{thm}

\begin{thm}[6 Points]
  Suppose that \(c\) is a constant and the limits
  \[\lim_{x \to a} f(x) = L\ \text{and}\ \lim_{x \to a} g(x) = M\]
  exist.
  Then
  \begin{enumerate}
  \item
    \(\displaystyle{\lim_{x \to a} \left[f(x) + g(x)\right] =\ \line(1,0){150}}\)
    \vspace{.25in}
  \item
    \(\displaystyle{\lim_{x \to a} \left[f(x) - g(x)\right] =\ \line(1,0){150}}\)
    \vspace{.25in}
  \item
    \(\displaystyle{\lim_{x \to a} [c(fx)] =\ \line(1,0){150}}\)
    \vspace{.25in}
  \item
    \(\displaystyle{\lim_{x \to a} \left[f(x)g(x)\right] =\ \line(1,0){150}}\)
    \vspace{.25in}
  \item
    \(\displaystyle{\lim_{x \to a} \frac{f(x)}{g(x)} =\ \line(1,0){150},\, \text{if}\ \line(1,0){150}}\)
    \vspace{.25in}
  \end{enumerate}
\end{thm}

\begin{thm}[2 Points]
  A function \(f\) is continuous at a number \(a\) if
  \vspace{.25in}
  \[\line(1,0){150}\ =\ \line(1,0){150}.\]
\end{thm}

\begin{thm}[1 Point]
  Suppose that \(f\) is continuous on the closed interval \([a,b]\).
  If \(N\) is any number between \(f(a)\) and \(f(b)\), then there exists some number \(a < c < b\) such that
  \vspace{.25in}
  \[\line(1,0){150}.\]
\end{thm}

\begin{thm}[1 Point]
  The derivative of the function \(f\) is
  \vspace{.5in}
  \[f^\prime(x) = \lim_{h \to 0}\ \line(1,0){150}\]
\end{thm}

\vspace{.25in}

\begin{thm}[7 Points]
  Let \(c\) be a constant.
  If \(f\) and \(g\) are differentiable functions, then
  \begin{enumerate}
  \item
    \(\displaystyle{\frac{\dif}{\dif x}(c) =\ \line(1,0){150}}\)
    \vspace{.25in}
  \item
    \(\displaystyle{\frac{\dif}{\dif x}(x^n) =\ \line(1,0){150}}\)
    \vspace{.25in}
  \item
    \(\displaystyle{\frac{\dif}{\dif x}(cf(x)) =\ \line(1,0){150}}\)
    \vspace{.25in}
  \item
    \(\displaystyle{\frac{\dif}{\dif x}(f(x) + g(x)) =\ \line(1,0){150}}\)
    \vspace{.25in}
  \item
    \(\displaystyle{\frac{\dif}{\dif x}(f(x) - g(x)) =\ \line(1,0){150}}\)
    \vspace{.25in}
  \item
    \(\displaystyle{\frac{\dif}{\dif x}(f(x)(g(x)) =\ \line(1,0){200}}\)
    \vspace{.25in}
  \item
    \(\displaystyle{\frac{\dif}{\dif x}\left(\frac{f(x)}{g(x)}\right) =\ \line(1,0){200}}\)
  \end{enumerate}
\end{thm}

\section*{Problems}
\begin{thm}[10 Points]
  Consider the function
  \[f(x) = \frac{x^2 + 3x}{x^2 - x - 12}.\]
  \begin{enumerate}[(a)]
  \item
    This function has a discontinuity at \(x = 3\).
    \textit{Briefly} explain why.
  \item
    Write down a function, \(g(x)\), that agrees with \(f(x)\) everywhere except at \(x = 3\) (that is to say, g(x) = f(x), except when x = 3).
  \item
    This function also has a discontinuity at \(x = 4\).
    \textit{Briefly} explain why.
    Can you \lq fix\rq\ this discontinuity in the same way as in part (b)?
  \end{enumerate}
    
\end{thm}

\begin{thm}[10 Points]
\end{thm}


\begin{thm}[10 Points]
\end{thm}

\begin{thm}[10 Points]
\end{thm}



\begin{thm}[10 Points]
\end{thm}

\begin{thm}[10 Points]
\end{thm}



\begin{thm}[10 Points]
\end{thm}

\begin{thm}[10 Points]
\end{thm}

\begin{thm}[10 Points]
\end{thm}

\begin{thm}[10 Points]
\end{thm}

\end{document}
